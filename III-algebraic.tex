\chapter{Locally algebraic abelian representations}
\label{ch:iii}
\chaptermark{Locally algebraic representations}

In this Chapter, we define what it means for an abelian $\ell$-adic
representation to be \emph{locally algebraic} and we prove (cf.\ \ref{sec:III_23}) that such
a representation, when rational, comes from a linear representation
of one of the groups $S_{\mathfrak{m}}$ of Chapter \ref{ch:ii}.

When the ground field is a composite of quadratic extensions of
$\Q$, any rational semi-simple $\ell$-adic representation is \emph{ipso facto}
locally algebraic; this is proved in \S\ref{sec:III_3}, as a consequence of a result
on transcendental numbers due to Siegel and Lang.

In the local case, an abelian semi-simple representation is
locally algebraic if and only if it has a ``Hodge-Tate decomposition''.
This fact, due to Tate (College de France, 1966), is proved in the
Appendix, together with some complements.

\section{The local case}

\subsection{Definitions}
\label{sec:III_11}
Let $p$ be a prime number and $K$ a finite extension of $\Q_p$; let $\TT =
\WRes_{K/\Q_p}(\GG_{m, K})$ be the corresponding algebraic torus over
\dpage
$\Q_p$ (cf.\ \citeauthor{43}~\cite{43}, Chap.~I).
\todo{Belen.}

\subsection{Alternative definition of ``locally algebraic'' via Hodge-Tate
modules}
\label{sec:III_12}
Let us recall first the notion of a \strong{Hodge-Tate module} (cf.\ \cite{27},
\S 2); here $K$ is only assumed to be complete with respect to a discrete
valuation, with perfect residue field $k$ and $\char(K) = 0$, $\char(k) = p$.
Denote by $C$ the \emph{completion $\widehat{\algcl K}$ of the algebraic
closure} of $K$.

The group $G = \Gal(\algcl K/K)$ acts continuously on $K$. This action extends
continuously to $C$. Let $W$ be a $C$-vector space of finite dimension upon
which $G$ acts continuously and semi-linearly according to the formula
\[
	s(cw) = s(c) \cdot s(w) \qquad
	(s \in G, \, c \in C \text{ and } w \in W).
\]
Let $\chi\colon G \to U_p$ be the homomorphism of $G$ into the group $U_p =
\Z_p^\times$ of $p$-adic units, defined by its action on the $p^\nu$-th roots
of unity (cf.\ chap.~\ref{ch:i}, \ref{sec:I_12}):
\dpage
\[
	s(z) = z^{\chi(s)} \qquad \text{if } s \in G \text{ and } z^{p^\nu} = 1.
\]
Define for every $i \in \Z$ the subspace
\[
	W^i = \{ w \in W : sw = \chi(s)^i w \text{ for all } s\in G \}
\]
of $W$. This is a $K$-vector subspace of $W$. Let $W(i) = C \otimes_K W^i$.
This is a $C$-vector space upon which $G$ acts in a natural way (i.e.\ by the
formula $s(c \otimes y) = s(c) \otimes s(y)$). The inclusion $W^i \to W$
extends uniquely to a $C$-linear map $\alpha_i\colon W(i) \to W$, which
commutes with the action of $G$.

\begin{prop}[Tate]
	Let $\coprod_{i\in \Z} W(i)$ be the direct sum of the $W(i)$. Let
	$\alpha\colon \coprod_i W(i) \to W$ be the sum of the $\alpha_i$'s
	defined above. Then $\alpha$ is injective.
\end{prop}
For the proof see \cite{27}, \S 2, prop.~4.
\begin{corp}
	The $K$-spaces $W^i$ ($i \in \Z$) are of finite dimension.
	They are linearly independent over $C$.
\end{corp}

\begin{mydef}\label{def:III_12_1}
	The module $W$ is of \strong{Hodge-Tate type}\index{Hodge-Tate module}%
	\index{Hodge-Tate type (module)}
	if the homomorphism $\alpha\colon \coprod_{i\in\Z} W(i) \to W$ is an
	isomorphism.
\end{mydef}

Let now $V$ be as in \ref{sec:III_11}, a vector space over $\Q_p$, of finite
dimension. Let $\rho\colon G \to \Aut(V)$ be a $p$-adic representation. Let $W
= C \otimes_{\Q_p} V$ and let $G$ act on $W$ by the formula
\dpage
\[
	s(c\otimes v) = s(c) \otimes s(v) \qquad
	s\in \Gamma, \; c\in C, \; v\in V.
\]
\begin{mydef}
	The representation $\rho$ is of \strong{Hodge-Tate type}%
	\index{Hodge-Tate representation}\index{Hodge-Tate type
	(representation)} if the $C$-space $W = C \otimes_{\Q_p} V$ is of
	Hodge-Tate type (cf.\ def.~\ref{def:III_12_1}).
\end{mydef}

\begin{ex}
	Let $F$ be a $p$-divisible group of finite height (cf.\ \cite{26},
	\cite{39}); let $T$ be its Tate module (\emph{loc. cit.}) and $V = \Q_p
	\otimes T$. The group $G$ acts on $V$, and Tate has proved (\cite{39},
	Cor.~2 to Th.~3) that this Galois module is of Hodge-Tate type; more
	precisely, one has $W = W(0) \oplus W(1)$, where $W = C \otimes V$ as
	above.
\end{ex}
\begin{thm}[Tate]
Assume $K$ is a finite extension of $\Q_p$ (i.e.\ its residue field is finite).
Let $\rho\colon G \to \Aut(V)$ be an abelian $p$-adic representation of $K$.
The following properties are equivalent:
\begin{enumerate}[(a)]
\item $\rho$ is locally algebraic (cf.\ \ref{sec:III_11}).
\item $\rho$ is of Hodge-Tate type and its restriction to the inertia group is
	semi-simple.
\end{enumerate}
\end{thm}
For the proof, see the Appendix.

\section{The global case}
\subsection{Definitions}
\label{sec:III_21}
\todo{Belen.}

\subsection{Modulus of a locally algebraic abelian representation}
\label{sec:III_22}
Let $\rho\colon \abcl{\Gal(\algcl K/K)} \to \Aut(V_\ell)$ be as above; by
composition with the class field homomorphism $i\colon I \to \abcl{\Gal(\algcl
K/K)}$, $\rho$ defines a homomorphism $\rho \circ i\colon I \to \Aut(V_\ell)$.

We assume that $p\rho$ is locally algebraic and we denote by $f$ the associated
\dpage
algebraic morphism $T_{/\Q_\ell} \to \GL_{V_\ell}$.
\begin{mydef}
Let $\mathfrak{m}$ be a modulus (chap.~\ref{ch:ii}, \ref{sec:II_11}).
One says that $\rho$ is defined mod $\mathfrak{m}$ (or that
$\mathfrak{m}$ is a modulus of definition for $\rho$) if
\begin{enumerate}[(i)]
	\item $\rho \circ i$ is trivial on $U_{v, \mathfrak{m}}$ when $p_v \ne
		\ell$.
	\item $\rho \circ i_\ell(x) = f(x^{-1})$ for \smash{$\displaystyle x
		\in \prod_{v\mid\ell} U_{v, \mathfrak{m}}$}.
\end{enumerate}
\end{mydef}
(Note that $\prod_{v\mid\ell} U_{v, \mathfrak{m}}$ is an open subgroup of
$K_\ell^\times = T_{/\Q_\ell}(\Q_\ell)$.)

In order to prove the existence of a modulus of definition, we
need the following auxiliary result:
\begin{prop}\label{prop:III_22_1}
Let $H$ be a Lie group over $Q_\ell$ (resp.\ $\R$) and let
$\alpha$ be a continuous homomorphism of the idèle group $I$ into $H$.
\begin{enumerate}[(a)]
\item\label{prop:III_22a}
	If $p_v \ne \ell$ (resp.\ $p_v \ne \infty$), the restriction of
	$\alpha$ to $K$ is equal to 1 on an open subgroup of $K_v^\times$.
\item\label{prop:III_22b}
	The restriction of $\alpha$ to the unit group $U_v$ of $K_v^\times$ is
	equal to 1 for almost all $v$'s.
\end{enumerate}
\end{prop}
\begin{proof}
	Part \ref{prop:III_22a} follows from the fact that $K_v^\times$ is a
	$p_v$-adic Lie group and that a homomorphism of a $p$-adic Lie group
	into an $\ell$-adic one is locally equal to 1 if $p \ne \ell$.

	To prove \ref{prop:III_22b}, let $N$ be a neighborhood of 1 in $H$
	which contains no finite subgroup except $\{ 1 \}$; the existence of
	such an $N$ is classical for real Lie groups, and quite easy to prove
	for $\ell$-adic ones. By definition of the idèle topology,
	$\alpha(U_v)$ is contained in $N$ for almost all $v$'s. But
	\ref{prop:III_22a} shows that, if $p_v \ne \ell$, the group
	\dpage
	$\alpha(U_v)$ is finite; hence $\alpha(U_v) = \{ 1 \}$ for almost all
	$v$'s.
\end{proof}

\begin{corp}\label{cor:III_22}
	Any abelian $\ell$-adic representation of $K$ is unramified outside a
	finite set of places.
\end{corp}
This follows from \ref{prop:III_22b} applied to the homomorphism $\alpha$ of $I$
induced by the given representation, since the $\alpha(U_v)$ are known to be
the inertia subgroups.
% 2d by the given r

\begin{obs}
This does not extend to non-abelian representations (even solvable ones), cf.\ Exercise.
\end{obs}
\begin{prop}
	Every locally algebraic abelian $\ell$-adic representation has a
	modulus of definition.
\end{prop}
Let $\rho\colon \abcl{\Gal(\algcl K/K)} \to \Aut(V_\ell)$ be the given
representation and $f$ the associated morphism of $T_{/\Q_\ell}$ into
$\GL_{V_\ell}$. Let $X$ be the set of places $v \in M_K^0$, with $p_v \ne
\ell$, for which $\rho$ is ramified; the corollary~\ref{cor:III_22} to
Prop.~\ref{prop:III_22_1} shows that $X$ is finite. By
Prop.~\ref{prop:III_22_1}, \ref{prop:III_22a}, we can choose a modulus
$\mathfrak{m}$ such that $\rho \circ i\colon I \to \Aut(V_\ell)$ is trivial on
all the $U_{v, \mathfrak{m}}$, $v \in X$. Enlarging $\mathfrak{m}$ if
necessary, we can assume that $\rho \circ i_\ell(x) = f(x^{-1})$ for $x \in
\prod_{p_v = \ell} U_{v, \mathfrak{m}}$. Hence, $\mathfrak{m}$ is a modulus of
definition for $\rho$.

\begin{obs}
It is easy to show that there is a smallest modulus of definition for $\rho$;
it is called the \strong{conductor}\index{Conductor} of $\rho$.
\end{obs}

\subsubsection*{Exercise}
Let $z_1, \dots, z_n, \dots \in K^\times$. For each $n$, let $E_n$ be the
\dpage
subfield of $\algcl K$ generated by all the $\ell^n$-th roots of the element
$z_1 z_2^\ell \cdots z_n^{\ell^{n-1}}$.
\begin{enumerate}[a)]
\item Show that $E_n$ is a Galois extension of $K$, containing the $\ell^n$-th
	roots of unity and that its Galois group is isomorphic to a subgroup of
	the affine group $
	\begin{psmallmatrix}
		* & * \\
		0 & 1
	\end{psmallmatrix} 
	$ in $\GL(2, \Z/\ell^n\Z)$.
\item Let $E$ be the union of the $E_n$'s. Show that $E$ is a Galois extension
	of $K$, whose Galois group is a closed subgroup of the affine group
	relative to $\Z_\ell$.
\item Give an example where $E$ (and hence the corresponding 2-dimensional
	$\ell$-adic representation) is ramified at all places of $K$.
\end{enumerate}

\subsection{Back to \texorpdfstring{$S_{\mathfrak{m}}$}{Sm}}
\label{sec:III_23}
\todo{Belen.}

\subsection{A mild generalization}
\label{sec:III_24}
\todo{Belen.}

\subsection{The function field case}
\label{sec:III_25}
The above constructions have a (rather elementary) analogue
for \emph{function fields of one variable over a finite field:}

Let $K$ be such a field, and let $p$ be its characteristic. If $\mathfrak{m}$
is a modulus for $K$ (i.e.\ a positive divisor) we define the subgroup
$U_{\mathfrak{m}}$ of the idèle group $I$ as in chap.~\ref{ch:ii},
\ref{sec:II_21}, and we put
\[
	\Gamma_{\mathfrak{m}} = I/U_{\mathfrak{m}} K^\times.
\]
\dpage
The degree map $\deg\colon I \to \Z$ is trivial on $U_{\mathfrak{m}}$, hence defines an
exact sequence
\[\begin{tikzcd}
	1 \rar & J_{\mathfrak{m}} \rar & \Gamma_{\mathfrak{m}} \rar & \Z \rar & 1.
\end{tikzcd}\]
One sees easily that the group $J_{\mathfrak{m}}$ is finite; moreover, it may
be interpreted as the group of rational points of the ``generalized Jacobian
variety defined by $\mathfrak{m}$''. If $\widehat{\Gamma}_{\mathfrak{m}}$
denotes the completion of r with respect to the topology of subgroups of finite
index, it is known (class field theory) that $\abcl{\Gal(\algcl K/K)} \cong
\invlim_{\mathfrak{m}} \widehat{\Gamma}_{\mathfrak{m}}$.

Let now $\rho\colon \abcl{\Gal(\algcl K/K)} \to \Aut(V_\ell)$ be an abelian
$\ell$-adic representation of $K$, with $\ell \ne p$. One proves as in
\ref{sec:III_22} that there exists a modulus $\mathfrak{m}$ such that $\rho$ is
trivial on $U_{\mathfrak{m}}$, i.e.\ such that $\rho$ may be identified with a
\emph{homomorphism of $\widehat{\Gamma}_{\mathfrak{m}}$ into $\Aut(V_\ell)$.}
Moreover

\begin{prop}
	A homomorphism $\phi\colon \Gamma_{\mathfrak{m}} \to \Aut(V_\ell)$
	can be extended to a continuous homomorphism of
	$\widehat{\Gamma}_{\mathfrak{m}}$ if and only if there exists a lattice
	of $V_\ell$ which is stable by $\rho(\Gamma_{\mathfrak{m}})$.
\end{prop}

The necessity follows from Remark~\ref{rmk:I_11_1} of chap.~\ref{ch:i},
\ref{sec:I_11}. The sufficiency is clear.

Note that, as in the number field case, we have Frobenius
elements and we can define the notion of \emph{rationality} of an $\ell$-adic
representation.

\begin{thm}
	An abelian $\ell$-adic representation
	\[
		\phi \colon \widehat{\Gamma}_{\mathfrak{m}} \to \Aut(V_\ell)
	\]
	\dpage
	of $K$ is rational if and only if $\Tr\phi(\gamma)$ belongs to $\Q$
	for every $y \in \Gamma_{\mathfrak{m}}$.
\end{thm}

If $v \notin \Supp(\mathfrak{m})$, and if $f_v$ is a uniformizing parameter at
$v$, the image $F_v$ of $f_v$ in $\Gamma_{\mathfrak{m}}$ is the Frobenius
element of the Galois group $\widehat{\Gamma}_{\mathfrak{m}}$. Hence, if
$\Tr\phi$ takes rational values on $\Gamma_{\mathfrak{m}}$, the
characteristic polynomial of $\phi(F_v)$ has rational coefficients for all
$v \notin \Supp(\mathfrak{m})$ and $\phi$ is rational.

To prove the converse, note first that \v Cebotarev's theorem
(Chap.~\ref{ch:i}, \ref{sec:I_22}) is valid for $K$, if one uses a somewhat
weaker definition of equipartition. Hence, the Frobenius elements $F_v$ are
\emph{dense} in $\widehat{\Gamma}_{\mathfrak{m}}$. In particular, they generate
$\Gamma_{\mathfrak{m}}$, and, from this, one sees that $\Tr\rho(\gamma)$
belongs to some number field $E$. We can then construct an $E$-linear
representation $\phi\colon \Gamma_{\mathfrak{m}} \to \GL(n, E)$ with the same
trace as $\rho$, and the theorem follows from:

\begin{lem}
	Let $\Gamma$ be a finitely generated abelian group, and $\phi\colon
	\Gamma \to \GL(n, E)$ a linear representation of $\Gamma$ over a number
	field $E$. Let $\Sigma$ be a subset of $\Gamma$, which is dense in
	$\Gamma$ for the topology of subgroups of finite index. Assume that
	$\Tr\phi(\gamma) \in \Q$ for all $\gamma \in \Sigma$. Then
	$\Tr\phi(\gamma) \in \Q$ for all $\gamma \in \Gamma$.
\end{lem}
\begin{proof}
	Since $\phi(\Gamma)$ is finitely generated, there is a finite $S$ of
	places of $E$ such that all the elements of $\phi(\Gamma)$ are
	$S$-integral matrices. If $\ell'$ is a prime number not divisible by
	any element of $S$, the image of $\phi(\Gamma)$ in $\GL(n, E \otimes
	\Q_{\ell'})$ is contained in a compact subgroup of $\GL(n, E \otimes
	\Q_{\ell'})$; hence $\phi$ extends by continuity to
	\dpage
	\[
		\widehat{\phi} \colon \widehat{\Gamma} \to \GL(n, E \otimes
		\Q_{\ell'})
	\]
	where $\widehat{\Gamma}$ is the completion of $\Gamma$ for the topology
	of subgroups of finite index. Since $\Sigma$ is dense in
	$\widehat{\Gamma}$, it follows that $\Tr\widehat{\phi}(\hat\gamma)$
	belongs to the adherence $\Q_{\ell'}$ of $\Q$ in $E \otimes \Q_{\ell'}$
	for every $\hat\gamma \in \widehat{\Gamma}$.  Hence, if $\gamma \in
	\Gamma$, we have
	\begin{equation}
		\Tr \phi(\Gamma) \in E \cap \Q_{\ell'} = \Q.
		\tqedhere
	\end{equation}
\end{proof}

\subsubsection*{Exercises}
\begin{enumerate}[1)]
\item Let $\phi\colon \widehat{\Gamma}_{\mathfrak{m}} \to \Aut(V_\ell)$ be a
	semi-simple $\ell$-adic representation of $\Gamma_{\mathfrak{m}}$. Show
	the equivalence of:
	\begin{enumerate}[(a)]
	\item $\phi$ extends continuously to $\widehat{\Gamma}_{\mathfrak{m}}$.
	\item For every $\gamma \in \Gamma_{\mathfrak{m}}$, the eigenvalues of
		$\phi(\gamma)$ are units (in a suitable extension of
		$\Q_\ell$).
	\item There exists $\gamma \in \Gamma_{\mathfrak{m}}$, with
		$\deg(\gamma) \ne 0$, such that the eigenvalues of
		$\phi(\gamma)$ are units.
	\item For every $\gamma \in \Gamma_{\mathfrak{m}}$, one has
		$\Tr\phi(\gamma) \in \Z_\ell$.
	\end{enumerate}

\item Let $\phi\colon \widehat{\Gamma}_{\mathfrak{m}} \to \Aut(V_\ell)$ be a
	rational $\ell$-adic representation of $K$.  Show that, for almost all
	prime number $\ell'$, there is a rational $\ell'$-adic representation
	of $K$ compatible with $\phi$. Show that this holds for all $\ell' \ne
	p$ if and only if the following property is valid: for all $\gamma \in
	\Gamma_{\mathfrak{m}}$, the coefficients of the characteristic
	polynomial of $\phi(\gamma)$ are $p$-integers.
\end{enumerate}

\section{The case of a composite of quadratic fields}
\dpage
\subsection{Statement of the result}
\label{sec:III_31}
\todo{Belen.}

\subsection{A criterion for local algebraicity}
\label{sec:III_32}
\begin{prop}
	Let $\rho\colon \abcl{\Gal(\algcl K/K)} \to \Aut(V_\ell)$ be a rational
	semi-simple $\ell$-adic abelian representation of $K$. Assume that
	there exists an integer $N \ge 1$ such that $\rho^N$ is locally
	algebraic. Then $\rho$ is locally algebraic.
\end{prop}
\begin{proof}
	\dpage
	Since $\rho$ is semi-simple, it can be brought in diagonal form over a
	finite extension of $\Q_\ell$, hence gives rise to a family $\{ \psi_1,
	\dots, \psi_n \}$ of $n$ continuous characters $\psi_i\colon C_K \to
	\algcl{\Q}_\ell^\times$, where $C_K$ is the idèle-class group of $K$,
	and $n = \dim V_\ell$.
	Let $\chi_1 = \psi_1^N, \dots, \chi_n = \psi_n^N$ be the corresponding
	characters occurring in $\rho^N$. Since $\rho^N$ is locally algebraic,
	to each $\chi_i^N$ corresponds an algebraic character $\chi_i^{\rm alg}
	\in X(\TT)$ of the torus $\TT$ (here we identify $X(\TT)$ with
	$\Hom(\TT_{/\algcl\Q_\ell}, \GG_{m, \algcl\Q_\ell})$, since
	$\algcl\Q_\ell$ is algebraically closed). Each $\chi_i^{\rm alg}$ is of
	the form $\prod_{\sigma \in \Gamma} [\sigma]^{n_\sigma(i)}$, where
	$\Gamma$ is the set of embeddings of $K$ into $\algcl\Q_\ell$, cf.\ 
	Chap.~\ref{ch:ii}, \ref{sec:II_11}. One has
	\[
		\chi_i(x) = \chi_i^{\rm alg}(x^{-1}) = \prod_{\sigma \in
		\Gamma} \sigma(x)^{-n_\sigma(i)}
	\]
	for all $x \in K_\ell^\times$ close enough to 1.
\end{proof}

\begin{lem}
	All the integers $n_\sigma(i)$, $1 \le i \le n$, $\sigma \in \Gamma$,
	are divisible by $N$.
\end{lem}
\begin{proof}
	Let $U$ be an open subgroup of $\algcl{\Q}_\ell^\times$ containing no
	$N$\textsuperscript{th}-root of unity except 1, and let $\mathfrak{m}$
	be a modulus of $K$ such that $\psi_i(x) \in U$ for all $x \in
	U_{\mathfrak{m}}$ and $i = 1, \dots, n$; the existence of such an
	$\mathfrak{m}$ follows from the continuity of $\psi_1, \dots, \psi_n$.
	We take $\mathfrak{m}$ large enough so that:
	\begin{enumerate}[a)]
	\item It is a modulus of definition for $\rho^N$.
	\item $\rho$ is unramified at all $v \in \Supp(\mathfrak{m})$, and the
		corresponding Frobenius elements $F_{v, \rho}$ have a
		characteristic polynomial with
		\dpage
		rational coefficients.
	\end{enumerate}
	Let $K_{\mathfrak{m}}$ be the abelian extension of $K$ corresponding to
	the open subgroup $K^\times U_{\mathfrak{m}}$ of the idèle group $I$,
	and let $L$ be a finite Galois extension of $\Q$ containing
	$K_{\mathfrak{m}}$. Choose a prime number $p$ which is distinct from 1,
	is not divisible by any place of $\Supp(\mathfrak{m})$, and splits
	completely in $L$. Let $v$ be a place of $K$ dividing $p$, and let
	$f_v$ be an idèle which is a uniformizing element at $v$ and is equal
	to 1 elsewhere. The fact that $v$ splits completely in
	$K_{\mathfrak{m}}$ (since it does in $L$) implies that $f_v$ is the
	norm of an idèle of $K_{\mathfrak{m}}$, hence (by class-field theory)
	belongs to $K^\times U_{\mathfrak{m}}$; this means that the prime ideal
	$\mathfrak{p}_v$ is a principal ideal $(\alpha)$, with $\alpha \equiv 1
	\mod{\mathfrak{m}}$ and $\alpha$ positive at all real places of $K$.

	Let $x = \psi_i(f_v)$ and $y = \chi_i(f_v)$, so that $y = x^N$; these
	are the Frobenius elements of $\psi_i$ and $\chi_i$ relative to $v$. By
	definition of $\chi_i^{\rm alg}$. we have
	\[
		y = \chi_i^{\rm alg}(\alpha) = \prod_{\sigma \in \Gamma}
		\sigma(\alpha)^{n_\sigma(i)}
	\]
	where $\alpha$ is as above.

	Hence $y$ belongs to the subfield $\widetilde{L}$ of $\Q$ corresponding
	to $L$ (this field is well defined since $L$ is a Galois extension of
	$\Q$).  Moreover, if $w_\sigma$ is any place of $L$ such that $w_\sigma
	\circ \sigma$ induces $v$ on $K$, we have (as in chap.~\ref{ch:ii},
	\ref{sec:II_34}):
	\[
		w_\sigma(y) = n_\sigma(i).
	\]
	Assume now that $n_\sigma(i)$ is not divisible by $N$. Then $x$, which
	is an $N$\textsuperscript{th}-root of $y$, does not belong to
	$\widetilde{L}$. Hence there is a
	\dpage
	non-trivial $N$\textsuperscript{th}-root of unity $z$ such that $x$ and
	$zx$ are conjugate over $\widetilde{L}$, and \emph{a fortiori} over
	$\Q$. Since the characteristic polynomial of $F_{v, \rho}$ has rational
	coefficients, any coniugate over $\Q$ of an eigenvalue of $F_{v, \rho}$
	is again an eigenvalue of $F_{v, \rho}$. Hence, there exists an index
	$j$ such that
	\[
		\psi_j(f_v) = z\, x = z\, \psi_i(f_v).
	\]
	But $f_v \in K^\times U_{\mathfrak{m}}$ and all $\psi_j$ are trivial on
	$K^\times$ and map $U_{\mathfrak{m}}$ into the open subgroup $U$ we
	started with. Hence $z = \psi_j(f_v) \, \psi_i(f_v)^{-1}$ belongs to
	$U$, and this contradicts the way $U_{\mathfrak{m}}$ has been chosen.
\end{proof}

\begin{proof}[ of the proposition]
	Since the $n_\sigma(i)$ are divisible by $N$, there exist $\varphi_i
	\in X(\TT)$ with $\varphi_i^N = \chi_i^{\rm alg}$. If $x \in
	K_\ell^\times$, we have:
	\[
		\varphi_i(x^{-1})^N = \chi_i^{\rm alg}(x^{-1}) = \chi_i(x) =
		\psi_i(x)^N
	\]
	if $x$ is close enough to 1. Hence $\varphi_i(x) \psi_i(x)$ is an
	$N$\textsuperscript{th}-root of unity when $x$ is close enough to 1,
	and, by continuity, it is equal to 1 in a neighbourhood of 1. Hence,
	the restriction of $\rho$ to $K_\ell^\times$ is locally equal to
	$\varphi^{-1}$, where $\varphi$ is the (algebraic) representation of
	$\TT$ defined by the family $(\varphi_1, \dots, \varphi_n)$. The
	representation $\varphi$, \emph{a priori} defined over $\algcl\Q_\ell$,
	can be defined over $\Q_\ell$ (and even over $\Q$); this follows, for
	instance, from the fact that the family $(\varphi_1, \dots, \varphi_n)$
	is \emph{stable} under the action of $\Gal(\algcl\Q/\Q)$, since the
	family $(\chi_1^{\rm alg}, \dots, \chi_n^{\rm alg})$ is.

	Hence $\rho$ is locally algebraic.
\end{proof}

\subsection{An auxiliary result on tori}
\label{sec:III_33}
\todo{José.}

\subsection{Proof of the theorem}
\label{sec:III_34}
\todo{Belen.}
