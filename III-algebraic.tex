\chapter{Locally algebraic abelian representations}
\label{ch:iii}
\chaptermark{Locally algebraic representations}

\dpage
In this Chapter, we define what it means for an abelian $\ell$-adic
representation to be \emph{locally algebraic} and we prove (cf.\
\ref{sec:III_23}) that such a representation, when rational, comes from a
linear representation of one of the groups $S_{\mathfrak{m}}$ of Chapter
\ref{ch:ii}.

When the ground field is a composite of quadratic extensions of $\Q$, any
rational semi-simple $\ell$-adic representation is \emph{ipso facto} locally
algebraic; this is proved in \S\ref{sec:III_3}, as a consequence of a result on
transcendental numbers due to Siegel and Lang.

In the local case, an abelian semi-simple representation is
locally algebraic if and only if it has a ``Hodge-Tate decomposition''.
This fact, due to Tate (College de France, 1966), is proved in the
Appendix, together with some complements.

\section{The local case}
\label{sec:III_1}

\subsection{Definitions}
\label{sec:III_11}
Let $p$ be a prime number and $K$ a finite extension of $\Q_p$; let $\TT =
\WRes_{K/\Q_p}(\GG_{m, K})$ be the corresponding algebraic torus over
\dpage
$\Q_p$ (cf.\ \citeauthor{43}~\cite{43}, Chap.~I).

Let $V$ be a finite dimensional $\Q_p$-vector space and denote, as usual, by
$\GL_V$ the corresponding linear group; it is an algebraic group over $\Q_{p'}$
and $\GL_V(\Q_p) = \Aut(V)$.

Let $\rho \colon \abcl{\Gal(algcl{K}/K)} \to \Aut(V)$ be an abelian $p$-adic
representation of $K$ in $V$, where $\abcl{\Gal(\algcl K/K)}$ denotes the Galois
group of the maximal abelian extension of $K$.
If $i \colon K^\times \to \abcl{\Gal(\algcl K/K)}$ is the canonical homomorphism
of local class field theory (cf.\ for instance Cassels - Fr\"ohlich \cite{6},
chap.~VI, \S2), we then get a continuous homomorphism $\rho \circ i$ of
$K^\times = \TT(\Q_p)$ into $\Aut(V)$.

\begin{mydef}
The representation $\rho$ is said to be \strong{locally algebraic} if there is
an algebraic morphism $r \colon \TT \to \GL_V$ such that $\rho \circ i(x) =
r(x^{-1})$ for all $x \in K^\times$ close enough to $1$. 
\end{mydef}

Note that if $r \colon \TT \to \GL_V$ satisfies the above condition, it is
unique; this follows from the fact that any non-empty open set of $K^\times =
\TT(\Q_p)$ is Zariski dense in $\TT$. We say that $r$ is the algebraic morphism
\strong{associated with $\rho$}\index{Associated with (algebraic morphism...\
with a locally algebraic representation)}.

\begin{ex}
\begin{enumerate}
\item Take $K = \Q_p$, and $\dim V = 1$, so that $\rho$ is given by a
	continuous homomorphism $\abcl{\Gal(\overline{\Q}/\Q)} \to U_p$, where
	$U_p$ is the group of $p$-adic units. It is easy to see there exists an
	element $v \in \Z_p$ such that $\rho \circ i (x) = x^v$ if $x$ is close
	enough to $1$. The representation $\rho$ is locally algebraic if and
	only if $v$ belongs to $\Z$.  This happens for instance when $V =
	V_p(\mu)$, cf.\ Chap. \ref{ch:i}, \ref{sec:I_12}, in which case $v = -1$
	and $r$ is the canonical one-dimensional representation of $\TT =
	\GG_{m, \Q_p}$. 
\item The abelian representation associated to a Lubin-Tate formal group
	\dpage
	(cf.\ \cite{17} and \cite{6}, Chap. VI, \S 3) is locally algebraic (and
	also of the form $u \mapsto u^{-1}$ on the inertia group).  
\end{enumerate}
\end{ex}

\begin{prop}
	Let $\rho \colon \abcl{\Gal(algcl{K}/K)} \to \Aut(V)$ be a locally
	algebraic abelian representation of $K$. The restriction of $\rho$ to
	the inertia subgroup of $\abcl{\Gal(\algcl K/K)}$ is semi-simple.
\end{prop}
\begin{proof}
Let us identify the inertia subgroup of $\abcl{\Gal(\algcl K/K)}$ with the
group $U_K$ of units of $K$.
By assumption, there is an open subgroup $U'$ of $U_K$ and an algebraic morphism
$r$ of $\TT$ into $\GL_V$ such that $\rho(x) = r (x^{-1})$ if $x \in U'$.
Let $W$ be a sub-vector space of $V$ stable by $\rho(U_K)$; it is then stable by
$\rho(U')$, hence by $r(T)$. But every linear representation of a torus is
semi-simple. Hence, there exists a projector $\pi \colon V \to W$ which commutes
with the action of $\TT$. If we put $\pi' = \frac{1}{[U_K:U']}\sum_{s \in U_K
\setminus U'} \rho(S) \pi \rho(s^{-1})$,
% \todo[pinktask]{qué notación es (:)}
we obtain a projector $\pi' \colon V \to W$ which commutes with all $\rho(s)$,
$s \in U_K$. 
\end{proof}
\todo[pinktask]{aquí termina la dem??}

Conversely, let us start from a representation $\rho$ whose restriction to $U_K$
is semi-simple. If we make a suitable large finite extension $E$ of $\Q_p$, the
restriction of $\rho$ to $U_K$ may be brought into diagonal form, i.e.\ is given
by continuous characters $\chi_i \colon U_K \to E^{\times}$, $i = 1,\dots,n$.
We assume $E$ large enough to contain all conjugates of $K$ into $E$. Recall
(cf.\ Chap. \ref{ch:ii}, \ref{sec:II_11}) that the $[\sigma]$, $\sigma\in
\Gamma_K$, make a basis of the character group $X(\TT)$ of $\TT$.

\begin{prop}
The representation $\rho$ is locally algebraic if and only if there exists
integer $n_\sigma(i)$ such that
\dpage
\[
	\chi_i(u) = \prod_{\sigma \in \Gamma_K} \sigma(u)^{-n_\sigma(i)}
\]
for all $i$ and all $u$ close enough to $1$.
\end{prop}
\begin{proof}
The necessity if trivial. Conversely, if there exist such integers
$n_\sigma(i)$, they define algebraic characters $r_i = \prod
[\sigma]^{n_\sigma(i)}$ of $\TT$, hence a linear representation $r$ of $T_{/E}$.
It is clear that there is an open subgroup $U'$ of $U_K$, such that $\rho(u) =
r(u^{-1})$ for all $u \in U'$. Hence it remains to see that $r$ can be defined
over $\Q_p$ (cf.\ Chap. \ref{ch:ii}, \ref{sec:II_24}). But the trace $\theta_r =
\sum r_i$ of $r$ (\emph{loc.\ cit.}) is such that $\theta_r(u) \in \Q_p$ for all $u
\in U'$. Since $U'$ is Zariski-dense in $\TT$, this implies that $\theta_r$ is
\textquote{defined over $\Q_p$}, hence that $r$ can be defined over $\Q_p$
(\emph{loc.\ cit.}).
\end{proof}
 
\subsubsection*{Extension of the ground field}
Let $K'$ be a finite extension of $K$, and let $\rho'$ be the restriction of the
given representation $\rho$ to $\Gal(algcl{K}/K')$. Then $\rho'$ is locally
algebraic if and only if
% \todo[pinktask]{falta un if?}
$\rho$ is; moreover, if this is so, the associated algebraic morphisms
\[
r \colon \TT \to \GL_V, \qquad r' \colon \TT' \to \GL_V
\]
are such that $r' = \Nm_{K'/K} \circ r$, where $\TT'$ is the
torus associated with $K'$ and $N_{K'/K} \colon \TT' \to T$ is the algebraic
morphism defined by the norm from $K'$ to $K$.

All this follows easily from the commutativity of the diagram
\dpage
% https://q.uiver.app/#q=WzAsNCxbMCwwLCJLXlxcdGltZXMiXSxbMSwwLCJcXEdhbChcXG92ZXJsaW5le0t9L0spXlxcYWIiXSxbMCwxLCJLJyJdLFsxLDEsIlxcR2FsKFxcb3ZlcmxpbmV7S30vSycpXlxcYWIiXSxbMCwxXSxbMiwwLCJOIiwyXSxbMiwzXSxbMywxXV0=
\[\begin{tikzcd}
	{K^\times} & {\abcl{\Gal(\algcl K/K)}} \\
	{{K'}^\times} & {\abcl{\Gal(\algcl K/K')}}
	\arrow[from=1-1, to=1-2]
	\arrow["\Nm"', from=2-1, to=1-1]
	\arrow[from=2-1, to=2-2]
	\arrow[from=2-2, to=1-2]
\end{tikzcd}\]
and from the fact that the kernel of $\Nm_{K'/K} \colon \TT' \to \TT$ is connected
for the Zariski topology.

\subsubsection*{Exercise}
Give an example of a locally algebraic abelian $p$-adic representation of
dimension $2$ which is not semi-simple.

\subsection{Alternative definition of ``locally algebraic'' via Hodge-Tate
modules}
\label{sec:III_12}
Let us recall first the notion of a \strong{Hodge-Tate module} (cf.\ \cite{27},
\S 2); here $K$ is only assumed to be complete with respect to a discrete
valuation, with perfect residue field $k$ and $\char(K) = 0$, $\char(k) = p$.
Denote by $C$ the \emph{completion $\widehat{\algcl K}$ of the algebraic
closure} of $K$.

The group $G = \Gal(\algcl K/K)$ acts continuously on $K$. This action extends
continuously to $C$. Let $W$ be a $C$-vector space of finite dimension upon
which $G$ acts continuously and semi-linearly according to the formula
\[
	s(cw) = s(c) \cdot s(w) \qquad
	(s \in G, \, c \in C \text{ and } w \in W).
\]
Let $\chi\colon G \to U_p$ be the homomorphism of $G$ into the group $U_p =
\Z_p^\times$ of $p$-adic units, defined by its action on the $p^\nu$-th roots
of unity (cf.\ chap.~\ref{ch:i}, \ref{sec:I_12}):
\dpage
\[
	s(z) = z^{\chi(s)} \qquad \text{if } s \in G \text{ and } z^{p^\nu} = 1.
\]
Define for every $i \in \Z$ the subspace
\[
	W^i = \{ w \in W \mid sw = \chi(s)^i w \text{ for all } s\in G \}
\]
of $W$. This is a $K$-vector subspace of $W$. Let $W(i) = C \otimes_K W^i$.
This is a $C$-vector space upon which $G$ acts in a natural way (i.e.\ by the
formula $s(c \otimes y) = s(c) \otimes s(y)$). The inclusion $W^i \to W$
extends uniquely to a $C$-linear map $\alpha_i\colon W(i) \to W$, which
commutes with the action of $G$.

\begin{prop}[Tate]
	Let $\coprod_{i\in \Z} W(i)$ be the direct sum of the $W(i)$. Let
	$\alpha\colon \coprod_i W(i) \to W$ be the sum of the $\alpha_i$'s
	defined above. Then $\alpha$ is injective.
\end{prop}
For the proof see \cite{27}, \S 2, prop.~4.
\begin{corp}
	The $K$-spaces $W^i$ ($i \in \Z$) are of finite dimension.
	They are linearly independent over $C$.
\end{corp}

\begin{mydef}\label{def:III_12_1}
	The module $W$ is of \strong{Hodge-Tate type}\index{Hodge-Tate module}%
	\index{Hodge-Tate type (module)}
	if the homomorphism $\alpha\colon \coprod_{i\in\Z} W(i) \to W$ is an
	isomorphism.
\end{mydef}

Let now $V$ be as in \ref{sec:III_11}, a vector space over $\Q_p$, of finite
dimension. Let $\rho\colon G \to \Aut(V)$ be a $p$-adic representation. Let $W
= C \otimes_{\Q_p} V$ and let $G$ act on $W$ by the formula
\dpage
\[
	s(c\otimes v) = s(c) \otimes s(v) \qquad
	s\in \Gamma, \; c\in C, \; v\in V.
\]
\begin{mydef}
	The representation $\rho$ is of \strong{Hodge-Tate type}%
	\index{Hodge-Tate representation}\index{Hodge-Tate type
	(representation)} if the $C$-space $W = C \otimes_{\Q_p} V$ is of
	Hodge-Tate type (cf.\ def.~\ref{def:III_12_1}).
\end{mydef}

\begin{ex}
	Let $F$ be a $p$-divisible group of finite height (cf.\ \cite{26},
	\cite{39}); let $T$ be its Tate module (\emph{loc. cit.}) and $V = \Q_p
	\otimes T$. The group $G$ acts on $V$, and Tate has proved (\cite{39},
	Cor.~2 to Th.~3) that this Galois module is of Hodge-Tate type; more
	precisely, one has $W = W(0) \oplus W(1)$, where $W = C \otimes V$ as
	above.
\end{ex}
\begin{thm}[Tate]
Assume $K$ is a finite extension of $\Q_p$ (i.e.\ its residue field is finite).
Let $\rho\colon G \to \Aut(V)$ be an abelian $p$-adic representation of $K$.
The following properties are equivalent:
\begin{enumerate}[(a)]
\item $\rho$ is locally algebraic (cf.\ \ref{sec:III_11}).
\item $\rho$ is of Hodge-Tate type and its restriction to the inertia group is
	semi-simple.
\end{enumerate}
\end{thm}
For the proof, see the Appendix.

\section{The global case}
\subsection{Definitions}
\label{sec:III_21}
We now go back to the notations of Chap.~\ref{ch:ii}, i.e.\ $K$ denotes a number
field. Let $\ell$ be a prime number and let 
\[
\rho \colon \abcl{\Gal(\algcl{K}/K)} \longrightarrow \Aut(V_\ell)
\]
be an abelian $\ell$-adic representation
\dpage
of $K$.
Let $v \in M_K^0$ be a place of $K$ of residue characteristic $\ell$ and let
$D_v \subset \abcl{\Gal(\algcl{K}/K)}$ be the corresponding decomposition group.
This group is a quotient of the local Galois group
$\abcl{\Gal(\algcl{K_v}/K_v)}$ (there two group are, in fact, isomorphic, but we
do not need this here). Hence, we get by composition an $\ell$-adic
representation of $K_v$ 
\[\begin{tikzcd}[sep=large]
	\rho_v \colon \abcl{\Gal(\algcl{K_v}/K_v)} \rar & D_v \rar["\rho"] &
	\Aut(V_\ell).
\end{tikzcd}\]

\begin{mydef}
The representation $\rho$ is said to be \strong{locally algebraic} if all the
local representations $\rho_v$, with $p_v = \ell$, are locally algebraic (in the
sense defined in \ref{sec:III_11}, with $p = \ell$). 
\end{mydef}

It is convenient to reformulate this definition, using the torus $T =
\WRes_{K/\Q} (\GG_{m, K})$ of Chap.~\ref{ch:ii},~\ref{sec:II_11}. Let
$\TT_{/\Q_\ell} = \TT \otimes_{\Q} \Q_\ell$ be the $\Q_\ell$-torus obtained
from $\TT$ by extending the ground field from $\Q$ to $\Q_\ell$. We have
\[
\TT_{/\Q_\ell}(\Q_\ell) = (K \otimes \Q_\ell)^{\times} = K_{\ell}^\times,
\]
where $K_\ell = K \otimes \Q_\ell$. 

Let $I$ be the idèle group of $K$, cf.\ Chap.~\ref{ch:ii},~\ref{sec:II_21}. The
injection $K_\ell^\times \to I$, followed by the class field homomorphism $i
\colon I \to \abcl{\Gal(\algcl{K}/K)}$, define a homomorphism
\[
i_\ell \colon K_\ell^\times \longrightarrow \abcl{\Gal(\algcl{K}/K)}.
\]

\begin{prop}
The representation $\rho$
\dpage
is locally algebraic if and only if there exists an algebraic morphism 
\[
f \colon \TT_{/\Q_\ell} \longrightarrow \GL_{V_\ell}
\]
such that $\rho \circ i_\ell(x) = f(x^{-1})$ for all $x \in K_\ell^\times$
close enough to $1$. 
\end{prop}

(Note that, as in the local case, the above condition determines $f$ uniquely;
one says it is the algebraic morphism \emph{associated with} $\rho$.)

Since $K \otimes_\Q \Q_\ell = \prod_{v \mid \ell} K_v$, we have 
\[
\TT_{/\Q_\ell} = \prod_{v \mid \ell} T_v,
\]
where $T_v$ is the $\Q_\ell$-torus defined by $K_v$, cf.~\ref{III:sec_11}. The 
proposition follow from this decomposition. 

\subsubsection*{Exercise}
Give a criterion for local algebraicity analogous to the one of
Prop.~\ref{prop:III_11_2} of \ref{sec:III_11}.

\subsection{Modulus of a locally algebraic abelian representation}
\label{sec:III_22}
Let $\rho\colon \abcl{\Gal(\algcl K/K)} \to \Aut(V_\ell)$ be as above; by
composition with the class field homomorphism $i\colon I \to \abcl{\Gal(\algcl
K/K)}$, $\rho$ defines a homomorphism $\rho \circ i\colon I \to \Aut(V_\ell)$.

We assume that $\rho$ is locally algebraic and we denote by $f$ the associated
\dpage
algebraic morphism $\TT_{/\Q_\ell} \to \GL_{V_\ell}$.
\begin{mydef}
Let $\mathfrak{m}$ be a modulus (chap.~\ref{ch:ii}, \ref{sec:II_11}).
One says that $\rho$ is defined mod $\mathfrak{m}$ (or that
$\mathfrak{m}$ is a modulus of definition for $\rho$) if
\begin{enumerate}[(i)]
	\item $\rho \circ i$ is trivial on $U_{v, \mathfrak{m}}$ when $p_v \ne
		\ell$.
	\item $\rho \circ i_\ell(x) = f(x^{-1})$ for \smash{$\displaystyle x
		\in \prod_{v\mid\ell} U_{v, \mathfrak{m}}$}.
\end{enumerate}
\end{mydef}
(Note that $\prod_{v\mid\ell} U_{v, \mathfrak{m}}$ is an open subgroup of
$K_\ell^\times = \TT_{/\Q_\ell}(\Q_\ell)$.)

In order to prove the existence of a modulus of definition, we
need the following auxiliary result:
\begin{prop}\label{prop:III_22_1}
Let $H$ be a Lie group over $Q_\ell$ (resp.\ $\R$) and let
$\alpha$ be a continuous homomorphism of the idèle group $I$ into $H$.
\begin{enumerate}[(a)]
\item\label{prop:III_22a}
	If $p_v \ne \ell$ (resp.\ $p_v \ne \infty$), the restriction of
	$\alpha$ to $K$ is equal to 1 on an open subgroup of $K_v^\times$.
\item\label{prop:III_22b}
	The restriction of $\alpha$ to the unit group $U_v$ of $K_v^\times$ is
	equal to 1 for almost all $v$'s.
\end{enumerate}
\end{prop}
\begin{proof}
	Part \ref{prop:III_22a} follows from the fact that $K_v^\times$ is a
	$p_v$-adic Lie group and that a homomorphism of a $p$-adic Lie group
	into an $\ell$-adic one is locally equal to 1 if $p \ne \ell$.

	To prove \ref{prop:III_22b}, let $N$ be a neighborhood of 1 in $H$
	which contains no finite subgroup except $\{ 1 \}$; the existence of
	such an $N$ is classical for real Lie groups, and quite easy to prove
	for $\ell$-adic ones. By definition of the idèle topology,
	$\alpha(U_v)$ is contained in $N$ for almost all $v$'s. But
	\ref{prop:III_22a} shows that, if $p_v \ne \ell$, the group
	\dpage
	$\alpha(U_v)$ is finite; hence $\alpha(U_v) = \{ 1 \}$ for almost all
	$v$'s.
\end{proof}

\begin{corp}\label{cor:III_22}
	Any abelian $\ell$-adic representation of $K$ is unramified outside a
	finite set of places.
\end{corp}
This follows from \ref{prop:III_22b} applied to the homomorphism $\alpha$ of $I$
induced by the given representation, since the $\alpha(U_v)$ are known to be
the inertia subgroups.
% 2d by the given r

\begin{obs}
This does not extend to non-abelian representations (even solvable ones), cf.\ Exercise.
\end{obs}
\begin{prop}
	Every locally algebraic abelian $\ell$-adic representation has a
	modulus of definition.
\end{prop}
Let $\rho\colon \abcl{\Gal(\algcl K/K)} \to \Aut(V_\ell)$ be the given
representation and $f$ the associated morphism of $T_{/\Q_\ell}$ into
$\GL_{V_\ell}$. Let $X$ be the set of places $v \in M_K^0$, with $p_v \ne
\ell$, for which $\rho$ is ramified; the corollary~\ref{cor:III_22} to
Prop.~\ref{prop:III_22_1} shows that $X$ is finite. By
Prop.~\ref{prop:III_22_1}, \ref{prop:III_22a}, we can choose a modulus
$\mathfrak{m}$ such that $\rho \circ i\colon I \to \Aut(V_\ell)$ is trivial on
all the $U_{v, \mathfrak{m}}$, $v \in X$. Enlarging $\mathfrak{m}$ if
necessary, we can assume that $\rho \circ i_\ell(x) = f(x^{-1})$ for $x \in
\prod_{p_v = \ell} U_{v, \mathfrak{m}}$. Hence, $\mathfrak{m}$ is a modulus of
definition for $\rho$.

\begin{obs}
It is easy to show that there is a smallest modulus of definition for $\rho$;
it is called the \strong{conductor}\index{Conductor!(of a locally algebraic
representation)} of $\rho$.
\end{obs}

\subsubsection*{Exercise}
Let $z_1, \dots, z_n, \dots \in K^\times$. For each $n$, let $E_n$ be the
\dpage
subfield of $\algcl K$ generated by all the $\ell^n$-th roots of the element
$z_1 z_2^\ell \cdots z_n^{\ell^{n-1}}$.
\begin{enumerate}[a)]
\item Show that $E_n$ is a Galois extension of $K$, containing the $\ell^n$-th
	roots of unity and that its Galois group is isomorphic to a subgroup of
	the affine group $
	\begin{psmallmatrix}
		* & * \\
		0 & 1
	\end{psmallmatrix} 
	$ in $\GL(2, \Z/\ell^n\Z)$.
\item Let $E$ be the union of the $E_n$'s. Show that $E$ is a Galois extension
	of $K$, whose Galois group is a closed subgroup of the affine group
	relative to $\Z_\ell$.
\item Give an example where $E$ (and hence the corresponding 2-dimensional
	$\ell$-adic representation) is ramified at all places of $K$.
\end{enumerate}

\subsection{Back to \texorpdfstring{$S_{\mathfrak{m}}$}{Sm}}
\label{sec:III_23}
Let $\mathfrak{m}$ be a modulus of $K$ and let
\[
	\phi \colon S_{\mathfrak{m}/\Q_\ell} \longrightarrow \GL_{V_\ell}
\]
be a linear representation of $S_{\mathfrak{m}/\Q_\ell}$. Let
\[
	\phi_\ell \colon \abcl{\Gal(\algcl K/K)} \longrightarrow \Aut(V_\ell)
\]
be the corresponding $\ell$-adic representation (cf.\ chap.~\ref{ch:ii},
\ref{sec:II_25}).

\begin{thm}\label{thm:III_23_1}
	The representation $\phi_\ell$ is locally algebraic and defined mod
	$\mathfrak{m}$. The associated algebraic morphism
	\dpage
	\[
		f \colon \TT_{/\Q_\ell} \longrightarrow \GL_{V_\ell}
	\]
	is $\phi \circ \pi$, where $\pi$ denotes the canonical morphism of
	$\TT$ into $S_{\mathfrak{m}}$ (cf.\ chap.~\ref{ch:ii},
	\ref{sec:II_22}).
\end{thm}
This is trivial from the construction of $\phi_\ell$ as $\phi \circ
\varepsilon$ (chap.~\ref{ch:ii}, \ref{sec:II_25}) and the corresponding
properties of $\varepsilon_\ell$ (chap.~\ref{ch:ii}, \ref{sec:II_23}).

The converse of Theorem~\ref{thm:III_23_1} is true. We state it only for the
case of rational representations:
\begin{thm}
	Let $\rho\colon \abcl{\Gal(\algcl K/K)} \to \Aut(V_\ell)$ be an abelian
	$\ell$-adic representation of the number field $K$. Assume $\rho$ is
	rational (chap.~\ref{ch:i}, \ref{sec:I_23}) and is locally algebraic
	with $\mathfrak{m}$ as a modulus of definition (cf.\ \ref{sec:III_22}).
	Then, there exist a $\Q$-vector subspace $V_0$ of $V_\ell$, with
	$V_\ell = V_0 \otimes_\Q \Q_\ell$, and a morphism $\phi_0 \colon
	S_{\mathfrak{m}} \to \GL_{V_\ell}$ of $\Q$-algebraic groups such that
	$\rho$ is equal to the $\ell$-adic representation $\phi_\ell$
	associated to $\phi_0$ (cf.\ chap.~\ref{ch:ii}, \ref{sec:II_25}).
\end{thm}
(The condition $V_\ell = V_0 \otimes_\Q \Q_\ell$ means that $V_0$ is a
\textquote{$\Q$-structure} on $V_\ell$, cf.\ Bourbaki Alg., chap.~II,
3\textsuperscript{rd} ed.)

\begin{proof}
	Let $r \colon \TT_{/\Q_\ell} \to \GL_{V_\ell}$ be the algebraic
	morphism associated with $\rho$. We have
	\[
		\rho\circ i(x) = r(x^{-1}) \qquad \text{for } x\in
		K_\ell^\times \cap U_{\mathfrak{m}} = \prod_{v\mid\ell} U_{v,
		\mathfrak{m}}
	\]
	Define a map $\psi \colon I \to \Aut(V_\ell)$ by
	\dpage
	\[
		\psi(x) = \rho\circ i(x) \cdot r(x_\ell)
	\]
	where $x_\ell$ is the $\ell$\textsuperscript{th} component of the idèle
	$x$.
	\todo[bluetask]{¿Estandarizar notación $\vec x$ para idèles?}
	One checks immediately that \emph{$\psi$ is trivial on
	$U_{\mathfrak{m}}$ and coincides with $r$ on $K^\times$.}
	Hence $r$ is trivial on $E_{\mathfrak{m}} = K^\times \cap
	U_{\mathfrak{m}}$ and factors through an algebraic morphism
	$r_{\mathfrak{m}} \colon T_{\mathfrak{m}/\Q_\ell} \to \GL_{V_\ell}$. By
	the universal property
	of the $\Q_\ell$-algebraic group $S_{\mathfrak{m}/\Q_\ell}$ (cf.\ chap.~\ref{ch:ii}, \ref{sec:II_13} and \ref{sec:II_22}),
	there exists an algebraic morphism
	\[
		\phi \colon S_{\mathfrak{m}/\Q_\ell} \longrightarrow
		\GL_{V_\ell}
	\]
	with the following properties:
	\begin{enumerate}[(a)]
	\item\label{thm:III_23_2a} The morphism 
		\begin{tikzcd}[cramped, sep=small]
			T_{\mathfrak{m}/\Q_\ell} \rar &
			S_{\mathfrak{m}/\Q_\ell} \rar["\phi"] & \GL_{V_\ell}
		\end{tikzcd}
		is $r_{\mathfrak{m}}$.
	\item The map 
		\begin{tikzcd}[cramped, sep=small]
			I \rar["\varepsilon"] & S_{\mathfrak{m}}(\Q_\ell)
			\rar["\phi"] & \Aut(V_\ell)
		\end{tikzcd}
		is $\psi$.
	\end{enumerate}
	It is trivial to check that the $\ell$-adic representation
	$\phi_\ell$ attached to $\phi$ as above coincides with $\rho$.
	Indeed, if $a \in I$, we have (with the notations of chap.~\ref{ch:ii})
	\begin{align*}
		\phi_\ell\circ i(a) &= \phi(\varepsilon_\ell(a))
		= \phi(\varepsilon(a))
		\phi\big( \pi_\ell(a_\ell^{-1}) \big)
		= \psi(a) \phi\big( \pi_\ell(a_\ell^{-1}) \big) \\
				       &= \rho\circ i(a) r(a_\ell)
				       \phi\big( \pi_\ell(a_\ell^{-1}) \big)
				       = \rho\circ i(a).
	\end{align*}
	since $\phi\circ\pi_\ell = r$ by \ref{thm:III_23_2a} above.
	\dpage

	Hence $\phi_\ell = \rho$; the fact that $\rho$ is \emph{rational}
	then implies that \emph{$\phi$ can be defined over $\Q$}
	(chap.~\ref{ch:ii}, \ref{sec:II_24}, Prop.), and this gives $V_0$ and
	$\phi_0$.
\end{proof}

\begin{obs}
	The subspace $V_0$ of $V_\ell$ constructed in the proof of the theorem
	is \emph{not} unique; however, any other choice gives us a space of the
	form $\sigma V_0$, where $\sigma$ is an automorphism of $V_\ell$
	commuting with $\rho$. To a given $V_0$ corresponds of course a unique
	$\phi$.
\end{obs}

\begin{cor}
	For each prime number $\ell'$ there exists a unique (up to isomorphism)
	$\ell'$-adic rational semi-simple representation $\rho$ of $K$,
	compatible with $\rho$. It is abelian and locally algebraic.  These
	representations form a strictly compatible system (cf.\
	chap.~\ref{ch:i}, \ref{sec:I_23}) with exceptional set contained in
	$\Supp(\mathfrak{m})$. For an infinite number of $\ell'$,
	$\rho_{\ell'}$ can be brought in diagonal form.
\end{cor}
\begin{proof}
	The unicity of the $\rho_{\ell'}$, follows from the theorem of
	chap.~\ref{ch:i}, \ref{sec:I_23}. For the existence, take
	$\rho_{\ell'}$ to be the $\phi_{\ell'}$ associated to $\phi$ as
	in chapter~\ref{ch:ii}, \ref{sec:II_25}. The remaining assertion
	follows from the proposition in chap.~\ref{ch:ii}, \ref{sec:II_25}.
\end{proof}

\begin{cor}
	The eigenvalues of the Frobenius elements $F_{v, \rho}$ ($v \notin
	\Supp(\mathfrak{m})$, $p_v \ne \ell$) generate a finite extension of
	$\Q$.
\end{cor}
This follows from the corresponding property of $\phi_\ell$, cf.\
chapter~\ref{ch:ii}, \ref{sec:II_25}, Remark~\ref{rmk:II_25_1}.
\dpage
\subsection{A mild generalization}
\label{sec:III_24}

Most results of this and the previous Chapter may be extended to the case where
we take for ground field of the linear representation a number field $E$
(instead of $\Q$). More precisely, let $\lambda$ be a finite place of $E$ and
let $E_\lambda$ be the $\lambda$-adic completion of $E$. The notion of an
$E$-rational $\lambda$-adic representation of $K$ has been defined in chap.
\ref{ch:i}, \ref{sec:I_23}, Remark. \todo[pinktask]{referenciar el remark?} Let 
\[
	\rho \colon \Gal(algcl{K}/K) \to \Aut(V_\lambda)
\]
be such a representation, and assume $r$ is abelian. Let $\ell$ be the residue
characteristic of $\lambda$, so that $E_\lambda$ contains $\Q_\ell$. As in
\ref{sec:III_21}, we say that $\rho$ is \strong{locally algebraic} if there
exists an algebraic morphism
\[
	f \colon \TT_{/E_\lambda} \to GL_{V_\lambda}
\]
such that $\rho \circ i_\lambda(x) = f(x^{-1})$ for $x \in K^\times_\ell$ close
enough to $1$ (note that $K^\times_\ell = \TT(\Q_p)$ is subgroup of
$\TT(E_\ell)$. As in \ref{sec:III_23}, one proves that such a $\rho$ \emph{comes
from an $E$-linear representation of some $S_\mathfrak{m}$} (and conversely).  

\subsection{The function field case}
\label{sec:III_25}
The above constructions have a (rather elementary) analogue
for \emph{function fields of one variable over a finite field:}

Let $K$ be such a field, and let $p$ be its characteristic. If $\mathfrak{m}$
is a modulus for $K$ (i.e.\ a positive divisor) we define the subgroup
$U_{\mathfrak{m}}$ of the idèle group $I$ as in chap.~\ref{ch:ii},
\ref{sec:II_21}, and we put
\[
	\Gamma_{\mathfrak{m}} = I/U_{\mathfrak{m}} K^\times.
\]
\dpage
The degree map $\deg\colon I \to \Z$ is trivial on $U_{\mathfrak{m}}$, hence defines an
exact sequence
\[\begin{tikzcd}
	1 \rar & J_{\mathfrak{m}} \rar & \Gamma_{\mathfrak{m}} \rar & \Z \rar & 1.
\end{tikzcd}\]
One sees easily that the group $J_{\mathfrak{m}}$ is finite; moreover, it may
be interpreted as the group of rational points of the ``generalized Jacobian
variety defined by $\mathfrak{m}$''. If $\widehat{\Gamma}_{\mathfrak{m}}$
denotes the completion of r with respect to the topology of subgroups of finite
index, it is known (class field theory) that $\abcl{\Gal(\algcl K/K)} \cong
\invlim_{\mathfrak{m}} \widehat{\Gamma}_{\mathfrak{m}}$.

Let now $\rho\colon \abcl{\Gal(\algcl K/K)} \to \Aut(V_\ell)$ be an abelian
$\ell$-adic representation of $K$, with $\ell \ne p$. One proves as in
\ref{sec:III_22} that there exists a modulus $\mathfrak{m}$ such that $\rho$ is
trivial on $U_{\mathfrak{m}}$, i.e.\ such that $\rho$ may be identified with a
\emph{homomorphism of $\widehat{\Gamma}_{\mathfrak{m}}$ into $\Aut(V_\ell)$.}
Moreover

\begin{prop}
	A homomorphism $\phi\colon \Gamma_{\mathfrak{m}} \to \Aut(V_\ell)$
	can be extended to a continuous homomorphism of
	$\widehat{\Gamma}_{\mathfrak{m}}$ if and only if there exists a lattice
	of $V_\ell$ which is stable by $\rho(\Gamma_{\mathfrak{m}})$.
\end{prop}

The necessity follows from Remark~\ref{rmk:I_11_1} of chap.~\ref{ch:i},
\ref{sec:I_11}. The sufficiency is clear.

Note that, as in the number field case, we have Frobenius
elements and we can define the notion of \emph{rationality} of an $\ell$-adic
representation.

\begin{thm}
	An abelian $\ell$-adic representation
	\[
		\phi \colon \widehat{\Gamma}_{\mathfrak{m}} \to \Aut(V_\ell)
	\]
	\dpage
	of $K$ is rational if and only if $\Tr\phi(\gamma)$ belongs to $\Q$
	for every $y \in \Gamma_{\mathfrak{m}}$.
\end{thm}

If $v \notin \Supp(\mathfrak{m})$, and if $f_v$ is a uniformizing parameter at
$v$, the image $F_v$ of $f_v$ in $\Gamma_{\mathfrak{m}}$ is the Frobenius
element of the Galois group $\widehat{\Gamma}_{\mathfrak{m}}$. Hence, if
$\Tr\phi$ takes rational values on $\Gamma_{\mathfrak{m}}$, the
characteristic polynomial of $\phi(F_v)$ has rational coefficients for all
$v \notin \Supp(\mathfrak{m})$ and $\phi$ is rational.

To prove the converse, note first that \v Cebotarev's theorem
(Chap.~\ref{ch:i}, \ref{sec:I_22}) is valid for $K$, if one uses a somewhat
weaker definition of equipartition. Hence, the Frobenius elements $F_v$ are
\emph{dense} in $\widehat{\Gamma}_{\mathfrak{m}}$. In particular, they generate
$\Gamma_{\mathfrak{m}}$, and, from this, one sees that $\Tr\rho(\gamma)$
belongs to some number field $E$. We can then construct an $E$-linear
representation $\phi\colon \Gamma_{\mathfrak{m}} \to \GL(n, E)$ with the same
trace as $\rho$, and the theorem follows from:

\begin{lem}
	Let $\Gamma$ be a finitely generated abelian group, and $\phi\colon
	\Gamma \to \GL(n, E)$ a linear representation of $\Gamma$ over a number
	field $E$. Let $\Sigma$ be a subset of $\Gamma$, which is dense in
	$\Gamma$ for the topology of subgroups of finite index. Assume that
	$\Tr\phi(\gamma) \in \Q$ for all $\gamma \in \Sigma$. Then
	$\Tr\phi(\gamma) \in \Q$ for all $\gamma \in \Gamma$.
\end{lem}
\begin{proof}
	Since $\phi(\Gamma)$ is finitely generated, there is a finite $S$ of
	places of $E$ such that all the elements of $\phi(\Gamma)$ are
	$S$-integral matrices. If $\ell'$ is a prime number not divisible by
	any element of $S$, the image of $\phi(\Gamma)$ in $\GL(n, E \otimes
	\Q_{\ell'})$ is contained in a compact subgroup of $\GL(n, E \otimes
	\Q_{\ell'})$; hence $\phi$ extends by continuity to
	\dpage
	\[
		\widehat{\phi} \colon \widehat{\Gamma} \to \GL(n, E \otimes
		\Q_{\ell'})
	\]
	where $\widehat{\Gamma}$ is the completion of $\Gamma$ for the topology
	of subgroups of finite index. Since $\Sigma$ is dense in
	$\widehat{\Gamma}$, it follows that $\Tr\widehat{\phi}(\hat\gamma)$
	belongs to the adherence $\Q_{\ell'}$ of $\Q$ in $E \otimes \Q_{\ell'}$
	for every $\hat\gamma \in \widehat{\Gamma}$.  Hence, if $\gamma \in
	\Gamma$, we have
	\begin{equation}
		\Tr \phi(\Gamma) \in E \cap \Q_{\ell'} = \Q.
		\tqedhere
	\end{equation}
\end{proof}

\subsubsection*{Exercises}
\begin{enumerate}[1)]
\item Let $\phi\colon \widehat{\Gamma}_{\mathfrak{m}} \to \Aut(V_\ell)$ be a
	semi-simple $\ell$-adic representation of $\Gamma_{\mathfrak{m}}$. Show
	the equivalence of:
	\begin{enumerate}[(a)]
	\item $\phi$ extends continuously to $\widehat{\Gamma}_{\mathfrak{m}}$.
	\item For every $\gamma \in \Gamma_{\mathfrak{m}}$, the eigenvalues of
		$\phi(\gamma)$ are units (in a suitable extension of
		$\Q_\ell$).
	\item There exists $\gamma \in \Gamma_{\mathfrak{m}}$, with
		$\deg(\gamma) \ne 0$, such that the eigenvalues of
		$\phi(\gamma)$ are units.
	\item For every $\gamma \in \Gamma_{\mathfrak{m}}$, one has
		$\Tr\phi(\gamma) \in \Z_\ell$.
	\end{enumerate}

\item Let $\phi\colon \widehat{\Gamma}_{\mathfrak{m}} \to \Aut(V_\ell)$ be a
	rational $\ell$-adic representation of $K$.  Show that, for almost all
	prime number $\ell'$, there is a rational $\ell'$-adic representation
	of $K$ compatible with $\phi$. Show that this holds for all $\ell' \ne
	p$ if and only if the following property is valid: for all $\gamma \in
	\Gamma_{\mathfrak{m}}$, the coefficients of the characteristic
	polynomial of $\phi(\gamma)$ are $p$-integers.
\end{enumerate}

\section{The case of a composite of quadratic fields}
\label{sec:III_3}
\dpage

\subsection{Statement of the result}
\label{sec:III_31}

\begin{thm}
Let $\rho$ be a rational, semi-simple, $\ell$-adic abelian representation of
$K$. Assume

\begin{center}
	\textnormal{(*)} $K$ is a composite of quadratic extensions of $\Q$.
\end{center}
\todo[pinktask]{¿está bien el formato de (*)?}

Then $\rho$ is locally algebraic (and hence stems from a linear representation
of some $S_\mathfrak{m}$, cf.\ \ref{sec:III_23}).
\end{thm}

This applies in particular when $K = \Q$ or when $K$ is quadratic over $\Q$.

\begin{obs}
\begin{enumerate}
	\item An analogous result holds for $E$-rational semi-simple abelian
$\lambda$-adic representations (cf.\ \ref{sec:III_24}).
	\item It is quite likely that condition (*) is not necessary. But
proving this seems to require stronger results on transcendental numbers than
the ones now available.
\end{enumerate}
\end{obs}

\subsection{A criterion for local algebraicity}
\label{sec:III_32}
\begin{prop}
	Let $\rho\colon \abcl{\Gal(\algcl K/K)} \to \Aut(V_\ell)$ be a rational
	semi-simple $\ell$-adic abelian representation of $K$. Assume that
	there exists an integer $N \ge 1$ such that $\rho^N$ is locally
	algebraic. Then $\rho$ is locally algebraic.
\end{prop}
\begin{proof}
	\dpage
	Since $\rho$ is semi-simple, it can be brought in diagonal form over a
	finite extension of $\Q_\ell$, hence gives rise to a family $\{ \psi_1,
	\dots, \psi_n \}$ of $n$ continuous characters $\psi_i\colon C_K \to
	\algcl{\Q}_\ell^\times$, where $C_K$ is the idèle-class group of $K$,
	and $n = \dim V_\ell$.
	Let $\chi_1 = \psi_1^N, \dots, \chi_n = \psi_n^N$ be the corresponding
	characters occurring in $\rho^N$. Since $\rho^N$ is locally algebraic,
	to each $\chi_i^N$ corresponds an algebraic character $\chi_i^{\rm alg}
	\in X(\TT)$ of the torus $\TT$ (here we identify $X(\TT)$ with
	$\Hom(\TT_{/\algcl\Q_\ell}, \GG_{m, \algcl\Q_\ell})$, since
	$\algcl\Q_\ell$ is algebraically closed). Each $\chi_i^{\rm alg}$ is of
	the form $\prod_{\sigma \in \Gamma} [\sigma]^{n_\sigma(i)}$, where
	$\Gamma$ is the set of embeddings of $K$ into $\algcl\Q_\ell$, cf.\ 
	Chap.~\ref{ch:ii}, \ref{sec:II_11}. One has
	\[
		\chi_i(x) = \chi_i^{\rm alg}(x^{-1}) = \prod_{\sigma \in
		\Gamma} \sigma(x)^{-n_\sigma(i)}
	\]
	for all $x \in K_\ell^\times$ close enough to 1.
\end{proof}

\begin{lem}
	All the integers $n_\sigma(i)$, $1 \le i \le n$, $\sigma \in \Gamma$,
	are divisible by $N$.
\end{lem}
\begin{proof}
	Let $U$ be an open subgroup of $\algcl{\Q}_\ell^\times$ containing no
	$N$\textsuperscript{th}-root of unity except 1, and let $\mathfrak{m}$
	be a modulus of $K$ such that $\psi_i(x) \in U$ for all $x \in
	U_{\mathfrak{m}}$ and $i = 1, \dots, n$; the existence of such an
	$\mathfrak{m}$ follows from the continuity of $\psi_1, \dots, \psi_n$.
	We take $\mathfrak{m}$ large enough so that:
	\begin{enumerate}[a)]
	\item It is a modulus of definition for $\rho^N$.
	\item $\rho$ is unramified at all $v \in \Supp(\mathfrak{m})$, and the
		corresponding Frobenius elements $F_{v, \rho}$ have a
		characteristic polynomial with
		\dpage
		rational coefficients.
	\end{enumerate}
	Let $K_{\mathfrak{m}}$ be the abelian extension of $K$ corresponding to
	the open subgroup $K^\times U_{\mathfrak{m}}$ of the idèle group $I$,
	and let $L$ be a finite Galois extension of $\Q$ containing
	$K_{\mathfrak{m}}$. Choose a prime number $p$ which is distinct from 1,
	is not divisible by any place of $\Supp(\mathfrak{m})$, and splits
	completely in $L$. Let $v$ be a place of $K$ dividing $p$, and let
	$f_v$ be an idèle which is a uniformizing element at $v$ and is equal
	to 1 elsewhere. The fact that $v$ splits completely in
	$K_{\mathfrak{m}}$ (since it does in $L$) implies that $f_v$ is the
	norm of an idèle of $K_{\mathfrak{m}}$, hence (by class-field theory)
	belongs to $K^\times U_{\mathfrak{m}}$; this means that the prime ideal
	$\mathfrak{p}_v$ is a principal ideal $(\alpha)$, with $\alpha \equiv 1
	\mod{\mathfrak{m}}$ and $\alpha$ positive at all real places of $K$.

	Let $x = \psi_i(f_v)$ and $y = \chi_i(f_v)$, so that $y = x^N$; these
	are the Frobenius elements of $\psi_i$ and $\chi_i$ relative to $v$. By
	definition of $\chi_i^{\rm alg}$. we have
	\[
		y = \chi_i^{\rm alg}(\alpha) = \prod_{\sigma \in \Gamma}
		\sigma(\alpha)^{n_\sigma(i)}
	\]
	where $\alpha$ is as above.

	Hence $y$ belongs to the subfield $\widetilde{L}$ of $\Q$ corresponding
	to $L$ (this field is well defined since $L$ is a Galois extension of
	$\Q$).  Moreover, if $w_\sigma$ is any place of $L$ such that $w_\sigma
	\circ \sigma$ induces $v$ on $K$, we have (as in chap.~\ref{ch:ii},
	\ref{sec:II_34}):
	\[
		w_\sigma(y) = n_\sigma(i).
	\]
	Assume now that $n_\sigma(i)$ is not divisible by $N$. Then $x$, which
	is an $N$\textsuperscript{th}-root of $y$, does not belong to
	$\widetilde{L}$. Hence there is a
	\dpage
	non-trivial $N$\textsuperscript{th}-root of unity $z$ such that $x$ and
	$zx$ are conjugate over $\widetilde{L}$, and \emph{a fortiori} over
	$\Q$. Since the characteristic polynomial of $F_{v, \rho}$ has rational
	coefficients, any coniugate over $\Q$ of an eigenvalue of $F_{v, \rho}$
	is again an eigenvalue of $F_{v, \rho}$. Hence, there exists an index
	$j$ such that
	\[
		\psi_j(f_v) = z\, x = z\, \psi_i(f_v).
	\]
	But $f_v \in K^\times U_{\mathfrak{m}}$ and all $\psi_j$ are trivial on
	$K^\times$ and map $U_{\mathfrak{m}}$ into the open subgroup $U$ we
	started with. Hence $z = \psi_j(f_v) \, \psi_i(f_v)^{-1}$ belongs to
	$U$, and this contradicts the way $U_{\mathfrak{m}}$ has been chosen.
\end{proof}

\begin{proof}[ of the proposition]
	Since the $n_\sigma(i)$ are divisible by $N$, there exist $\varphi_i
	\in X(\TT)$ with $\varphi_i^N = \chi_i^{\rm alg}$. If $x \in
	K_\ell^\times$, we have:
	\[
		\varphi_i(x^{-1})^N = \chi_i^{\rm alg}(x^{-1}) = \chi_i(x) =
		\psi_i(x)^N
	\]
	if $x$ is close enough to 1. Hence $\varphi_i(x) \psi_i(x)$ is an
	$N$\textsuperscript{th}-root of unity when $x$ is close enough to 1,
	and, by continuity, it is equal to 1 in a neighbourhood of 1. Hence,
	the restriction of $\rho$ to $K_\ell^\times$ is locally equal to
	$\varphi^{-1}$, where $\varphi$ is the (algebraic) representation of
	$\TT$ defined by the family $(\varphi_1, \dots, \varphi_n)$. The
	representation $\varphi$, \emph{a priori} defined over $\algcl\Q_\ell$,
	can be defined over $\Q_\ell$ (and even over $\Q$); this follows, for
	instance, from the fact that the family $(\varphi_1, \dots, \varphi_n)$
	is \emph{stable} under the action of $\Gal(\algcl\Q/\Q)$, since the
	family $(\chi_1^{\rm alg}, \dots, \chi_n^{\rm alg})$ is.

	Hence $\rho$ is locally algebraic.
\end{proof}

\subsection{An auxiliary result on tori}
\label{sec:III_33}
In \cite{15}, Lang proved that two exponential functions $\exp(b_1 z)$,
$\exp(b_2z)$, $b_1, b_2 \in \C$, which take algebraic values for at least three
$\Q$-linearly independent values of $z$, are multiplicatively dependent: the
ratio $b_1/b_2$ is a rational number. This had also been noticed by Siegel.

Lang proved the following $\ell$-adic analogue:
\begin{prop}\label{prop:III_33_1}
Let $E$ be a field containing $\Q_\ell$ and complete for a real valuation
extending the valuation of $\Q_\ell$. Let $b_1, b_2 \in E$ and let $\Gamma$ be
an additive subgroup of $E$. Assume:
\begin{enumerate}
	\item $\Gamma$ is of rank at least 3 over $\Z$.
	\item The exponential series $\exp(z) = \sum_{n=1}^{\infty} z^n/n!$
		converges absolutely on $b_1 \Gamma$ and $b_2 \Gamma$.
	\item For all $z \in \Gamma$ the elements $\exp(b_1z)$ and $\exp(b_2z)$ are
	algebraic over $\Q$.
\end{enumerate}
Then $b_1$ and $b_2$ are linearly dependent over $\Q$ (i.e.\ $b_1/b_2$ belongs
to $\Q$ if $b_2 \ne 0$).
\end{prop}
For the proof, see \cite{15}, Appendix, or \cite{30}, \S 1.

We will apply this result to tori, taking for $E$ the completion
of $\algcl{\Q}_\ell$. We need a few definitions first:
\begin{enumerate}[a/]
\item Let $T$ be an $n$-dimensional torus over $\Q$, with character
	group $X(T)$. As before, we identify $X(T)$ with the group of morphisms
	of $T_{/E}$ into $\GG_{m, E}$. We say that \emph{$T$ is a sum of
	one-dimensional tori} if there exist one-dimensional subtori $T_i$ of
	$T$, $1 \le i \le n$, such that the sum map $T_1 \times \cdots \times
	T_n \to T$ is surjective (and hence has a finite kernel). An equivalent
	condition is:
	\begin{displayquote}
		\slshape
		$X(T) \otimes \Q$ is a direct sum of one-dimensional
		\dpage
		subspaces stable by $\Gal(\algcl\Q/\Q)$.
	\end{displayquote}

\item Let $f$ be a continuous homomorphism of $T(\Q_\ell)$ into $E$. We say
	that $f$ is \strong{locally algebraic}\index{Locally algebraic
	(homomorphism)} if there is a neighbourhood $U$ of 1 in the $\ell$-adic
	Lie group $T(\Q_\ell)$, and an element $\varphi \in X(T)$ such that
	$f(x) = \varphi(x)$ for all $x \in U$. We say that $f$ is
	\strong{almost locally algebraic}\index{Almost locally algebraic
	(homomorphism)} if there is an integer $N \ge 1$ such that $f^N$ is
	locally algebraic.

\item Let $S$ be a finite set of prime numbers, and, for each $p \in S$, let
	$W_p$ be an open subgroup of $T(\Q_p)$; denote by $W$ the family
	$(W_p)_{p\in S}$.
\end{enumerate}
Let $T(\Q)_W$ be the set of elements $x \in T(\Q)$ whose images in
$T(\Q_p)$ belong to $W$ for all $p \in S$; this is a subgroup of $T(\Q)$.
With these notations, we have:

\begin{prop}\label{prop:III_33_2}
	Let $f\colon T(\Q_\ell) \to E^\times$ be a continuous homomorphism. Assume:
	\begin{enumerate}[(a)]
	\item There exists a family $W = (W_p)_{p\in S}$ such that $f(x)$ is
		algebraic over $\Q$ for all $x \in T(\Q)_W$.
	\item\label{item:III_33_2b}
		$T$ is a sum of one-dimensional tori.
	\end{enumerate}
	Then $f$ is almost locally algebraic.
\end{prop}
\begin{proof}
\begin{enumerate}[i)]
\item\label{item:III_33_2i}
	We suppose first that $T$ is \emph{one-dimensional}, and we denote by
	$\chi$ a generator of $X(T)$. If $\chi$ is invariant by
	$\Gal(\algcl\Q/\Q)$, $T$ is isomorphic to $\GG_m$ and $T(\Q) \cong
	\Q^\times$. If not, $\Gal(\algcl\Q/\Q)$ acts on $X(T)$ \emph{via} a group of
	order 2, corresponding to some quadratic
	\dpage
	extension $F$ of $\Q$; the character $\chi$ defines an isomorphism of
	$T(\Q)$ onto the group $F_1^\times$ of elements of $F$ of norm 1. In both
	cases, one sees that $T(\Q)$ is an abelian group of \emph{infinite
	rank} (for a more precise result, see Exercise below). On the other
	hand, each quotient $T(\Q_p)/W_p$ is a finitely generated abelian group
	of rank $\le 1$.  Hence $T(\Q)/T(\Q)_W$ is finitely generated, and this
	implies that $T(\Q)_W$ is also of \emph{infinite rank}.

	Since $T(\Q_\ell)$ is an $\ell$-adic Lie group of dimension 1, it is
	locally isomorphic to the \emph{additive group} $\Q_\ell$. This means
	that there exists a homomorphism
	\[
		e \colon \Z_\ell \longrightarrow T(\Q_\ell)
	\]
	which is an isomorphism of $\Z$ onto an open subgroup of $T(\Q_\ell)$.
	By composition we get two continuous homomorphisms
	\[
		f \circ e\colon \Z_\ell \longrightarrow E^\times, \qquad
		\chi \circ e\colon \Z_\ell \longrightarrow E^\times.
	\]
	But any continuous homomorphism of $\Z$ into $E^*$ is locally an
	exponential. This implies that, after replacing $\Z_\ell$ by $\ell^m
	\Z_\ell$ if necessary, there exist $b_1, b_2 \in E$ such that
	\[
		f \circ e(z) = \exp(b_1 z), \qquad
		\chi\circ e(z) = \exp(b_2 z),
	\]
	with absolute convergence of the exponential series.

	Let now $\Gamma$ be the set of elements $z \in \Z_\ell$ such that $e(z)
	\in T(\Q)_W$. Since $T(\Q_\ell)/e(\Z_\ell)$ is finitely generated, and
	$T(\Q)_W$ is of infinite rank, $\Gamma$ is of infinite rank. If $z \in
	\Gamma$, $e(z)$
	\dpage
	belongs to $T(\Q)_W$, hence $f \circ e(z)$ is algebraic over $\Q$; the
	same is true for $\chi\circ e(z)$ since $\chi$ maps $T(\Q)$ either into
	$\Q^\times$ or into the group $F$ defined above.
	Proposition~\ref{prop:III_33_1} then shows that $b_1/b_2$ is rational.
	This means that some integral power $f^N$ of $f$, with $N \ge 1$, is
	locally equal to an integral power of $\chi$, hence $f$ is \emph{almost
	locally algebraic}.

\item \emph{General case.} Write $T = T_1 \cdots T_n$ where $T_1, \dots, T_n$
	are one-dimensional subtori of $T$. Since $X(T) \otimes \Q$ is the
	direct sum of the $X(T_i) \otimes \Q$, it is enough to show that, for
	all $i$, the restriction $f_i$ of $f$ to $T_i(\Q_\ell)$ is almost
	locally algebraic. But we may choose open subgroups $W_{i, p}$ of
	$T_i(\Q_p)$ such that $W_{1,p} \cdots W_{n,p} \subset W_p$. If we put
	$W_i = (W_{i,p})_{p\in S}$, we then see that $f_i$ takes algebraic
	values on $T_i(\Q)_{W_i}$, hence is almost locally algebraic by
	\ref{item:III_33_2i} above. \qedhere
\end{enumerate}
\end{proof}

\begin{obs}
	If one could suppress condition \ref{item:III_33_2b} from
	Prop.~\ref{prop:III_33_2}, all the results of this \S{} would extend to
	arbitrary number fields. This would be possible if one had a
	sufficiently strong $n$-dimensional version of
	Prop.~\ref{prop:III_33_1} above; the one given in \cite{30}, \S 2 does
	not seem strong enough (it requires density properties which are
	unknown in the case considered here). $\to$ [This has been done by
	Waldschmidt: see \cite{63}, \cite{83}.]
\end{obs}

\subsubsection*{Exercise}
Let $T$ be a non-trivial torus over $\Q$. Show that $T(\Q)$ is the direct sum
of a finite group and a free abelian group of infinite rank.

\dpage 

\subsection{Proof of the theorem}
\label{sec:III_34}

We go back to the notations and hypotheses of \ref{sec:III_31}. Let
\[
	\rho \colon \abcl{\Gal(algcl{K}/K)} \rightarrow \Aut(V_\ell)
\]
be a rational, semi-simple, abelian $\ell$-adic representation of $K$. If $E$ is
the completion of $ \overline{\Q_\ell} $, as in \ref{sec:III_33}, we may bring 
$\rho$ in diagonal form over $E$. This gives rise to a family $(\psi_1,
\hdots,\psi_n)$ of continuous characters of $\abcl{\Gal(algcl{K}/K)}$ (hence
also of the idèle group $I$) into $E^\times$; here, $n = \dim V_\ell$.

Let $f_i \colon  K_\ell^\times \to E^\times$ be the restriction of $\psi_i$ to
the $\ell^{\text{th}}$- component $K^{\times}_\ell$ of $I$. Note that
$K^\times_\ell = \TT(\Q_\ell)$, where $\TT$ is, as usual, the torus defined by
$K$ (chap. \ref{ch:ii}, \ref{sec:II_11}).

\begin{lem}
The torus $\TT$ and the homomorphism $f_i$ satisfy the assumptions (a) and (b)
of Prop.~2, \ref{sec:III_33}. 
\end{lem}

\subsubsection*{Verification of (a)}

Let $S$ be a finite set of primes, with $\ell \not\in S$, such that if $v \in
M_K$, $p_v \neq \ell$, $p_v \not\in S$, the representation $\rho$ is unramified
at $v$, and the characteristic polynomial of $F_{v,\rho}$ has rational
coefficients. If $p
\in S$, Prop.~1 of \ref{sec:III_22} shows that there exists an open subgroup
$W_p$ of $K_p^\times = \TT(\Q_p)$ such that $\psi_i(W_p) = 1$. Let $W =
(W_p)_{p \in S}$ and let $x \in \TT(\Q)_W$. Since $x \in K$, we have $\psi_i(x)
= 1$, when $x$ is identified with an idèle of $K$. On the other hand, let us
split the idèle $x$ in its components.
\[
	x = x_\infty \cdot x_\ell \cdot x_S \cdot x'
\]
\dpage
according to the descomposition of $I$ in
\[ I = K_\infty^\times \times K_\ell^\times \times K_S^\times \times I'.\]

(Here $K^\times = (K \otimes R)^\times$, $K_S^\times = \prod_{p \in S}
K^\times_p$ and $I'$ is the restricted product of the $K_v^\times$, for $v \in
M_K$, and $p_v \neq \ell$, $p_v \not\in S$.) The relation $\psi_i(x)=1$,
together with $\psi_i(x_\ell) = f_i(x)$, gives
\[
	f_i(x)^{-1} = \psi_i(x_\infty) \psi_i(x_S) \psi_i(x').
\]

By construction, we have $\psi_i(x_S) = 1$ and it is clear that
$\psi_i(x_\infty) = \pm 1$.

Hence:
\[
	f_i(x) = \pm \psi_i(x')^{-1}.
\]

But, for each $v \in M_K$, with $p_v \not\in S$, $p_v \neq \ell$, we know that
the eigenvalues of $F_{v,\rho}$ are algebraic; hence, if $f_v$ is an idèle which
is a uniformizing element at $v$, and is equal to $1$ elsewhere, $\psi_i(f_v)$
\emph{is algebraic}. If $a(v)$ is the valuation of $x'$ at $v$, we have:
\[
	\psi_i(x') = \prod \psi_i(f_v)^{a(v)};
\]
hence $\psi_i(x')$ and $f_i(x)$ are algebraic and we have checked (a).

\subsubsection*{Verification of (b)}

Since $K$ is a composite of quadratic fields, it is Galois extension of $\Q$,
and its Galois group $G$ is a product of groups of order $2$. The character
group $X(\TT)$ of $\TT$ is isomorphic to the 
\dpage 
regular representation of $G$, and it is clear that $X(\TT) \otimes \Q$ splits
as a direct sum of one-dimensional $G$-stable subspaces (each correspond to a
character of $G$). Hence $\TT$ is a sum of one-dimensional tori.

\subsubsection*{End of the proof of the theorem}

\begin{proof}
Using prop.~2 of \ref{sec:III_33}, we see that each $f_i$ is \emph{almost
locally algebraic}. Hence there is an integer $N \geq 1$ such that the
$f_i^{N}$ are locally algebraic. This implies, cf.\ \ref{sec:III_11}, that
$\rho^N$ is locally algebraic, hence (cf.\ \ref{sec:III_32}) that $\rho$ itself
is locally algebraic. 
\end{proof}

\subsubsection*{Exercise}

Assume that $K$ is a composite of quadratic fields. Let $\chi$ be a
\strong{Grössencharakter} of $K$ and suppose that the values of $\chi$ (on the
ideals prime to the conductor) are algebraic numbers. Show that $\chi$ is ``of
type of $(A)$'' in the sense of Weil \ref{41}. (Use the same method than above,
with $E$ replaced by C.) If the values of $\chi$ lie in a finite extension of
$\Q$, show that $\chi$ is ``of type $(A_\circ)$''.

\begin{subappendices}

\section{Hodge-Tate decompositions and locally algebraic representations}
\label{sec:III_App}
Let $K$ be a field of characteristic zero, complete with respect
to a discrete valuation and with perfect residue field $k$ of 
characteristic $p > 0$. In this Appendix we deal with Hodge-Tate 
decomposition of $p$-adic abelian representations of $K$.

Sections \ref{sec:III_A1} and \ref{sec:III_A2} give invariance properties of
\dpage
these decompositions under ground field extensions. Special characters of
$\Gal(\algcl K/K)$ are defined in \ref{sec:III_A4}; they are closely connected
both with Hodge-Tate modules (\ref{sec:III_A4} and \ref{sec:III_A5}) and local
algebraicity (\ref{sec:III_A6}). The proof of Tate's theorem (cf.\
\ref{sec:III_12}) is given in the last section.

\subsection{Invariance of Hodge-Tate decompositions}
\label{sec:III_A1}
Let $C$ be the completion of $\algcl K$ (cf.\ \ref{sec:III_12}); the group
$\Gal(\algcl K/K)$ acts continuously on $C$. Let $\chi$ be the character of
$\Gal(\algcl K/K)$ into the group of $p$-adic units defined in
chap.~\ref{ch:i}, \ref{sec:I_12}.  Let $K' / K$ be a subextension of $\algcl K/
K$ on which the valuation $\overline{v}$ of $\algcl K$ is discrete; this means
that $K'$ is a finite extension of an unramified one of $K$. Let $\widehat{K'}$
denote the closure of $K'$ in $C$.

Let now $W$ be a finite dimensional $C$-vector space on which $\Gal(\algcl
K/K)$ acts continuously and semi-linearly (see \ref{sec:III_12}). As before, we
denote by $W^n$ (resp.\ $W^n_{K'}$) the $K$- (resp.\ $\widehat{K'}$-)vector
space defined by
\begin{multline*}
	W^n = \{ w \in W \mid s(w) = \chi(s)^n w \text{ for all } s \in
	\Gal(\algcl K / K) \\
	\text{(resp.\ $s \in \Gal(\algcl K / K')$)} \}
\end{multline*}
Let $W(n) = C \otimes_K W^n$ and $W(n)' = C \otimes_{\widehat{K'}} W^n_{K'}$.
Identifying the modules $W(n)$ and $W(n)'$ with their canonical images in $W$,
we prove

\begin{thm}\label{thm:III_A1_1}
	The canonical map $\widehat{K'} \otimes_K W^n \to W^n_{K'}$ is a
	$\widehat{K'}$-isomorphism.
\end{thm}
\begin{cor}
	The Galois modules $W(n)$ and $W(n)'$ are equal.
	\dpage
\end{cor}
Indeed, Theorem~\ref{thm:III_A1_1} shows that $W^n$ and $W^n_{K'}$, generate
the same $C$-vector subspace of $W$.

\begin{cor}\label{cor:III_A1_12}
	The Galois module $W$ is of Hodge-Tate type over $K$ if and only if it
	is so over $\widehat{K'}$.
\end{cor}
\begin{proof}[ of Theorem~\ref{thm:III_A1_1}]
	Note first that replacing the action of $\Gal(\algcl K/K)$ on $W$ by
	$(s,w) \mapsto \chi(s)^{-i}sw$, $i \in \Z$, just changes $W^n$ to
	$W^{n+i}$. This shifting process reduces the problem to the case $n =
	0$; in that case, $W^n$ (resp.\ $W^n_{K'}$) is the set of elements of
	$W$ which are invariant under $\Gal(\algcl K/K)$ (resp.\ under
	$\Gal(\algcl K/K')$). Note also that the injectivity of $\widehat{K'}
	\otimes W^0 \to W^0_{K'}$ is trivial, since we know that $C \otimes_K
	W^0 \to W$ is injective (cf.\ \ref{sec:III_12}).

	On the other hand, an easy up-and-down argument shows that one can
	assume $K'/K$ to be either \emph{finite Galois} or \emph{unramified
	Galois}. In both cases, since $\Gal(\algcl K/K')$ acts trivially on
	$W^0_{K'}$, we have a semi-linear action of $\Gal(K'/K)$ on $W_{K'}^0$.
	When $K'/K$ is finite, it is well known that this implies that
	$W_{K'}^0$, is generated by the elements invariant by $\Gal(K'/K)$,
	i.e.\ by $W^0$ (this is a non-commutative analogue of Hilbert's
	``Theorem 90'', cf.\ for instance \cite[159]{29}).

	Let now $K'/K$ be unramified Galois and let $G$ be its Galois group.
	Let $\widehat{\mathcal{O}'}$ denote the ring of integers of
	$\widehat{K'}$. Let $\Lambda$ be an $\widehat{\mathcal{O}'}$-lattice of
	$W_{K'}^0$ (i.e.\ a free $\widehat{\mathcal{O}'}$-submodule of
	$W_{K'}^0$ of the same rank as $W_{K'}^0$). Since $G$ acts continuously
	on $W_{K'}^0$, the stabilizer in $G$ of $\Lambda$ is open, hence of
	finite index, and the lattice $\Lambda$ has finitely many transforms.
	\dpage
	The sum $\Lambda^0$ of these transforms is invariant by $G$. Let $e_1,
	\dots, e_N$ be a basis of $\Lambda^0$. Let $s \in G$.  Then
	\[
		s(e_j) = \sum_{i=1}^{N} a_{ij}(s) e_i,
		\qquad a_{ij} \in \widehat{\mathcal{O}'}
	\]
	and the matrix $a(s) = (a_{ij}(s))$ belongs to $\GL(N,
	\widehat{\mathcal{O}'})$. We have $a(st) = a(s) \, s(a(t))$; this means
	that $a$ is a \emph{continuous $1$-cocycle on $G$ with values in
	$\GL(N, \widehat{\mathcal{O}'})$}. Recall that two such cocycles $a$
	and $a'$ are said to be cohomologous if there exists $b \in \GL(N,
	\widehat{\mathcal{O}'})$ such that $a'(s) = b^{-1} a(s) \, s(b)$ for
	all $s \in G$. This is an equivalence relation on the set of cocycles
	and the corresponding quotient space is denoted by $H^1(G, \GL(N,
	\widehat{\mathcal{O}'}))$. In fact:
	\begin{lem}
		$H^1(G, \GL(N, \widehat{\mathcal{O}'})) = \{ 1 \}$.
	\end{lem}
	Assuming the lemma, the proof of the theorem is concluded as follows.
	Since $a(s)$ is cohomologous to $1$, there exists $b \in \GL(N,
	\widehat{\mathcal{O}'})$ such that $b = a(s) \, s(b)$ for all $s \in
	G$. If $b = (b_{ij})$, define a new basis $e_1^\prime, \dots,
	e_N^\prime$ of $W^0_{K'}$ by
	\[
		e_j^\prime = \sum_{i=1} b_{ij} e_i.
	\]
	Using the identity $b = a(s) \, s(b)$, one sees that $e_1^\prime,
	\dots, e_N^\prime$ are invariant under $G$, hence belong to $W^0$; this
	proves the surjectivity of $\widehat{K'} \otimes_K W^0 \to W_{K'}^0$.
\end{proof}

\begin{proof}[ of the lemma]
	Let $\pi$ be a uniformizing element of $\widehat{\mathcal{O}'}$. Filter
	\dpage
	the ring $A = \GL(N, \widehat{\mathcal{O}'})$ by means of $A_n = \{a
	\in A \mid a \equiv 1 \mod{\pi^n} \}$. We get $A/A_1 \cong \GL(N, k' /
	k)$, where $k' / k$ is the residue field extension of $K' / K$.
	Moreover, for $n \ge 1$, there is an isomorphism \label{errata:An+k}
	$A_n / A_{n+1} \cong \Mat_N(k')$, where $\Mat_N(k')$ is the additive
	group of $N \times N$ matrices with coefficients in $k'$. The lemma
	follows now from the triviality of $H^1(G, \GL(N, k'))$ and $H^1(G,
	\Mat_N(k'))$, where now $k'$ is endowed with the discrete topology (so
	this is ordinary Galois cohomology, cf.\ \cite[158-159]{29}).
\end{proof}

\subsection{Admissible characters}
\label{sec:III_A2}
Let $G = \Gal(\algcl K/K)$ and let $\varphi\colon G \to K^\times$ be a
continuous homomorphism.
\begin{mydef}
	The character $\varphi$ is said to be \strong{admissible}%
	\index{Admissible (character)} (notation: $\varphi \sim 1$) if there
	exists $x \in C$, $x \ne 0$, such that $s(x) = \varphi(s)\, x$ for all
	$s \in G$.
\end{mydef}

\begin{obs}
\begin{enumerate}
\item The admissible characters form a subgroup of the group of all characters
	of $G$ with values in $K^\times$; if $\varphi$, $\varphi'$ are two
	characters, we write $\varphi \sim \varphi'$ if $\varphi^{-1} \varphi'
	\sim 1$.
\item Let $H^1(G, C^\times)$ be the first cohomology group of $G$ with values
	in $C$ (cohomology being defined by \emph{continuous} cochains, as in
	\ref{sec:III_A1}). A continuous character $\varphi \colon G \to
	K^\times$ is a 1-cocycle, hence defines an element $\overline{\varphi}$
	of $H^1(G, C^\times)$. One has $\overline{\varphi} =
	\overline{\varphi'}$ if and only if $\varphi \sim \varphi'$.

\item\label{rmk:III_A2_3}
	Define a new action of $G$ on $C^\times$ by means of
	\dpage
	\[
		(s, c) \longmapsto \varphi(s)\, s(c),
		\qquad s\in G,\; c \in C,
	\]
	Denote the $C$-$G$-module thus obtained by $C(\varphi)$. Then $\varphi$
	is admissible if and only if $C(\varphi)$ and $C$ are isomorphic as
	$C$-$G$-modules.
\end{enumerate}
\end{obs}

\begin{prop}
	Suppose there exists $c \in C^\times$ such that $\varphi(s) = s(c)/c$
	for $s$ in some open subgroup $N$ of the inertia group of $G$. Then
	$\varphi$ is admissible.
\end{prop}
\begin{proof}
	Let $K'/K$ be the subextension of $\algcl K/K$ corresponding to $N$; it
	is a finite extension of an unramified one. Let $W = C(\varphi)$, as in
	Remark~\ref{rmk:III_A2_3}, and let $W^0$ (resp.\ $W^0_{K'}$) be the
	subspace of $W$ consisting of elements invariant by $G$ (resp.\ by
	$N$). By hypothesis, $W_{K'}^0$ is $\ne 0$. Hence, by \ref{sec:III_A1},
	Theorem~\ref{thm:III_A1_1}, we also have $W^0 \ne 0$,
	and this means that $\varphi$ is admissible.
\end{proof}

Let now $U_C$ be the group of units of $C$, $U_C^1$ the subgroup of units
congruent to $1$ modulo the maximal ideal, and identify ${\algcl k}^\times$
with the group of multiplicative representatives, so that $U_C = U_C^1 \times
{\algcl k}^\times$, cf.\ \cite[44]{29}. Define the logarithm map by
\begin{align*}
	\log \colon U_C &\longrightarrow C \\
	x &\longmapsto
	\begin{cases}
		0, & \text{if } x \in {\algcl k}^\times \\
		\displaystyle \sum_{n=1}^{\infty} \frac{(-1)^{n-1}}{n} (x-1)^n, & \text{if } x \in U_C^1
	\end{cases}
\end{align*}
This is a continuous homomorphism and even a local isomorphism.

\subsection{A criterion for local triviality}
\label{sec:III_A3}
\todo[section]{Belén.}

\subsection{The character \texorpdfstring{$\xi$}{ξ}}%
\label{sec:III_A4}
\todo[section]{Belén.}

\subsection{Characters associated with Hodge-Tate decompositions}
\label{sec:III_A5}
\todo[section]{Belén.}

\begin{thm}\label{thm:III_A5_2}
Let $\rho$, $V$, $W$ be as above and, for each $\sigma \in \Gamma_E$, let
$n_\sigma$ be an integer. The following are equivalent:
\begin{enumerate}[(i), series=thm_IIIA5_2]
	\item\label{thm:III_A5_2i}
		$\displaystyle \rho \equiv \prod_{\sigma \in \Gamma_E}
		\sigma^{-1} \circ \chi_{\sigma E}^{n_\sigma},$
	\item\label{thm:III_A5_2ii}
		$\sigma\circ \rho \sim \chi^{n_\sigma}$ for all $\sigma \in
		\Gamma_E$,
	\item\label{thm:III_A5_2iii}
		for every $\sigma \in \Gamma_E$ the Galois-module $W_\sigma$ is
		isomorphic to $C(\chi^{n_\sigma})$.
\end{enumerate}
\end{thm}

\subsection{Locally compact case}
\label{sec:III_A6}
We now add to all the previous assumptions regarding $K$ and
$E$, the assumption that $K$ is \emph{finite} over $\Q$ (i.e.\ $K$ is locally
compact). By local class field theory, we may then identify $\abcl G$
with $\widehat{K}^\times$, and the inertia subgroup of $\abcl G$ with $U_K$, the group
of units of $K$.

Let $T$ (resp.\ $T_E$, $T_{\sigma E}$) be the $\Q_p$-torus associated to $K$
(resp.\ to $E$, $\sigma E$, where $\sigma \in \Gamma_E$), cf.\
\ref{sec:III_11}. The norm map from $K$ to $\sigma E$ defines an algebraic
morphism
\[
	\Nm_{K/\sigma E} \colon T \longrightarrow T_{\sigma E}.
\]
By composition with $\sigma^{-1} \colon T_{\sigma E} \to T_E$,
\dpage
this gives a morphism
\[
	r_\sigma = \sigma^{-1} \circ \Nm_{K/\sigma E} \colon T \to T_E.
\]
\begin{prop}\label{prop:III_A6_5}
\begin{enumerate}[(a)]
	\item\label{prop:III_A6_5a}
		$r_\sigma(u^{-1}) = \sigma^{-1}\circ\chi_{\sigma E}(u)$ for all
		$u \in U_K$,
	\item\label{prop:III_A6_5b}
		the $r_\sigma$ ($\sigma \in \Gamma_E$) make a $\Z$-basis of
		$\Hom_{\rm alg}(T, T_E)$.
\end{enumerate}
\end{prop}
(Note that \ref{prop:III_A6_5a} makes sense, since $U_K$ has been identified with
the inertia group of $\abcl G$.)

Assertion \ref{prop:III_A6_5a} follows from the remark at the end of
\ref{sec:III_A4}. On the other hand, let $X(T)$ and $X(T_E)$ be the character
groups of $T$ and $T_E$ respectively. The characters $[s]$, $s \in \Gamma_K$
(resp.\ $(\sigma)$, $\sigma \in \Gamma_E$) make a basis of $X(T)$ (resp.\ of
$X(T_E)$). The morphism $r_\sigma \colon T \to T_E$ defines by transposition a homomorphism
\[
	X(r_\sigma)\colon X(T_E) \longrightarrow X(T).
\]
One checks easily that the effect of $X(r_\sigma)$ on the basis $[\tau]$, $\tau \in \Gamma_E$ is:
\[
	X(r_\sigma)\big( [\tau] \big) = \sum_{s \sigma = \tau} [s].
\]
Assertion \ref{prop:III_A6_5b} then follows from:

\begin{lem}
	The elements $X(r_\sigma)$, $\sigma \in \Gamma_E$,
	\dpage
	form a basis of $\Hom_{\Gal}(X(T_E),\break X(T))$.
\end{lem}
\begin{proof}
	The independence of the $X(r_\sigma)$ is clear. On the other hand,
	let $\varphi \in \Hom_{\Gal}(X(T_E),X(T))$ be such that
	\[
		\varphi([\tau]) = \sum_{s} n(\tau, s)[s].
	\]
	If $\alpha \in \Gal(\algcl\Q_p/\Q_p)$ is equal to the identity on $\tau
	E$, we have $\alpha[\tau] = [\tau]$, hence $\alpha \varphi([\tau]) =
	\varphi([\tau])$, i.e.\ $n(\tau, \alpha s) = n(\tau, s)$ for all $s \in
	\Gamma_K$. This means that $n(\tau, s)$ depends only on the element
	$\sigma = s^{-1} \tau$; if we put $n_\sigma = n(\tau, s)$, we then have
	\[
		\varphi([\tau]) = \sum_{\sigma \in \Gamma_E} n_\sigma
		\sum_{s\sigma = \tau} [s]
		= \sum_{\sigma \in \Gamma_E} n_\sigma X(r_\sigma)([\tau]).
	\]
	This proves the lemma.
\end{proof}

\begin{prop}\label{prop:III_A6_6}
	Let $\rho$ and $(n_\sigma)$, $\sigma \in \Gamma_E$, be as in
	Th.~\ref{thm:III_A5_2} of \ref{sec:III_A5}.  Let $r\colon T \to T_E$ be
	the morphism defined by
	\[
		r = \prod_{\sigma \in \Gamma_E} r_\sigma^{n_\sigma}.
	\]
	The equivalent properties \ref{thm:III_A5_2i}, \ref{thm:III_A5_2ii},
	\ref{thm:III_A5_2iii} of Th.~\ref{thm:III_A5_2} are equivalent to:
	\begin{enumerate}[resume*=thm_IIIA5_2]
	\item\label{thm:III_A5_2iv}
		There exists an open subgroup $U'$ of the inertia subgroup
		$U_K$ of $\abcl G$ such that $r(u) \rho(u) = 1$ if $u \in U'$.
	\end{enumerate}
\end{prop}
Indeed, \ref{thm:III_A5_2iv} is just a reformulation of \ref{thm:III_A5_2i},
since we know that $\sigma^{-1} \circ \chi_{\sigma E}(u) = r_\sigma(u^{-1})$ if
$u \in U_K$.

\begin{corp}
	The following are equivalent:
	\begin{enumerate}[(a)]
	\item $\rho$ is locally algebraic.
	\item The Galois module $V$ attached to $\rho$ is of Hodge-Tate type.
	\end{enumerate}
\end{corp}
This follows from Theorem~\ref{thm:III_A5_2}, combined with
Prop.~\ref{prop:III_A6_5} and Prop.~\ref{prop:III_A6_6}.

\subsubsection*{Exercises}
\begin{enumerate}
\item \begin{enumerate}[a)]
	\item Let $A = \End_{\Q_p}(K)$ be the space of $\Q_p$-linear
		endomorphisms of $K$; if $a \in A$, denote by $\Tr(a)$ the
		trace of $a$. If $x \in K$, denote by $u_x$ the endomorphism $y
		\mapsto xy$ of $K$. Show that, for any $a \in A$, there exists
		a unique element $c_K(a)$ of $K$ such that
		\[
			\Tr(u_x \circ a) = \Tr_{K/\Q_p}(x\cdot c_K(a)) \qquad
			\text{for all } x\in K.
		\]
	\item  Show that the map $c_K \colon A \to K$ so defined is $K$-linear
		for both the natural structures of $K$-vector space on $A$.
	\item  Let $e_i$ be a $\Q_p$-basis of $K$ and let $e_i^\prime$ be the
		\dpage
		dual basis, so that $\Tr_{K/\Q_p}(e_i e_j^\prime) =
		\delta_{ij}$. Show that
		\[
			c_K(a) = \sum_{i=1}^{n} a(e_i) e_i^\prime,
			\qquad \text{if } a \in A.
		\]
	\item If $L \supset K$ and $a \in A$, show that
		\[
			c_L(a \circ \Tr_{L/K}) = c_K(a).
		\]
		Show that $c_K(\Tr_{K/\Q_p}) = 1$.
	\item  If $K$ is a Galois extension of $\Q_p$, show that $c_K(\sigma) =
		0$ for every $\sigma \in \Gal(K/\Q_p)$, $\sigma \ne \id$, and
		$c_K(\id) = 1$.
\end{enumerate}
\item Let $\varphi \colon \abcl G \to K^\times$ be a continuous homomorphism,
	and let $a_\varphi$, be the $\Q_p$-linear endomorphism of $K$ such that
	the diagram
	\[\begin{tikzcd}
		U_K \rar["\varphi"] \dar["\log"'] & U_K \dar["\log"] \\
		K \rar["a_\varphi"] & K
	\end{tikzcd}\]
	is commutative. Let $L\overline{\varphi}$ (resp.\ $L\overline{\chi}$)
	be the image of $\varphi$ (resp.\ $\chi$) in the one-dimensional
	$K$-vector space $H^1(G,C)$, cf.\ \ref{sec:III_A2}. Show that
	\[
		L\overline{\varphi} = c \cdot L\overline{\chi},
	\]
	where $c = -c_K(a_\varphi)$. (Check the formula first when $K$ is a
	Galois extension of $\Q_p$ and $\varphi = \sigma^{-1}\circ \chi_K$,
	\dpage
	$\sigma \in \Gal(K/\Q_p)$, in which case $a_\varphi = -\sigma^{-1}$ and
	$c_K(a_\varphi)$ is given by Exer.~1, d.)

	In particular, $\varphi$ is admissible if and only if $c_K(a_\varphi) =
	0$.
\end{enumerate}

\subsection{Tate's theorem}
\label{sec:III_A7}
We recall the statement (cf.\ \ref{sec:III_12}); here again, $K$ is locally
compact.

\begin{thm}
	Let $V$ be a finite dimensional vector space over $\Q_p$ and let $\rho
	\colon G \to \Aut(V)$ be an abelian $p$-adic representation of $K$.
	The following are equivalent:
	\begin{enumerate}[(1)]
	\item\label{thm:III_A7_1} $\rho$ is locally algebraic
	\item\label{thm:III_A7_2} $\rho$ is of Hodge-Tate type and its
		restriction to the inertia group is semi-simple.
	\end{enumerate}
\end{thm}
\begin{proof}
	We have already remarked (cf.\ \ref{sec:III_11}) that
	\ref{thm:III_A7_1} implies:
	\begin{displayquote}
		\emph{\textbf{(*)} The restriction of $\rho$ to the inertia
		group is semi-simple.}
	\end{displayquote}
	Hence we may assume that (*) holds.

	Let $\pi$ be a uniformizing element of $K$, and let $\pr_\pi$ denote
	the projection map of $\abcl G$ onto its inertia group $U_K$ associated
	to $\pi$ (cf.\ \ref{sec:III_A4} and \citeauthor{6}~\cite[144-145]{6}).
	Define a new representation $\rho'$ of $\abcl G$ by
	\[
		\rho' = \rho \circ \pr_\pi.
	\]
	Replacing $\rho$ by $\rho'$ does not affect the local algebraicity
	(clear), nor the Hodge-Tate property (this follows from
	\ref{sec:III_A1}, Cor.~\ref{cor:III_A1_12} to Th.~\ref{thm:III_A1_1}).
	Since (*) implies that $\rho'$ is semi-simple, this means that, after
	\dpage
	replacing $\rho$ by $\rho'$, we may assume that $\rho$ is semi-simple
	and even (by an easy reduction) that it is \emph{simple}. Let then $E
	\subset \End(V)$ be the commuting algebra of $\rho$. Since $\rho$ is
	abelian and simple, $E$ is a commutative field, of finite degree over
	$\Q_p$, and $V$ is a one-dimensional vector space over $E$; the
	representation $\rho$ is given by a continuous character $\rho \colon G
	\to E^\times$.

	Let now $K'$ be a finite extension of $K$ which is large enough to
	contain all the $\Q_p$-conjugates of $E$. Call ($1'$) and ($2'$) the
	properties corresponding to \ref{thm:III_A7_1} and \ref{thm:III_A7_2},
	when $K'$ is taken as groundfield instead of $K$. We know (cf.\
	\ref{sec:III_11}) that \ref{thm:III_A7_1} ${}\iff (1')$. By
	Cor.~\ref{cor:III_A1_12} to Th.~\ref{thm:III_A1_1} of \ref{sec:III_A1},
	we have \ref{thm:III_A7_2} $\iff (2')$. Hence it is enough to prove
	that $(1') \iff (2')$, and this has been done in \ref{sec:III_A6} (Cor.
	to Prop.~\ref{prop:III_A6_5}).
\end{proof}

\end{subappendices}
