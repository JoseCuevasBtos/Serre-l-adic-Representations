\chapter{Locally algebraic abelian representations}
\label{ch:iii}

In this Chapter, we define what it means for an abelian $\ell$-adic
representation to be \emph{locally algebraic} and we prove (cf.\ \ref{sec:III_23}) that such
a representation, when rational, comes from a linear representation
of one of the groups $S_{\mathfrak{m}}$ of Chapter \ref{ch:ii}.

When the ground field is a composite of quadratic extensions of
$\Q$, any rational semi-simple $\ell$-adic representation is \emph{ipso facto}
locally algebraic; this is proved in \S\ref{sec:III_3}, as a consequence of a result
on transcendental numbers due to Siegel and Lang.

In the local case, an abelian semi-simple representation is
locally algebraic if and only if it has a ``Hodge-Tate decomposition''.
This fact, due to Tate (College de France, 1966), is proved in the
Appendix, together with some complements.

\section{The local case}

\subsection{Definitions}
\label{sec:III_11}
Let $p$ be a prime number and $K$ a finite extension of $\Q_p$; let $\TT =
\WRes_{K/\Q_p}(\GG_{m, K})$ be the corresponding algebraic torus over
\dpage
$\Q_p$ (cf.\ \citeauthor{43}~\cite{43}, Chap.~I).
\todo{Belen.}

\subsection{Alternative definition of ``locally algebraic'' via Hodge-Tate
modules}
\label{sec:III_12}
Let us recall first the notion of a \strong{Hodge-Tate module} (cf.\ \cite{27},
\S 2); here $K$ is only assumed to be complete with respect to a discrete
valuation, with perfect residue field $k$ and $\char(K) = 0$, $\char(k) = p$.
Denote by $C$ the \emph{completion $\widehat{\algcl K}$ of the algebraic
closure} of $K$.

The group $G = \Gal(\algcl K/K)$ acts continuously on $K$. This action extends
continuously to $C$. Let $W$ be a $C$-vector space of finite dimension upon
which $G$ acts continuously and semi-linearly according to the formula
\[
	s(cw) = s(c) \cdot s(w) \qquad
	(s \in G, \, c \in C \text{ and } w \in W).
\]
Let $\chi\colon G \to U_p$ be the homomorphism of $G$ into the group $U_p =
\Z_p^\times$ of $p$-adic units, defined by its action on the $p^\nu$-th roots
of unity (cf.\ chap.~\ref{ch:i}, \ref{sec:I_12}):
\dpage
\[
	s(z) = z^{\chi(s)} \qquad \text{if } s \in G \text{ and } z^{p^\nu} = 1.
\]
Define for every $i \in \Z$ the subspace
\[
	W^i = \{ w \in W : sw = \chi(s)^i w \text{ for all } s\in G \}
\]
of $W$. This is a $K$-vector subspace of $W$. Let $W(i) = C \otimes_K W^i$.
This is a $C$-vector space upon which $G$ acts in a natural way (i.e.\ by the
formula $s(c \otimes y) = s(c) \otimes s(y)$). The inclusion $W^i \to W$
extends uniquely to a $C$-linear map $\alpha_i\colon W(i) \to W$, which
commutes with the action of $G$.

\begin{prop}[Tate]
	Let $\coprod_{i\in \Z} W(i)$ be the direct sum of the $W(i)$. Let
	$\alpha\colon \coprod_i W(i) \to W$ be the sum of the $\alpha_i$'s
	defined above. Then $\alpha$ is injective.
\end{prop}
For the proof see \cite{27}, \S 2, prop.~4.
\begin{corp}
	The $K$-spaces $W^i$ ($i \in \Z$) are of finite dimension.
	They are linearly independent over $C$.
\end{corp}

\begin{mydef}\label{def:III_12_1}
	The module $W$ is of \strong{Hodge-Tate type}\index{Hodge-Tate module}%
	\index{Hodge-Tate type (module)}
	if the homomorphism $\alpha\colon \coprod_{i\in\Z} W(i) \to W$ is an
	isomorphism.
\end{mydef}

Let now $V$ be as in \ref{sec:III_11}, a vector space over $\Q_p$, of finite
dimension. Let $\rho\colon G \to \Aut(V)$ be a $p$-adic representation. Let $W
= C \otimes_{\Q_p} V$ and let $G$ act on $W$ by the formula
\dpage
\[
	s(c\otimes v) = s(c) \otimes s(v) \qquad
	s\in \Gamma, \; c\in C, \; v\in V.
\]
\begin{mydef}
	The representation $\rho$ is of \strong{Hodge-Tate type}%
	\index{Hodge-Tate representation}\index{Hodge-Tate type
	(representation)} if the $C$-space $W = C \otimes_{\Q_p} V$ is of
	Hodge-Tate type (cf.\ def.~\ref{def:III_12_1}).
\end{mydef}

\begin{ex}
	Let $F$ be a $p$-divisible group of finite height (cf.\ \cite{26},
	\cite{39}); let $T$ be its Tate module (\emph{loc. cit.}) and $V = \Q_p
	\otimes T$. The group $G$ acts on $V$, and Tate has proved (\cite{39},
	Cor.~2 to Th.~3) that this Galois module is of Hodge-Tate type; more
	precisely, one has $W = W(0) \oplus W(1)$, where $W = C \otimes V$ as
	above.
\end{ex}
\begin{thm}[Tate]
Assume $K$ is a finite extension of $\Q_p$ (i.e.\ its residue field is finite).
Let $\rho\colon G \to \Aut(V)$ be an abelian $p$-adic representation of $K$.
The following properties are equivalent:
\begin{enumerate}[(a)]
\item $\rho$ is locally algebraic (cf.\ \ref{sec:III_11}).
\item $\rho$ is of Hodge-Tate type and its restriction to the inertia group is
	semi-simple.
\end{enumerate}
\end{thm}
For the proof, see the Appendix.

\section{The global case}
\subsection{Definitions}
\label{sec:III_21}
\todo{Belen.}

\subsection{Modulus of a locally algebraic abelian representation}
\label{sec:III_22}
Let $\rho\colon \abcl{\Gal(\algcl K/K)} \to \Aut(V_\ell)$ be as above; by
composition with the class field homomorphism $i\colon I \to \abcl{\Gal(\algcl
K/K)}$, $\rho$ defines a homomorphism $\rho \circ i\colon I \to \Aut(V_\ell)$.

We assume that $p\rho$ is locally algebraic and we denote by $f$ the associated
\dpage
algebraic morphism $T_{/\Q_\ell} \to \GL_{V_\ell}$.
\begin{mydef}
Let $\mathfrak{m}$ be a modulus (chap.~\ref{ch:ii}, \ref{sec:II_11}).
One says that $\rho$ is defined mod $\mathfrak{m}$ (or that
$\mathfrak{m}$ is a modulus of definition for $\rho$) if
\begin{enumerate}[(i)]
	\item $\rho \circ i$ is trivial on $U_{v, \mathfrak{m}}$ when $p_v \ne
		\ell$.
	\item $\rho \circ i_\ell(x) = f(x^{-1})$ for \smash{$\displaystyle x
		\in \prod_{v\mid\ell} U_{v, \mathfrak{m}}$}.
\end{enumerate}
\end{mydef}
(Note that $\prod_{v\mid\ell} U_{v, \mathfrak{m}}$ is an open subgroup of
$K_\ell^\times = T_{/\Q_\ell}(\Q_\ell)$.)

In order to prove the existence of a modulus of definition, we
need the following auxiliary result:
\begin{prop}\label{prop:III_22_1}
Let $H$ be a Lie group over $Q_\ell$ (resp.\ $\R$) and let
$\alpha$ be a continuous homomorphism of the idèle group $I$ into $H$.
\begin{enumerate}[(a)]
\item\label{prop:III_22a}
	If $p_v \ne \ell$ (resp.\ $p_v \ne \infty$), the restriction of
	$\alpha$ to $K$ is equal to 1 on an open subgroup of $K_v^\times$.
\item\label{prop:III_22b}
	The restriction of $\alpha$ to the unit group $U_v$ of $K_v^\times$ is
	equal to 1 for almost all $v$'s.
\end{enumerate}
\end{prop}
\begin{proof}
	Part \ref{prop:III_22a} follows from the fact that $K_v^\times$ is a
	$p_v$-adic Lie group and that a homomorphism of a $p$-adic Lie group
	into an $\ell$-adic one is locally equal to 1 if $p \ne \ell$.

	To prove \ref{prop:III_22b}, let $N$ be a neighborhood of 1 in $H$
	which contains no finite subgroup except $\{ 1 \}$; the existence of
	such an $N$ is classical for real Lie groups, and quite easy to prove
	for $\ell$-adic ones. By definition of the idèle topology,
	$\alpha(U_v)$ is contained in $N$ for almost all $v$'s. But
	\ref{prop:III_22a} shows that, if $p_v \ne \ell$, the group
	\dpage
	$\alpha(U_v)$ is finite; hence $\alpha(U_v) = \{ 1 \}$ for almost all
	$v$'s.
\end{proof}

\begin{corp}\label{cor:III_22}
	Any abelian $\ell$-adic representation of $K$ is unramified outside a
	finite set of places.
\end{corp}
This follows from \ref{prop:III_22b} applied to the homomorphism $\alpha$ of $I$
induced by the given representation, since the $\alpha(U_v)$ are known to be
the inertia subgroups.
% 2d by the given r

\begin{obs}
This does not extend to non-abelian representations (even solvable ones), cf.\ Exercise.
\end{obs}
\begin{prop}
	Every locally algebraic abelian $\ell$-adic representation has a
	modulus of definition.
\end{prop}
Let $\rho\colon \abcl{\Gal(\algcl K/K)} \to \Aut(V_\ell)$ be the given
representation and $f$ the associated morphism of $T_{/\Q_\ell}$ into
$\GL_{V_\ell}$. Let $X$ be the set of places $v \in M_K^0$, with $p_v \ne
\ell$, for which $\rho$ is ramified; the corollary~\ref{cor:III_22} to
Prop.~\ref{prop:III_22_1} shows that $X$ is finite. By
Prop.~\ref{prop:III_22_1}, \ref{prop:III_22a}, we can choose a modulus
$\mathfrak{m}$ such that $\rho \circ i\colon I \to \Aut(V_\ell)$ is trivial on
all the $U_{v, \mathfrak{m}}$, $v \in X$. Enlarging $\mathfrak{m}$ if
necessary, we can assume that $\rho \circ i_\ell(x) = f(x^{-1})$ for $x \in
\prod_{p_v = \ell} U_{v, \mathfrak{m}}$. Hence, $\mathfrak{m}$ is a modulus of
definition for $\rho$.

\begin{obs}
It is easy to show that there is a smallest modulus of definition for $\rho$;
it is called the \strong{conductor}\index{Conductor} of $\rho$.
\end{obs}

\subsubsection*{Exercise}
Let $z_1, \dots, z_n, \dots \in K^\times$. For each $n$, let $E_n$ be the
\dpage
subfield of $\algcl K$ generated by all the $\ell^n$-th roots of the element
$z_1 z_2^\ell \cdots z_n^{\ell^{n-1}}$.
\begin{enumerate}[a)]
\item Show that $E_n$ is a Galois extension of $K$, containing the $\ell^n$-th
	roots of unity and that its Galois group is isomorphic to a subgroup of
	the affine group $
	\begin{psmallmatrix}
		* & * \\
		0 & 1
	\end{psmallmatrix} 
	$ in $\GL(2, \Z/\ell^n\Z)$.
\item Let $E$ be the union of the $E_n$'s. Show that $E$ is a Galois extension
	of $K$, whose Galois group is a closed subgroup of the affine group
	relative to $\Z_\ell$.
\item Give an example where $E$ (and hence the corresponding 2-dimensional
	$\ell$-adic representation) is ramified at all places of $K$.
\end{enumerate}

\subsection{Back to \texorpdfstring{$S_{\mathfrak{m}}$}{Sm}}
\label{sec:III_23}
\todo{Belen.}

\subsection{A mild generalization}
\label{sec:III_24}
\todo{Belen.}

\subsection{The function field case}
\label{sec:III_25}
