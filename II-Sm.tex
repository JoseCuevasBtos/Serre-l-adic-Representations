\chapter{The groups \texorpdfstring{$S_m$}{Sm}}

Throughout this chapter,
\dpage
$K$ denotes an algebraic number field.
We associate to $K$ a projective family $(S_m)$ of commutative 
algebraic groups over $\Q$, and we show that each $S_m$ gives rise to a
strictly compatible system of rational $\ell$-adic representations of $K$.

In the next chapter, we shall see that all ``locally algebraic''
abelian rational representations are of the form described here.

\section{Preliminaries}

\subsection{The torus $\TT$}
Let $\TT = \WRes_{K/\Q}(\GG_{m, K})$ be the algebraic group over $\Q$, obtained
from the multiplicative group $\GG_m$ by restriction of scalars from $K$ to
$\Q$, cf.\ \citeauthor{43}~\cite{43}, \S 1.3. If $A$ is a commutative
$\Q$-algebra, the points of $\TT$ with values in $A$ form by definition the
multiplicative group $(K \otimes_\Q A)^\times$ of invertible elements of $K
\otimes_\Q A$.
In particular, $\TT(\Q) = K^\times$. If $d = [K : \Q]$, the group $\TT$ is a
torus of dimension $d$; this means that the group $\TT_{/\algcl\Q} = \TT
\otimes_\Q \algcl\Q$ obtained from $\TT$ by extending the scalars from $\Q$ to
$\algcl\Q$, is isomorphic
\dpage
to...

\subsection{Cutting down $\TT$}
Let $E$ be a subgroup of $K = \TT(\Q)$ and let $\overline{E}$ be the Zariski
closure of $E$ in $\TT$. Using the formula $\overline{E} \times \overline{E} =
\overline{E \times E}$, one sees that $E$ is an algebraic subgroup of $\TT$.
Let $\TT_E$ be the quotient group $\TT/E$; then $\TT_E$ is also a torus over
$\Q$. Its character group $X_E = X(\TT_E)$ is the subgroup of $X = X(T)$
consisting of those characters which take the value 1 on $E$.
If $\lambda = \prod_{\sigma \in \Gamma} [\sigma]^{n_\sigma}$ denotes a
character of $\TT$, then $X_E$ is the subgroup of those $\lambda \in X$ for which
$\prod_{\sigma \in \Gamma} [\sigma]^{n_\sigma} = 1$, for all $x \in E$.

\subsubsection*{Exercise}
\begin{enumerate}[label=\textit{\alph*}.]
\item Let $K$ be quadratic over $\Q$, so that $\dim T = 2$. Let $E$ be the
	group of units of $K$. Show that $T$ is of dimension 2 (resp.\ 1) if
	$K$ is imaginary (resp.\ real).
\item Take for $K$ a cubic field with one real place and one complex one, and
	let again $E$ be its group of units (of rank 1). Show that $\dim T = 3$
	and $\dim T_E = 1$.

	(For more examples, see 3.3.)\dpage
\end{enumerate}

\section{Enlarging groups}
...
