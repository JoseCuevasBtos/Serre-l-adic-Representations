\chapter{The groups \texorpdfstring{$S_{\mathfrak{m}}$}{Sm}}
\label{ch:ii}

Throughout this chapter,
\dpage
$K$ denotes an algebraic number field.
We associate to $K$ a projective family $(S_{\mathfrak{m}})$ of commutative
algebraic groups over $\Q$, and we show that each $S_{\mathfrak{m}}$ gives rise
to a strictly compatible system of rational $\ell$-adic representations of $K$.

In the next chapter, we shall see that all ``locally algebraic''
abelian rational representations are of the form described here.

\section{Preliminaries}
\label{sec:II_1}

\subsection{The torus $\TT$}
\label{sec:II_11}
Let $\TT = \WRes_{K/\Q}(\GG_{m, K})$\index{Tz@$\TT = \WRes_{K/\Q}(\GG_{m, K})$}
be the algebraic group over $\Q$, obtained from the multiplicative group
$\GG_m$\index{Gm@$\GG_m$} by restriction of scalars from $K$ to $\Q$, cf.\ 
\citeauthor{43}~\cite{43}, \S 1.3. If $A$ is a commutative $\Q$-algebra, the
points of $\TT$ with values in $A$ form by definition the multiplicative group
$(K \otimes_\Q A)^\times$ of invertible elements of $K \otimes_\Q A$.
In particular, $\TT(\Q) = K^\times$. If $d = [K : \Q]$, the group $\TT$ is a
\strong{torus}\index{Torus} of dimension $d$; this means that the group
$\TT_{/\algcl\Q} = \TT \otimes_\Q \algcl\Q$ obtained from $\TT$ by extending
the scalars from $\Q$ to $\algcl\Q$, is isomorphic
\dpage
to $\GG_{m, \algcl{\Q}} \times \hdots \times \GG_{m, \algcl{\Q}}$ ($d$ times).
More precisely, let $\Gamma$ be the set of embeddings of $K$ into $\algcl{\Q}$;
each $\sigma \in \Gamma$ extends to a homomorphism $K \otimes_{\Q} \algcl{\Q}
\to \algcl{\Q}$, hence defines a morphism $[\sigma]\colon \TT_{/\algcl{\Q}} \to
\GG_{m, \algcl{Q}}$. The collection of all $[\sigma]$'s gives the isomorphism
$T_{/ \algcl{\Q}} \to G_{m/\algcl{\Q}} \times \hdots \times G_{m/\algcl{\Q}}$.
Moreover, the $[\sigma]$'s form a basis of the \emph{character group} $X(T) =
\Hom_{\algcl{\Q}}(\TT_{/\algcl{\Q}}, \GG_{m, \algcl{\Q}})$ of $T$. Note that
the Galois group $\Gal(\algcl{\Q} / \Q)$ acts in a natural way on $X(T)$, viz.
by permuting the $[\sigma]'s$. (For the dictionary between tori and Galois
modules, see for instance T. Ono \cite{21}.)

\subsection{Cutting down $\TT$}
\label{sec:II_12}
Let $E$ be a subgroup of $K = \TT(\Q)$ and let $\overline{E}$ be the Zariski
closure of $E$ in $\TT$. Using the formula $\overline{E} \times \overline{E} =
\overline{E \times E}$, one sees that $E$ is an algebraic subgroup of $\TT$.
Let $\TT_E$ be the quotient group $\TT/E$; then $\TT_E$ is also a torus over
$\Q$. Its character group $X_E = X(\TT_E)$ is the subgroup of $X = X(T)$
consisting of those characters which take the value 1 on $E$.
If $\lambda = \prod_{\sigma \in \Gamma} [\sigma]^{n_\sigma}$ denotes a
character of $\TT$, then $X_E$ is the subgroup of those $\lambda \in X$ for 
which
$\prod_{\sigma \in \Gamma} [\sigma]^{n_\sigma} = 1$, for all $x \in E$.

\subsubsection*{Exercise}
\begin{enumerate}[label=\textit{\alph*}.]
\item Let $K$ be quadratic over $\Q$, so that $\dim T = 2$. Let $E$ be the
	group of units of $K$. Show that $T$ is of dimension 2 (resp.\ 1) if
	$K$ is imaginary (resp.\ real).
\item Take for $K$ a cubic field with one real place and one complex one, and
	let again $E$ be its group of units (of rank 1). Show that $\dim T = 3$
	and $\dim T_E = 1$.

	(For more examples, see \ref{sec:II_33}.)\dpage
\end{enumerate}

\subsection{Enlarging groups}
\label{sec:II_13}
Let $k$ be a field and $A$ a commutative algebraic group over $k$. Let
\begin{equation}\label{eq:II_13}
	\begin{tikzcd}
		0 \arrow[r] & Y_1 \arrow[r] & Y_2 \arrow[r] & Y_3 \arrow[r] & 0
	\end{tikzcd}
\end{equation}
an exact sequence of (abstract) commutative groups, with $Y_3$ \emph{finite}.
Let
\[
\varepsilon : Y_1 \rightarrow A(k)
\]
be a homomorphism of $Y_1$ into the group of $k$-rational points of $A$. We 
intend to construct an algebraic group $B$, together with a morphism of 
algebraic groups $A \to B$ and a homomorphism of $Y_2$ into $B(k)$ such that,
\begin{enumerate}[(a)]
	\item \emph{the diagram}
	\[
		\begin{tikzcd}
			Y_1 \arrow[r] \arrow[d] & A(k) \arrow[d] \\
			Y_2 \arrow[r] & B(k)
		\end{tikzcd}
	\]
	\emph{is commutative},
	\item $B$ \emph{is ``universal'' with respect to} (a).
\end{enumerate}
The universality of $B$ means that, for any algebraic group $B'$ over $k$ and 
morphism $A \to B'$, $Y_2 \to B'(k)$ such that $(a)$ is true (with $B$ replaced 
by $B'$), there exists a unique algebraic morphism $f:B \to B'$ such that the 
given maps $A \to B'$ and $Y_2 \to B(k)$ can
\dpage
be obtained by composing those of $B$ with $f$. (In other words, $B$ is a \emph
{push-out} over $Y_1$ of $A$ and the ``constant'' group scheme defined by 
$Y_2$.)

The uniqueness of $B$ is assured by its universality. Let us prove its 
existence. For each $y \in Y_3$ let $\overline{y}$ be a representative of $y$
in 
$Y_2$. If $y$, $y' \in Y_3$, we have
\[
\overline{y} + \overline{y'} = \overline{y + y'} + c(y, y')
\]
with $c(y, y') \in Y_1$; the cochain $c$ is a $2$-cocycle defining the 
extension~\ref{eq:II_13}. Let $B$ be the disjoint union of copies $A_y$ of $A$, 
indexed by $y \in Y_3$. Define a group law on $B$ via the mappings
\[
\pi_{y, y'} : A_y \times A_{y'} \rightarrow A_{y + y'} \qquad (y,~y' \in Y_3),
\]
given by addition in $A$ followed by translation by $\varepsilon(c(y,y'))$. One 
then checks easily that $B$ has the required universal  property, the maps $A 
\to B$ and $Y_2 \to B(k)$ being defined as follows:

$A \to B$ is the natural map $A \to A_0$ followed by translation by $- c(0,0)$,

$Y_2 \to B(k)$ maps an element $\overline{y} + z$, $y \in Y_3$, $z \in Y_1$ 
onto the image of z in $A_y$.

Note that for any extension \emph{field} $k'$ of $k$ we have an \emph{exact 
sequence}
\[
	\begin{tikzcd}
		0 \arrow[r] & A(k') \arrow[r] & B(k') \arrow[r] & Y_3 \arrow[r] & 0,
	\end{tikzcd}
\]
and a \emph{commutative diagram}
\dpage
\[
	\begin{tikzcd}
		0 \arrow[r] & Y_1 \arrow[r] \arrow[d] & Y_2 \arrow[r] \arrow[d] & Y_3 
		\arrow[r] \arrow[d] & 0 \\
		0 \arrow[r] & A(k') \arrow[r] & B(k') \arrow[r] & Y_3 \arrow[r] & 0
	\end{tikzcd}
\]
The algebraic group $B$ is thus an \emph{extension} of the ``constant'' algebraic group $Y_3$ by $A$.
\begin{obs}
\begin{enumerate}
	\item Let $k'$ be an extension of $k$ and $A' = A \times_k k'$. We may 
	apply the above construction to the $k'$-algebraic group $A'$, with respect 
	to the exact sequence~\ref{eq:II_13} and to the map $Y_1 \to A(k) \to A'(k')
	$. The group $B'$ thus obtained is canonically isomorphic to $B \times_k 
	k'$; this follows, for instance, from the explicit construction of $B$ and 
	$B'$.
	
	\item We will only use the above construction when $\char(k) = 0$ and $A$ 
	is a torus. The enlarged group $B$ is then a ``group of multiplicative 
	type''; this means that, after a suitable finite extension of the ground 
	field, $B$ becomes isomorphic to the product of a torus and a finite 
	abelian group. Such a group is uniquely determined by its character group $X
	(B) = \Hom_{\algcl{k}} (B_{/\algcl{k}}, \GG_{m, \algcl{k}})$, which is a 
	Galois-module of finite type over $\Z$. Here $X(B)$ can be described as the 
	set of pairs $(\phi, \chi)$, where $\phi \colon Y_2 \to
	{\algcl{k}}^\times$ is a homomorphism and $\chi \in Z(A)$ is such that
	$\phi(y_1) = \chi(y_1)$ for all $y_1 \in Y_1$. Note that this gives an
	alternate definition of $B$. 
\end{enumerate}
\end{obs}

\subsubsection*{Exercise}
\begin{enumerate}[a)]
	\item Let $k'$ be a commutative $k$-algebra, with $k' \neq 0$, and $\Spec
	(k')$ connected (i.e.\ $k'$ contains exactly two idempotents: $0$ and $1$). 
	Show the existence of an exact sequence:
	\dpage
	\[\begin{tikzcd}
		0 \rar & A(k') \rar & B(k') \rar & Y_3 \rar & 0.
	\end{tikzcd}\]
	\item What happens when $\Spec(k')$ is not connected?
\end{enumerate}

\section{Construction of \texorpdfstring{$T_{\mathfrak{m}}$}{Tm} and
\texorpdfstring{$S_{\mathfrak{m}}$}{Sm}}
\label{sec:II_2}

\subsection{Idèles and idèles-classes}
\label{sec:II_21}
We defined in Chapter~\ref{ch:i}, \ref{sec:I_21} the set $M_K^0$ of finite
places of the number field $K$. Let now $M_K^\infty$\index{MKinf@$M_K^\infty$}
be the set of equivalence classes of archimedian absolute values of $K$, and
let $M_K$\index{MK@$M_K$} be the union of $M_K$ and $M_K^\infty$. If $v \in
M_K$ then $K_v$ denotes the \emph{completion} of $K$ with respect to $v$. For
$v \in M_K^\infty$ we have $K_v = \R$ or $K_v = \C$, and $K$ is ultrametric if
$v \in M_K^0$. For $v \in M_K^0$, the group of units of $K_v$ is denoted by
$U_v$.\index{Uv@$U_v$} The \strong{idèle group}\index{Idèle} $I$\index{I@$I$}
of $K$ is the subgroup of
\[
	\prod_{v \in M_K} K_v^\times,
\]
consisting of the families $(a_v)$ with $a_v \in U_v$, for almost all $v \in
M_K^0$; it is given a topology by decreeing that the subgroup (with the product
topology)
\[
	\prod_{v \in M_K^\infty} K_v^\times \times \prod_{v \in M_K^0} U_v
\]
be open. We embed $K^\times$ into $I$ by sending $a \in K^\times$ onto the
idèle $(a_v)$, where $a_v = a$ for all $v$. The topology induced on $K$ is the
discrete topology. The quotient group $C_K = I/K^\times$\index{class
group@$C_K$} is called the \strong{idèle class group}\index{Idèle class} of
$K$.
(For all this, see \citeauthor{6}~\cite{6}, \citeauthor{13}~\cite{13} or
\citeauthor{44}~\cite{44}.)

Let $S$ be a finite subset of $M_K^0$.
\dpage
Then by a \strong{modulus of support $S$} we mean a family $\mathfrak{m} =
(m_v)_{v\in S}$ where the $m_v$ are integers $\ge 1$.  If $v \in M_K$ and
$\mathfrak{m}$ is a modulus of support $S$, we let $U_{v, \mathfrak{m}}$%
\index{Uvm@$U_{v, \mathfrak{m}}$} denote the connected component of
$K_v^\times$ if $v \in M_K^\infty$, the subgroup of $U_v$ consisting of those
$u \in U_v$ for which $v(1-u) \ge m_v$ if $v \in S$, and $U_v$ if $v \in M_K^0
\setminus S$. The group $U_{\mathfrak{m}} = \prod_{v} U_{v, \mathfrak{m}}$ is
an \emph{open subgroup} of $I$.  If $E$ is the group of units of $K$, let
$E_{\mathfrak{m}} = E \cap U_{\mathfrak{m}}$.\index{Em@$E_{\mathfrak{m}}$} The
subgroup $E_{\mathfrak{m}}$ is of finite index in $E$.
(Conversely, by a theorem of Chevalley (\cite{8}, see also \cite{24},
n\textsuperscript{o} 3.5) every subgroup of finite index in $E$ contains an
$E_{\mathfrak{m}}$ for a suitable modulus $\mathfrak{m}$.)

Let $I_{\mathfrak{m}}$\index{Im@$I_{\mathfrak{m}}$} be the quotient
$I/U_{\mathfrak{m}}$ and $C_{\mathfrak{m}}$\index{Cm@$C_{\mathfrak{m}}$} the
quotient $I/K^\times U_{\mathfrak{m}} = C/$(Image of $U_{\mathfrak{m}}$ in
$C_{\mathfrak{m}}$). One then has the exact sequence:
\[\begin{tikzcd}[sep=large]
	1 \rar & K^\times/E_{\mathfrak{m}} \rar & I_{\mathfrak{m}} \rar &
	C_{\mathfrak{m}} \rar & 1
\end{tikzcd}\]
\emph{The group $C_{\mathfrak{m}}$ is finite}; in fact, the image of
$U_{\mathfrak{m}}$ in $C$ is open, hence contains the connected component
$D$\index{D@$D$} of $C$, and the group $C/D$ is known to be compact (see
\cite{13}, \cite{44}).  Moreover, any open subgroup of $I$ contains one of the
$U_{\mathfrak{m}}$'s, hence \emph{$C/D$ is the projective limit} of the
$C_{\mathfrak{m}}$'s. Class field theory (cf.\ for instance
\citeauthor{6}~\cite{6}), gives \emph{an isomorphism of $C/D = \invlim
C_{\mathfrak{m}}$ onto the Galois group $\abcl G$} of the maximal abelian
extension of $K$.

\begin{obs}
A more classical definition of $C_{\mathfrak{m}}$ is as follows. Let $\Id_S$ be
the group of fractional ideals of $K$ prime to $S$, and $P$ the subgroup of
principal ideals $(\gamma)$, where $\gamma$ is totally positive and
\dpage
$\gamma \equiv 1 \mod{\mathfrak{m}}$ (i.e.\ $\gamma$ belongs to $U_{v,
\mathfrak{m}}$ for all $v \in S$ and $v \in M_K^\infty$). Let
$\Cl_{\mathfrak{m}} = \Id_S/P_{S, \mathfrak{m}}$. We have the exact sequence:
\[\begin{tikzcd}[sep=large]
	1 \rar & P_{S, \mathfrak{m}} \rar & \Id_S \rar & \Cl_{\mathfrak{m}}
	\rar & 1.
\end{tikzcd}\]
For each \vec $a = \prod_{v\notin S} v^{a_v} \in \Id_S$, choose an idèle
$\alpha = (\alpha_v)$, with $\alpha_v \in U_{v, \mathfrak{m}}$ if $v \in S$ or
$v \in M_K^\infty$, and $v(\alpha_v) = a_v$ if $v \in M_K^\infty \setminus S$.
The image of $\alpha$ in $I_{\mathfrak{m}} = I/U_{\mathfrak{m}}$ depends only
on $\vec a$. We then get a homomorphism $g \colon \Id_S \to I_{\mathfrak{m}}$.
One checks readily that $g$ extends to a commutative diagram
% https://q.uiver.app/#q=WzAsMTAsWzAsMCwiXFxidWxsZXQiXSxbMSwwLCJcXGJ1bGxldCJdLFsxLDEsIlxcYnVsbGV0Il0sWzAsMSwiXFxidWxsZXQiXSxbMiwxLCJcXGJ1bGxldCJdLFsyLDAsIlxcYnVsbGV0Il0sWzMsMCwiXFxidWxsZXQiXSxbMywxLCJcXGJ1bGxldCJdLFs0LDEsIlxcYnVsbGV0Il0sWzQsMCwiXFxidWxsZXQiXSxbMCwxXSxbMSwyXSxbMywyXSxbMiw0XSxbMSw1XSxbNSw2XSxbNSw0LCJnIl0sWzYsNywiZiJdLFs0LDddLFs3LDhdLFs2LDldXQ==
\[\begin{tikzcd}[row sep=large]
	1 & P_{S, \mathfrak{m}} & \Id_S & \Cl_{\mathfrak{m}} & 1 \\
	1 & K^\times/E_{\mathfrak{m}} & I_{\mathfrak{m}} & C_{\mathfrak{m}} & 1
	\arrow[from=1-1, to=1-2]
	\arrow[from=1-2, to=1-3]
	\arrow[from=1-2, to=2-2]
	\arrow[from=1-3, to=1-4]
	\arrow["g", from=1-3, to=2-3]
	\arrow[from=1-4, to=1-5]
	\arrow["f", from=1-4, to=2-4]
	\arrow[from=2-1, to=2-2]
	\arrow[from=2-2, to=2-3]
	\arrow[from=2-3, to=2-4]
	\arrow[from=2-4, to=2-5]
\end{tikzcd}\]
and that $f\colon \Cl_{\mathfrak{m}} \to C_{\mathfrak{m}}$ \emph{is an
isomorphism:} hence $C$ can be identified with the ideal class group mod
$\mathfrak{m}$ (and this shows again that it is finite).
\end{obs}

\subsection{The groups \texorpdfstring{$T_{\mathfrak{m}}$}{Tm} and
\texorpdfstring{$S_{\mathfrak{m}}$}{Sm}}
\label{sec:II_22}

We are now in a position to apply the group construction of~\ref{sec:II_13}.
We take for exact sequence~\ref{eq:II_13} the sequence 
\[
	\begin{tikzcd}
		1 \arrow[r] & K^{\times} / E_\mathfrak{m} \arrow[r] & I_\mathfrak{m} 
		\arrow[r] & C_\mathfrak{m} \arrow[r] & I
	\end{tikzcd}
\]
and for $A$ the algebraic group $T_\mathfrak{m} = \TT / \algcl{E_\mathfrak{m}}
$, where $E_\mathfrak{m}$ is as before, $\TT$ is the torus $\WRes_{K/\Q}(
\GG_{m, K})$ defined in~\ref{sec:II_11}, and $\algcl{E_\mathfrak{m}}$ is the 
\dpage
Zariski closure of $E_\mathfrak{m}$ in $\TT$, cf.\ ~\ref{sec:II_12}.

The construction of~\ref{sec:II_13} now yields a \emph{$\Q$-algebraic group} 
$S_\mathfrak{m}$ with an algebraic morphism $T_\mathfrak{m} \to S_\mathfrak{m}$ 
and a group homomorphism $\varepsilon\colonI_\mathfrak{m} \to S_\mathfrak{m} (\Q)$.
The sequence 
\[
\begin{tikzcd}
	1 \arrow[r] & T_\mathfrak{m} \arrow[r] & S_\mathfrak{m} \arrow[r] &
	C_\mathfrak{m} \arrow[r] & 1
\end{tikzcd}
\]
is exact ($C_\mathfrak{m}$ being identified with the corresponding constant 
algebraic group) and the diagram 
\begin{equation}\tag{$**$}\label{eq:II_22}
	\begin{tikzcd}
		1 \arrow[r] & K^\times / E_\mathfrak{m} \arrow[r] \arrow[d] & 
		I_\mathfrak{m} \arrow[r] \arrow[d, "\varepsilon"] & C_\mathfrak{m} 
		\arrow[r] \arrow[d, "\id"] & 1 \\
		1 \arrow[r] & T_\mathfrak{m}(\Q) \arrow[r] & S_\mathfrak{m}
		(\Q) \arrow[r] & C_\mathfrak{m} \arrow[r] & 1
	\end{tikzcd}
\end{equation}
is commutative.

\begin{obs}
Let $\mathfrak{m}'$ be another modulus; assume $\mathfrak{m}' \geq \mathfrak{m}
$, i.e.\ $\Supp(\mathfrak{m}') \supset \Supp(\mathfrak{m})$ and 
$\mathfrak{m}'_v \geq \mathfrak{m}_v$ if $v \in \Supp(\mathfrak{m})$. From the  
inclusion $U_{\mathfrak{m}'} \subset U_\mathfrak{m}$ one deduces maps $T_{
\mathfrak{m}'} \to T_\mathfrak{m}$ and $I_{\mathfrak{m}'} \to I_\mathfrak{m}$,
whence a morphism $S_{\mathfrak{m}'} \to S_\mathfrak{m}$. Hence the 
$S_\mathfrak{m}$'s form a projective system; their limit is a proalgebraic 
group over $\Q$, extension of the profinite group $C/D = \invlim
C_{\mathfrak{m}}$ by a torus.
\end{obs}

\subsubsection*{Exercises}
\begin{enumerate}[1)]
\item Let $\algcl{E_\mathfrak{m}}(\Q)$ be the Zariski-closure of
	$E_\mathfrak{m}$ in $K^\times = \T(\Q)$. Show that the kernel of
	$\varepsilon_\mathfrak{m} \colon I/U_\mathfrak{m} \to
	S_\mathfrak{m}(\Q)$ is the image of $\algcl{E_\mathfrak{m}}(\Q) \to
	I/U_\mathfrak{m}$.
\item Let $H_{\mathfrak{m}'/\mathfrak{m}}$ be the kernel of
	\dpage
	$S_{\mathfrak{m}'} \to S_\mathfrak{m}$, where $\mathfrak{m}'_v \geq
	\mathfrak{m}_v$.
	\begin{enumerate}[a)]
	\item Show that $H_{\mathfrak{m}'/\mathfrak{m}}$ is a finite subgroup
		of $S_{\mathfrak{m}'}(\Q)$ and that it is contained  in the
		image of $\varepsilon_{\mathfrak{m}'}$.
	\item Construct an exact sequence (cf.\ Exer.~1)
		\[\begin{tikzcd}
		1 \arrow[r] & \left({E_\mathfrak{m} \cap
		\algcl{E_{\mathfrak{m}'}}(\Q)}\right)/ E_{\mathfrak{m}'}
				\arrow[r] & U_\mathfrak{m} / U_{\mathfrak{m}'}
				\arrow[r] & H_{\mathfrak{m}'/\mathfrak{m}}
				\arrow[r] & 1.
		\end{tikzcd}\]
	\end{enumerate}	
\end{enumerate}

\subsection{The canonical \texorpdfstring{$\ell$}{ℓ}-adic representation with
values in \texorpdfstring{$S_{\mathfrak{m}}$}{Sm}}\label{sec:II_23}
Let $\mathfrak{m}$ be a modulus, and let $\ell$ be a prime number. Let
$\varepsilon\colon I \to I_{\mathfrak{m}} \to S_{\mathfrak{m}}(\Q)$ be the
homomorphism defined in \ref{sec:II_22}. Let $\pi\colon \TT \to S_{\mathfrak{m}}$
be the algebraic morphism $\TT \to T_{\mathfrak{m}} \to S_{\mathfrak{m}}$; by
taking points with values in $\Q_\ell$, $\pi$ defines a homomorphism
\[
	\pi_\ell \colon \TT(\Q_\ell) \longrightarrow S_{\mathfrak{m}}(\Q_\ell)
\]
Since $K \otimes \Q_\ell = \prod_{v\mid\ell} K_v$, the group $\TT(\Q_\ell)$ can
be identified with $K_\ell^\times = \prod_{v\mid\ell} K_v^\times$, and is
therefore a direct factor of the idele group $I$.  Let $\pr_\ell$ denote the
projection of $I$ onto this factor. The map
\[
	\alpha_\ell = \pi_\ell \circ \pr_\ell \colon I \longrightarrow \TT(\Q_\ell)
	\longrightarrow S_{\mathfrak{m}}(\Q_\ell)
\]
is a continuous homomorphism.
\begin{lem}
	$\alpha_\ell$ and $\varepsilon$ coincide on $K^\times$.
\end{lem}
This is trivial from the commutativity of the diagram (**) of \ref{sec:II_22}.

Now, let \index{el@$\varepsilon_\ell$}$\varepsilon_\ell \colon I \to
S_{\mathfrak{m}}(\Q_\ell)$ be defined by
\dpage
\begin{align}
	\varepsilon_\ell(\vec a) &= \varepsilon(\vec a) \alpha_\ell(\vec a^{-1})
	\tag{$***$}\label{eqn:II_23_star} \\
	\text{i.e.} \quad \varepsilon_\ell &= \varepsilon \cdot \alpha_\ell^{-1}.
	\notag
\end{align}
(If $\vec a \in I$, write $a_\ell$ the $\ell$-component of $\vec a$. Then
\[
	\varepsilon_\ell(\vec a) = \varepsilon(\vec a) \pi_\ell(a_\ell^{-1}).)
\]
By the lemma, $\varepsilon_\ell$ is trivial on $K$ and, hence, defines a map
$C \to S_{\mathfrak{m}}(\Q_\ell)$; since $S_{\mathfrak{m}}(\Q_\ell)$ is
totally disconnected (it is an $\ell$-adic Lie group), the latter homomorphism is
trivial on the connected component $D$ of $C$. We have already recalled that $C/D$
may be identified with the Galois group $\abcl G$ of the maximal abelian
extension of $K$. So we end up with a homomorphism $\varepsilon_\ell \colon
\abcl G \to S_{\mathfrak{m}}(\Q_\ell)$, i.e.\ with an \emph{$\ell$-adic
representation of $K$ with values in $S_{\mathfrak{m}}$} (cf.\ 
Chap.~\ref{ch:i}, \ref{sec:I_23}).

This representation is rational in the sense of Chapter~\ref{ch:i},
\ref{sec:I_23}.  More precisely, let $v \notin \Supp(\mathfrak{m})$, and let
$f_v \in I$ be an idèle which is a uniformizing parameter at $v$, and which is
equal to 1 everywhere else; let $F_v = \varepsilon(f_v)$ be the image of $f_v$
in $S_{\mathfrak{m}}(\Q_\ell)$. With these notations we have:

\begin{prop}
\begin{enumerate}[a)]
\item The representation $\varepsilon_\ell, \colon \abcl G \to
	S_{\mathfrak{m}}(\Q_\ell)$ is a rational representation with values in
	$S_{\mathfrak{m}}$.
\item $\varepsilon_\ell$ is unramified outside $\Supp(\mathfrak{m}) \cup
	S_\ell$, where $S_\ell = \{ v \mid p_v = \ell \}$.
\item If $v \notin \Supp(\mathfrak{m}) \cup S_\ell$,
	\dpage
	then the Frobenius element $F_{v, \varepsilon_\ell}$ (cf.\ 
	Chap.~\ref{ch:i}, \ref{sec:I_23}) is equal to $F_v \in
	S_{\mathfrak{m}}(\Q_\ell)$.
\end{enumerate}
\end{prop}
\begin{proof}
It is known that the class field isomorphism 
\begin{tikzcd}[cramped, sep=small]
	C/D \rar["\sim"] & \abcl G
\end{tikzcd}
maps $K_v^\times$ (resp.\ $U_v$) onto a dense subgroup of the decomposition
group of $v$ in $\abcl G$ (resp.\ onto the inertia group of $v$ in $\abcl G$),
and that a uniformizing element $f_v$ of $K_v^\times$ is mapped onto the
Frobenius class of $v$.

If $v \notin \Supp(\mathfrak{m})$ and $a \in U_v$, then $\varepsilon(a) = 1$;
if moreover $p_v \ne \ell$, $\alpha_\ell(a) = 1$, hence $\varepsilon_\ell(a) =
1$ and $\varepsilon_\ell$ is unramified at $v$; this proves b). For such a $v$,
we have $\varepsilon_\ell(f_v) = \varepsilon(f_v) = F_v$; hence
c), and a) follows from c).
\end{proof}

\begin{corp}
The representations $\varepsilon$ form a system of strictly compatible
$\ell$-adic representations with values in $S_{\mathfrak{m}}$.
\end{corp}
We also see that the exceptional set of this system is contained
in $\Supp(\mathfrak{m})$; for an example where it is different from $\Supp(\mathfrak{m})$,
see Exercise~\ref{exr:II_23_2}.

\begin{obs}
By construction, $\varepsilon_\ell\colon I \to S_{\mathfrak{m}}(\Q_\ell)$ is
given by $x \mapsto \pi_\ell(x^{-1})$ on the open subgroup $U_{\ell,
\mathfrak{m}} = \prod_{v\mid\ell} U_{v, \mathfrak{m}}$ of $K_\ell^\times$.
Hence, $\Img(\varepsilon_\ell)$ contains $\pi_\ell(U_{\ell, \mathfrak{m}})
\subset T_{\mathfrak{m}}(\Q_\ell) \subset S_{\mathfrak{m}}(\Q_\ell)$, and is an
\emph{open subgroup} of $S_{\mathfrak{m}}(\Q_\ell)$. This open subgroup maps
onto $C_{\mathfrak{m}}$, as remarked above. These properties imply, in
particular, that $\Img(\varepsilon_\ell)$ is \emph{Zariski-dense} in
S_{\mathfrak{m}}.
\end{obs}

\subsubsection*{Exercises}
\dpage
\begin{enumerate}[label=(\arabic*), ref=\arabic*]
\item\label{exr:II_23_1}
	Let $K = \Q$, $\Supp(\mathfrak{m}) = \emptyset$.
\begin{enumerate}[a)]
\item Show that $E_{\mathfrak{m}} = \{ 1 \}$, $C_{\mathfrak{m}} = \{ 1 \}$,
	hence $T_{\mathfrak{m}} = S_{\mathfrak{m}} = \GG_m$ and
	$S_{\mathfrak{m}}(\Q) = \Q^\times$, $S_{\mathfrak{m}}(\Q_\ell) =
	\Q_\ell^\times$.
\item Show that $I$ is the direct product of its subgroups $I_{\mathfrak{m}}$
	and $\Q^\times$; hence any $\vec a \in I$ may be written as
	\[
		\vec a = u\cdot \gamma, \qquad u \in U_{\mathfrak{m}},
		\; \gamma \in \Q^\times.
	\]
	Show that, if $\vec a = (a_p)$, one has
	\[
		\varepsilon(\vec a) = \gamma = \sgn(a_\infty) \prod_{p}
		p^{v_p(a_p)}.
	\]
\item Show that
	\begin{align*}
		\rho_\ell(\vec a) &= \gamma \cdot a_\ell^{-1}, \\
		\intertext{and}
		F_p &= p.
	\end{align*}
\item Show that $\rho_\ell$ coincides with the character $\chi_\ell$ of
	Chap.~\ref{ch:i}, \ref{sec:I_12}.
\end{enumerate}
\item\label{exr:II_23_2}
	Let $K = \Q$, $\Supp(\mathfrak{m}) = \{ 2 \}$ and $m_2 = 1$. Show that
	the groups $E_{\mathfrak{m}}$, $C_{\mathfrak{m}}$, $T_{\mathfrak{m}}$,
	$S_{\mathfrak{m}}$ coincide with those of Exercise~\ref{exr:II_23_1},
	hence that the exceptional set of the corresponding system is empty.
\end{enumerate}

\subsection{Linear representations of \texorpdfstring{$S_{\mathfrak{m}}$}{Sm}}
\label{sec:II_24}
We recall first some well known facts on representations.

\begin{enumerate}[a)]
\item Let $k$ be a field of characteristic 0; let $H$ be an affine
	\dpage
	commutative algebraic group over $k$. Let $X(H) = \Hom_{\algcl
	k}(H_{/\algcl k}, \GG_{m, \algcl k})$ be the group of characters of $H$
	(of degree 1). Here we write the characters of $X(H)$ multiplicatively.
	The group $G = \Gal(\algcl k/k)$ acts on $X(H)$.

	Let $\Lambda$ be the affine algebra of $H$, and let $\algcl\Lambda =
	\Lambda \otimes_k \algcl k$ be the one of $H_{/\algcl k}$. Every
	element $\chi \in X(H)$ can be identified with an invertible element of
	$\algcl\Lambda$. Hence, by linearity, a homomorphism
	\[
		\alpha \colon \algcl k[X(H)] \longrightarrow \algcl \Lambda
	\]
	where $\algcl k[X(H)]$ is the group algebra of $X(H)$ over $\algcl k$.
	This is a $G$-homomorphism if the action of $G$ is defined by
	\[
		s\left( \sum_{\chi} a_\chi \chi \right) = \sum s(a_\chi) s(\chi)
	\]
	for $a_\chi \in \algcl k$ and $\chi \in X(H)$. It is well-known (linear
	independence of characters) that $\alpha$ is injective.  It is
	bijective if and only if $H$ is a group of multiplicative type (cf.\ 
	\ref{sec:II_13}, remark 2). Hence we may identify $\algcl k[X(H)]$ with
	a subalgebra of $\Lambda$.

\item Let $V$ be a finite-dimensional $k$-vector space and let
	\[
		\phi \colon H \longrightarrow \GL_V
	\]
	be a \emph{linear representation} of $H$ into $V$. Assume $\phi$ is
	\emph{semi-simple} (this is always the case if $H$ is of multiplicative
	type). We associate to $\phi$ its \strong{trace}\index{Trace}
	\[
		\theta_\phi = \sum_{\chi} n_\chi(\phi) \, \chi
	\]
	in $\Z[X(H)]$, where $n_\chi(\phi)$ is the multiplicity of $\chi$ in
	the decomposition of $\chi$ over $\algcl k$.

	We have $\theta_\phi(h) = \Tr(\phi(h))$ for any point $h$ of $H$
	\dpage
	(with value in any commutative $k$-algebra). Let $\Rep_k(H)$ be the set
	of isomorphism classes of linear semi-simple representations of $H$. If
	$k_1$ is an extension of $k$, then scalar extension from $k$ to $k_1$
	defines a map $\Rep_k(H) \to \Rep_{k_1}(H_{/k_1})$ which is easily seen
	to be \emph{injective}. We say that an element of\break
	\emph{$\Rep_{k_1}(H_{/k_1})$ can be defined over $k$}, if it is in the
	image of this map.
\end{enumerate}
\begin{prop}

The map $\phi \mapsto \theta_\phi$ defines a bijection between
$\Rep_k(H)$ and the set of elements $\theta = \sum n_\chi \, \chi$ of
$\Z[X(H)]$ which satisfy:
\begin{enumerate}[(a)]
	\item $\theta$ is invariant by $G$ (i.e.\ $n_\chi =
		n_{s(\chi)}$ for all $s \in G$, $\chi \in X(H)$).
	\item $n_\chi \ge 0$ for every $\chi \in X(H)$.
\end{enumerate}
\end{prop}
\begin{proof}
	The injectivity of the map $\phi \mapsto \theta_\phi$ is
	well-known (and does not depend on the commutativity of $H$). To prove
	surjectivity, consider first the case where $\theta$ has the form
	$\theta = \sum_{i} \chi^{(i)}$ where $\chi^{(i)}$ is a full set of
	different conjugates of a character $\chi \in X(H)$. If $G(\chi)$ is
	the subgroup of $G$ fixing $\chi$, then
	\begin{equation}
		\theta = \sum_{s \in G/G(\chi)} s(\chi).
		\tag{$*$}
		\label{eqn:II_24star}
	\end{equation}
	The fixed field $k_\chi$ of $G(\chi)$ in $k$ is the smallest subfield
	of $k$ such that $\chi \in \Lambda \otimes k_\chi$. Consider $\chi$ as
	a representation of degree 1 of $H_{/k_\chi}$. One gets, by restriction
	of scalars to $k$, a representation
	\dpage
	$\phi$ of $H$ of degree $[k_\chi : k]$. One sees easily that the
	trace $\theta_\phi$ of $\phi$ is equal to $\theta$. The
	surjectivity of $\phi \mapsto \theta_\phi$ now follows from the
	fact that any $\theta$ satisfying (a) and (b) is a sum of elements of
	the form \eqref{eqn:II_24star} above.
\end{proof}

\begin{corp}
In order that $\phi_1 \in \Rep_{k_1}(H_{/k_1})$ can be defined over $k$, it
is necessary and sufficient that $\theta_{\phi_1} \in \Lambda \otimes_k k_1$
belongs to $k_1$.
\end{corp}

\begin{enumerate}[resume*]
	\item We return now to the groups $S_{\mathfrak{m}}$:
\end{enumerate}
\begin{prop}
Let $k_1$ be an extension of $k$ and let $\phi \in
\Rep_{k_1}(S_{\mathfrak{m} /k_1})$. The following properties are equivalent:
\begin{enumerate}[(i)]
\item $\phi$ can be defined over $k$,
\item for every $v \notin \Supp(\mathfrak{m})$, the coefficients of the
	characteristic polynomial $\phi(F_v)$ belong to $k$,
\item there exists a set $M$ of places of $k$ of density 1 (cf.\ 
	Chapter~\ref{ch:i}, \ref{sec:I_22}) such that $\Tr(\phi(F_v)) \in k$
	for all $v \in M$.
\end{enumerate}
\end{prop}
\begin{proof}
The implications (i) $\implies$ (ii) $\implies$ (iii) are trivial.
To prove (iii) $\implies$ (i) we need the following lemma.
\end{proof}

\begin{lem}
	The set of Frobeniuses $F_v$, $v \in M$, is dense in $S$ for
	the Zariski topology.
\end{lem}
\begin{proof}
Let $X$ be the set of all $F_v$'s, $v \in M$, and let $\ell$ be a prime number.
Let $\overline{X} \subseteq S_{\mathfrak{m}}$ (resp.\ $\overline{X}_\ell
\subseteq S_{\mathfrak{m}}(\Q_\ell)$) the closure of $X$ in the Zariski
topology (resp.\ $\ell$-adic topology). It is clear that
\dpage
$\overline{X} \subseteq \overline{X}(\Q_\ell)$. On the other hand, \v
Cebotarev's theorem (cf.\ Chapter~\ref{ch:i}, \ref{sec:I_22}) implies that
$\overline{X} = \Img(\varepsilon_\ell)$ (cf.\ \ref{sec:II_23}). The set
$\Img(\varepsilon_\ell)$, however, is Zariski dense in $S_{\mathfrak{m}}$ (cf.\ 
Remark in \ref{sec:II_23}). Hence $\overline{X} = S_{\mathfrak{m}}$, which
proves the lemma.
\end{proof}

Let us now prove that (iii) $\implies$ (i). Let $\theta_\phi$ be the trace
of $\theta$ in $\Lambda \otimes_k k_1$, where $\Lambda$ is the affine algebra
of $H = S_{\mathfrak{m} /k}$. Let $\{ \ell_\alpha \}$ be a basis of the
$k$-vector space $k_1$, with $\ell_{\alpha_0} = 1$ for some index $\alpha_0$.
We have $\theta_\phi = \sum_{\alpha} \lambda_\alpha \otimes \ell_\alpha$
($\lambda_\alpha \in \Lambda$); hence $\Tr( \phi(h) ) = \theta_\phi(h) =
\sum_{\alpha} \lambda_\alpha(h) \ell_\alpha$ for all $h \in H(k_1)$. Take $h =
F_v$, with $v \in M$, Since $F_v$ belongs to $H(k)$ we have
$\lambda_\alpha(F_v) \in k$ for all $\alpha$; since $\Tr(\phi(F_v)) \in k$,
we get $\lambda_\alpha(F_v) = 0$ for all $\alpha \ne \alpha_0$. By the lemma,
the $F_v$'s, $v \in M$, are Zariski-dense in $H$; hence $\lambda_\alpha = 0$
for $\alpha \ne \alpha_0$ and $\theta_\phi = \lambda_{\alpha_0}$ belongs to
$\Lambda$ and (i) follows from the corollary to Proposition 1.
\hfill
\qedsymbol

\subsubsection*{Exercise}
Show that the characters of $S_{\mathfrak{m}}$ correspond in a one-one way to
the homomorphisms $\chi \colon I \to \algcl{\Q}^\times$ having the following
two properties:
\begin{enumerate}[(a)]
\item $\chi(x) = 1$ if $x \in U_{\mathfrak{m}}$.
\item For each embedding $\sigma$ of $K$ into $\algcl\Q$, there exists an
	integral number $n(\sigma)$ such that
	\[
		\chi(x) = \prod_{\sigma \in \Gamma} \sigma(x)^{n(\sigma)}
	\]
	for all $x \in K^\times$.
\end{enumerate}

\subsection{\texorpdfstring{$\ell$}{ℓ}-adic representations associated to a
linear representation of \texorpdfstring{$S_{\mathfrak{m}}$}{Sm}}
\label{sec:II_25}
\dpage

\subsubsection{\texorpdfstring{$\ell$}{ℓ}-adic case}
% \label{sssec:II_25_1}

Let $V_\ell$ be a finite-dimensional $\Q_\ell$-vector space and
\[
	\phi \colon S_{\mathfrak{m} / \Q_\ell} \rightarrow \GL_{V_\ell}
\]
a linear representation of $S_{\mathfrak{m} / \Q_\ell}$ in $V_\ell$. This 
defines a homomorphism
\[
	\phi \colon S_\mathfrak{m}(\Q_\ell) \rightarrow \GL_{V_\ell}(\Q_\ell) = 
	\Aut(V_\ell)
\]
which is continuous for the $\ell$-adic topologies of those groups.

By composition with the map $\varepsilon_\ell \colon \abcl{G} \to
S_{\mathfrak{m}} (\Q_\ell)$ defined in~\ref{sec:II_23}, we get a map
\todo[pinktask]{Lo dejo cómo $G$ o como galois de algo}
\[
	\phi_\ell = \phi \circ \varepsilon \colon \abcl{G} \to \Aut(V_\ell),
\]
i.e.\ an \emph{abelian $\ell$-adic representation of} $K$ in $V_\ell$.

\begin{prop}
\begin{enumerate}[a)]
	\item The representation $\phi_\ell$ is semi-simple.
	\item Let $v \in M_K^0$, with $v \not\in \Supp(\mathfrak{m})$ and $p_v \neq 
	\ell$.
\todo[pinktask]{no pertence?}

Then $\phi_\ell$ is unramified at $v$; the corresponding Frobenius element 
$F_{v, \phi_v} \in \Aut(V_\ell)$ is equal to $\phi(F_v)$, where $F_v$ denotes 
the element of $S_\mathfrak{m}(\Q)$ defined in~\ref{sec:II_23}.
\dpage
	\item The representation $\phi_\ell$ is rational (Chap.~\ref{ch:i},
		\ref{sec:II_23}) if and only if $\phi$ can be defined over $\Q$
		(cf.\ \ref{sec:II_24}).
\end{enumerate}
\end{prop}

Since $S_\mathfrak{m}$ is a group of multiplicative type, all its 
representations can be brought to diagonal form on a suitable extension of the 
ground field; hence a). Assertion b) follows from~\ref{sec:II_23}, and 
assertion c) follows from prop.~2 of~\ref{sec:II_24}.

\begin{obs}
Let us identify $\phi_\ell$ with the corresponding homomorphism of the idèle 
group $I$ into $\Aut(V_\ell)$. Then
\begin{enumerate}
	\item[d)] $\Ker(\phi_\ell)$ contains $U_{v , \mathfrak{m}}$ if $v \not\in 
	\Supp(\mathfrak{m})$, $p_v \neq \ell$.
	\item[e)]  Let $\phi_T \colon T_{/\Q_\ell} \to \GL_{V_\ell}$ be defined by 
	composing $T_{/\Q_\ell} \to S_{\mathfrak{m}/\Q_\ell}$ with $\phi$. If $x$ 
	belongs to the open subgroup $U_{\ell , \mathfrak{m}} = \prod_{v \mid \ell} 
	U_{v , \mathfrak{m}}$ of $T(\Q_\ell)$, one has
\[
	\phi_\ell (x) = \phi_T(x^{-1}).
\]
These properties follow readily from those of $\varepsilon_\ell$.
\todo{Es T de toro?}
\end{enumerate}
\end{obs}

\subsubsection{The rational case}

Let now $V_0$ be a finite dimensional vector space over $\Q$ and 
$\phi_0 \colon S_\mathfrak{m} \to \GL_{V_0}$ a linear representation of 
$S_\mathfrak{m}$. For each prime number $\ell$ we may apply the preceding 
construction to the representation $\phi_{0/\ell} \colon S_{\mathfrak{m}/
\Q_{\ell}} \to \GL(V_\ell)$, where $V_\ell = V_0 \otimes \Q_\ell$; 
\dpage
we then get an $\ell$-adic representation $\phi_\ell \colon \abcl{G} \to \Aut
(V_\ell)$.

\begin{thm}
\begin{enumerate}[1)]
	\item The $\phi_\ell$ form a strictly compatible system of rational abelian 
	semi-simple representations. Its exceptional set is contained in $\Supp
	(\mathfrak{m})$.
	\item For each $v \not\in \Supp(\mathfrak{m})$ the Frobenius element of $v$ 
	with respect to the system $(\phi_\ell)$ is the element $\phi_0(F_v)$ 
	of $\Aut(V_0)$.
	\item There exist infinitely many primes $\ell$ such that $\phi_\ell$ is 
	diagonalizable over $\Q_\ell$.
\end{enumerate}
\end{thm}

The first two assertions follow directly from the proposition above. To prove 
the third one, note first that there exists a finite extension $E$ of $\Q$ over 
which $\phi_0$ becomes diagonalizable. If $\ell$ is a prime number which 
splits completely in $E$, one can embed $E$ into $\Q_\ell$ and this shows that 
$\phi_\ell$ is diagonalizable. Assertion 3) now follows from the well-known 
fact that there exist infinitely many such $\ell$ (this is, for instance, a 
consequence of \v Cebotarev's theorem, cf.\ Chap.~\ref{ch:i}, \ref{sec:II_22}).

\begin{obs}
The Frobenius elements $\phi_0(F_v) \in \Aut(V_0)$ can also be defined 
using the homomorphism
\[\begin{tikzcd}
	\phi_0 \circ \varepsilon \colon I \rar & S_\mathfrak{m}(\Q) \rar &
	\Aut(V_0).
\end{tikzcd}\]

Note that their eigenvalues generate \emph{a finite extension} of $\Q$; indeed 
they are contained in any field over which $\phi_0$ can be brought in 
diagonal form.
\end{obs}

\subsubsection*{Exercises}
{}\dpage
\begin{enumerate}[1)]
	\item Let $\phi_0 \colon S_\mathfrak{m} \to \GL_{V_0}$ be a linear
representation of $S_\mathfrak{m}$, and let $\ell$ be a prime number.
\begin{enumerate}[a)]
\item Show that the Zariski closure of $\Img(\phi_\ell)$ is the algebraic group
	$\phi_0(S_\mathfrak{m})$. (Use the fact that ($\Img(\varepsilon_\ell)$)
	is Zariski dense in $S_\mathfrak{m}$, cf.~\ref{sec:II_23}.)
\item Let $\mathfrak{s}_\mathfrak{m}$ be the Lie algebra of $S_\mathfrak{m}$
	and $\phi_0(\mathfrak{s}_\mathfrak{m})$ be its image by $\phi_0$, i.e.\
	the Lie algebra of $\phi_0(S_\mathfrak{m})$. Show that the Lie algebra
	of the $\ell$-adic Lie group $\Img(\phi_\ell)$ is
	$\phi_0(\mathfrak{s}_\mathfrak{m}) \otimes \Q_\ell$. (Use the fact that
	$\Img(\varepsilon_\ell)$ is open in $S_\mathfrak{m}(\Q_\ell)$, cf.\
	\ref{sec:II_23}.)
\end{enumerate}
	\item\label{exr:II_25_2}
	\begin{enumerate}[a)]
		\item Show that there exists a unique one-dimensional representation 
		\[
			\numnorm \colon S_{\mathfrak{m}} \rightarrow \GG_m
		\]
		such that $\numnorm(F_v) = \numnorm v \in \Q^{\times}$ for all 
		$v \not\in \Supp(\mathfrak{m})$.
		\item Show that the morphism 
		\begin{tikzcd}[cramped, sep=small]
		T \rar & S_\mathfrak{m} \rar["\numnorm"] & \GG_m
		\end{tikzcd}
		% $T \to S_\mathfrak{m} \overset{\numnorm}{\to} G_\mathfrak{m}$
		is the one induced by the norm 
		map from $K$ to $\Q$.
		\item Show that the $\ell$-adic representation defined by $\numnorm$ is 
		isomorphic to the representation $V_\ell(\mu)$ defined in 
		Chap.~\ref{ch:i},~\ref{sec:I_12}.
	\end{enumerate}
\end{enumerate}

\subsection{Alternative construction}
\label{sec:II_26}
Let $\phi_0 \colon S_{\mathfrak{m}} \to \GL_{V_0}$ be as in \ref{sec:II_25}. If
we compose $\phi_0$ with the map $\varepsilon \colon I \to
S_{\mathfrak{m}}(\Q)$ defined in \ref{sec:II_22}, we obtain a homomorphism
\[
	\phi_0 \circ \varepsilon \colon I \longrightarrow \GL_{V_0}(\Q) =
	\Aut(V_0).
\]
Conversely:
\dpage

\begin{prop}
Let $f \colon I \to \Aut(V_0)$ be a homomorphism. There exists a $\phi_0
\colon S_{\mathfrak{m}} \to \GL_{V_0}$ such that $\phi_0 \circ \varepsilon =
f$ if and only if the following conditions are satisfied:
\begin{enumerate}
	\item The kernel of $f$ contains $U_{\mathfrak{m}}$.
	\item There exists an algebraic homomorphism $\psi\colon \TT \to
		\GL_{V_0}$ such that $\psi(x) = f(x)$ for every $x \in K^\times
		= \TT(\Q)$.
\end{enumerate}
Moreover, such a $\phi_0$ is unique.
\end{prop}
\begin{proof}
The necessity of the conditions (a) and (b) is trivial. 
Conversely, if $f$ has properties (a), (b), it defines a homomorphism
$I/U_{\mathfrak{m}} \to \Aut(V_0)$. On the other hand, since $f$ and $\psi$ agree on $K^\times$
the morphism $\psi$ is equal to 1 on $E_{\mathfrak{m}} = K^\times \cap U_{\mathfrak{m}}$, hence on its
Zariski-closure $\overline{E}_{\mathfrak{m}}$. This means that $\psi$ factors through
\[\begin{tikzcd}[sep=large]
	T \rar & T_{\mathfrak{m}} \rar & \GL_{V_0}.
\end{tikzcd}\]
By the universal property of $S_{\mathfrak{m}}$ (cf.\ \ref{sec:II_13} and
\ref{sec:II_22}), the maps $I/U_{\mathfrak{m}} \to \GL_{V_0}(\Q)$ and
$T_{\mathfrak{m}} \to \GL_{V_0}$ define an algebraic morphism $\phi_0 \colon
S_{\mathfrak{m}} \to \GL_{V_0}$, and one checks easily that $\phi_0$ has the
required properties, and is unique.
\end{proof}

\begin{obs}
Since $U$ is open, property (a) implies that $f$ is \emph{continuous} with
\dpage
respect to the discrete topology of $\Aut(V_0)$. Conversely, any continuous
homomorphism $f\colon I \to \Aut(V_0)$ is trivial on some $U_{\mathfrak{m}}$;
moreover, there is a smallest such $\mathfrak{m}$; it is called the
\strong{conductor}\index{Conductor} of $f$.
\end{obs}

\subsubsection*{Exercise}
Let $\mathfrak{m}$ be a modulus and let $V_0$ be a finite dimensional
$\Q$-vector space. For each $v \notin \Supp(\mathfrak{m})$ let $F_v$ be an
element of $\Aut(V_0)$. Assume:
\begin{enumerate}
	\item The $F_v$'s commute pairwise.
	\item There exists an algebraic morphism $\psi\colon \TT \to \GL_{V_0}$
		such that $\psi(\alpha) = \prod F_v^{v(\alpha)}$ for $\alpha
		\in K^\times$, $\alpha \equiv 1 \pmod{\mathfrak{m}}$, and
		$\alpha > 0$ at each real place.
\end{enumerate}
Show that there exists an algebraic morphism $\phi_0 \colon S_{\mathfrak{m}}
\to \GL_{V_0}$ for which the Frobenius elements are equal to the $F_v$'s.

\subsection{The real case}
The preceding constructions are relative to a given prime number $\ell$.
However, they have an archimedean analogue, as follows: 

Let $\pi \colon \TT \to S_\mathfrak{m}$ be the canonical map defined 
in~\ref{sec:II_23}, and let 
\[
	\pi_{\infty} \colon \TT(R) \to S_\mathfrak{m}(R)
\]
be the corresponding homomorphism of real Lie groups. Since $\TT(R) = (K
\otimes R)^{\times} = \prod_{v \in M_K^\infty} K_v^{\times}$, we can identify
$\TT(R)$ with a direct factor of the idèle group $I$. Let $\pr_\infty$ be the
projection on this factor; the map 
\dpage
\[\begin{tikzcd}
	\alpha_\infty = \pi_\infty \circ \pr_\infty \colon I \arrow[r] & T(R) 
	\arrow[r] & S_\mathfrak{m}(R)
\end{tikzcd}\]
is continuous, and one checks as in~\ref{sec:II_23} that $\alpha_\infty$ 
coincides with $\varepsilon$ on $K^{\times}$. One may then define a map
\[
	\varepsilon_{\infty} \colon I \rightarrow S_{\mathfrak{m}}(R)
\]
by 
\[
	\varepsilon_\infty(a) = \varepsilon(a)\alpha_\infty(a^{-1}).
\]
One has $\varepsilon_\infty(a) = 1$ if $a \in K^{\times}$, hence 
$\varepsilon_\infty$ may be viewed as \emph{a homomorphism of the idèle 
class group} $C = I/K^{\times}$ \emph{into the real Lie gruop} $S_\mathfrak{m}
(R)$.

The main difference with the ``finite'' case is that $\varepsilon_\infty$ 
\emph{is not trivial on the connected component of} $C$, hence has no Galois 
group interpretation.

When one composes $\varepsilon_\infty \colon C \to S_\mathfrak{m}(R)$ with a 
complex character $S_{\mathfrak{m}/C} \to \GG_{m, C}$, one gets a 
homomorphism $C \to C^{\times}$, i.e.\ that the characters obtained in this way 
coincide with the ``Gr\"o\ss encharakter of type ($A_0$)'' of Weil (cf.\ 
\cite{35}, \cite{41}), whose conductor divide $\mathfrak{m}$.

\subsubsection*{Exercise}
Let 
\dpage
\[
	e \colon I \to S_\mathfrak{m}(R) \times \prod_{\ell} S_\mathfrak{m}(\Q_\ell)
\]
be the map defined by $\varepsilon_\infty$ and the $\varepsilon_\ell$'s.
\begin{enumerate}[a)]
\item Show that the image of $e$ is contained in the subgroup
	$S_\mathfrak{m}(A)$ of $S_\mathfrak{m}(R) \times \prod_{\ell}
	S_\mathfrak{m}(\Q_\ell)$, where $A$ denote the ring of adèles of $\Q$,
	and that $e \colon I \to S_\mathfrak{m}(A)$ is continuous (for the
	natural topology of the adelized group $S_\mathfrak{m}(A)$).
\item Let $\pi_A \colon T(A) \to S_\mathfrak{m}(A)$ be the map defined
	by $\pi \colon \TT \to S_\mathfrak{m}$. Show that, if one
	identifies $\TT(A)$ with $I$ in the obvious way, one has 
	\[
		e(x) = \varepsilon(x) \pi_A(x^{-1})
	\]
	where $\varepsilon \colon I \to S_\mathfrak{m}(\Q) \subset
	S_\mathfrak{m}$ is the map defined in~\ref{sec:II_23}. [Note that this
	gives an alternate definition of the $\varepsilon_\ell$'s.]
\item Show that $\varepsilon(I)$ is not open in $S_\mathfrak{m}(A)$ if
	$C_\mathfrak{m} \neq \{1\}$.
\end{enumerate}

\subsection{An example: complex multiplication of abelian varieties}
\label{sec:II_28}
(We give here only a brief sketch of the theory, with a few indications on the
proofs. For more details, see \citeauthor{34}~\cite{34},
\citeauthor{35}~\cite{35}, \citeauthor{41}~\cite{41}, \cite{42} and
\citeauthor{32}~\cite{32}.)

Let $A$ be an abelian variety of dimension $d$ defined over $K$.
Let $\End_K(A)$ be its ring of endomorphisms and put
$\End_K(A)_0 = \End_K(A) \otimes \Q$.

Let $E$ be a number field of degree $2d$, and
\dpage
\[
	i \colon E \to \End_K(A)_0
\]
be an injection of $E$ into $\End_K(A)_0$. The variety $A$ is then said to
have \textquote{complex multiplication} by $E$; in the terminology of
Shimura-Taniyama, it is a variety of \textquote{type (CM)}.

Let $\ell$ be a prime integer and define $T_\ell(A)$ and $V_\ell = T_\ell(A)
\otimes \Q_\ell$ as in Chapter~\ref{ch:i}, \ref{sec:I_12}. These are free
modules over $\Z_\ell$ and $\Q_\ell$, of rank $2d$. The $\Q$-algebra
$\End_K(A)_0$ acts on $V_\ell$; hence the same is true for $E$, and, by
linearity, for $E_\ell = E \otimes_\Q \Q_\ell$. One proves easily:

\begin{lem}
	$V_\ell$ is a free $E_\ell$-module of rank 1.
\end{lem}

Let $\rho\colon \Gal(\algcl K/K) \to \Aut(V_\ell)$ be the $\ell$-adic
representation defined by $A$. If $s \in \Gal(\algcl K/K)$, it is clear that
$\rho(s)$ commutes with $E$, hence with $E_\ell$. But the lemma above implies
that the commuting algebra of $E_\ell$ in $\End_K(V_\ell)$ is $E_\ell$ itself.
Hence, $\rho$ may be identified with a homomorphism
\[
	\rho_\ell\colon \Gal(\algcl K/K) \longrightarrow E^\times_\ell
\]
Let now $T_E$ be the $2d$-dimensional torus attached to $E$ (as $\TT$ is
attached to $K$), so that $T_E(\Q_\ell) = E_\ell^\times$, and $\rho$ takes
values in $T_E(\Q_\ell)$.

\begin{thm}\label{thm:II_28_1}
\begin{enumerate}[(a)]
\item\label{thm:II_28_1a}
	The system $(\rho_\ell)$ is a strictly compatible system of rational
	$\ell$-adic representations of $K$ with values in $T_E$ (in the sense
	\dpage
	of Chap.~\ref{ch:i}, \ref{sec:I_24}).
\item\label{thm:II_28_1b}
	There is a modulus $\mathfrak{m}$ and a morphism
	\[
		\varphi \colon S_{\mathfrak{m}} \longrightarrow T_E
	\]
	such that $\rho$ is the image by $\varphi$ of the canonical system
	$(\varepsilon_\ell)$ attached to $S_{\mathfrak{m}}$, cf.\ 
	\ref{sec:II_23}.
\end{enumerate}
\end{thm}

Moreover, the restriction of $\varphi$ to $T_{\mathfrak{m}}$ can be given
explicitly:

Let $t$ be the tangent space at the origin of $A$. It is a $K$-vector space on
which $E$ acts, i.e.\ an \emph{$(E, K)$-bimodule}. If we view it as an
$E$-vector space, the action of $K$ is given by a homomorphism $j\colon K \to
\End_E(t)$. In particular, if $x \in K^\times$, $\det_E j(x)$ is an element of
$E^\times$; the map $\det_E j\colon K^\times \to E^\times$ is clearly the
restriction of an algebraic morphism $\delta\colon \TT \to T_E$.

\begin{thm}\label{thm:II_28_2}
The map $\delta\colon \TT \to T_E$ coincides with the composition
map 
\begin{tikzcd}[cramped, sep=small]
	\TT \rar & T_{\mathfrak{m}} \rar & S_{\mathfrak{m}} \rar["\varphi"] &
	T_E.
\end{tikzcd}
\end{thm}

\begin{ex}
If $A$ is an \emph{elliptic curve}, $E$ is an imaginary quadratic field, and
the action of $E$ on the one-dimensional $K$-vector space $t$ defines an
embedding $E \to K$. The map $\det_E j\colon K^\times \to E^\times$ is just the
\emph{norm} relative to this embedding.
\end{ex}

\subsubsection*{Indications on the proofs of Theorems 1 and 2}
Part \ref{thm:II_28_1a} of Theorem~\ref{thm:II_28_1} is proved as follows: Let
$S$ denote the finite set of $v \in M_K^0$ where $A$ has \textquote{bad
reduction}. If $v \notin S$, and
\dpage
$\ell \ne p_v$, one shows easily that $p_\ell$ is unramified at $v$ (the
converse is also true, see \cite{32}); moreover the corresponding Frobenius
element $F_{v, \rho_\ell}$ may be identified with the Frobenius endomorphism
$F_v$ of the reduced variety $\widetilde{A}_v$. But $F_v$ commutes with $E$ in
$\End(\widetilde{A}_v)_0$ and the commuting algebra of $E$ in
$\End(\widetilde{A}_v)_0$ is $E$ itself (cf.\ \cite[39]{34}). Hence $F_v$
belongs to $E^\times = T_E(\Q)$ and this implies \ref{thm:II_28_1a}.

Theorem~\ref{thm:II_28_2} and part \ref{thm:II_28_1b} of
Theorem~\ref{thm:II_28_1} are less easy; they are proved, in a somewhat
different form in \citeauthor{34}~\cite{34} (see also \cite{32}). Note that one
could express them (as in \ref{sec:II_26}) by saying that \emph{there exists a
homomorphism} $f\colon I \to E^\times$ (where $I$ denotes, as usual, the group
of idèles of $K$) \emph{having the following properties:}
\begin{enumerate}
\item $f$ is trivial on $U_{\mathfrak{m}}$, for some modulus $\mathfrak{m}$
	with support $S$.
\item If $v \notin S$, the image by $f$ of a uniformizing parameter at
	$v$ is the Frobenius element $F_v \in E^\times$.
\item If $x \in K^\times$ is a principal idèle, one has $f(x) = \det_E j(x)$.
\end{enumerate}
This is essentially what is proved in \cite[148]{34}, formula (3),
except that the result is expressed in terms of ideals instead of
ideles, and $\det_E j(x)$ is written in a different form, namely
``$ \prod_{\alpha} \Nm_{K/K^\times}(x)^{\psi_\alpha} $''.

\begin{obs}
Another possible way of proving Theorems~\ref{thm:II_28_1} and
\ref{thm:II_28_2} is the following:

Let $\ell$ be a prime integer distinct from any of the $p_v$, $v \in S$.  One
then sees that the Galois-module $V_\ell$ is of Hodge-Tate type in the sense of
Chapter III, 1.2 (indeed, the corresponding local modules
\dpage
are associated with \emph{$\ell$-divisible groups}, and one may apply Tate's
theorem \cite{39}). Hence $\rho_\ell$ is \textquote{locally algebraic} (Chapter
III, \emph{loc.\ cit.}), and using the theorem of Chapter III, 2.3 one sees it
defines a morphism $\varphi \colon S_{\mathfrak{m}} \to T_E$. One has
$\varphi\circ\varepsilon_\ell = \rho_\ell$ by construction; the same is true
for any prime number $\ell'$, since $\varphi\circ\varepsilon_{\ell'}$ and
$\rho_{\ell'}$ have the same Frobenius elements for almost all $v$. This proves
part \ref{thm:II_28_1b} of Theorem~\ref{thm:II_28_1}. As for
Theorem~\ref{thm:II_28_2}, one uses the explicit form of the Hodge-Tate
decomposition of $V_\ell$, as given by \citeauthor{39}~\cite{39}, combined with
the results of the Appendix to Chapter III.
\end{obs}

\section{Structure of \texorpdfstring{$T_{\mathfrak{m}}$}{Tm} and applications}

\subsection{Structure of \texorpdfstring{$X(T_{\mathfrak{m}})$}{X(Tm)}}
\label{sec:II_31}
If $w$ is a complex place of $\algcl\Q$, the completion of $\algcl\Q$ with
respect to $w$ is isomorphic to $\C$; the decomposition group of $\omega$ is
thus cyclic of order 2; its non-trivial element will be denoted by $c_w$ (the
\textquote{Frobenius at the infinite place $w$}). The $c_w$'s are conjugate in
$G = \Gal(\algcl\Q/\Q)$; let $C_\infty$ denote their conjugacy class. (By a
theorem of Artin \cite[257]{1}, the elements of $C_\infty$ are the only
non-trivial elements of finite order in $G$.)

Let $X(\TT)$ be the character group of the torus $\TT$, cf.\ \ref{sec:II_11}; we
write $X(\TT)$ additively and put $Y(\TT) = X(\TT) \otimes_\Z \Q$. We decompose
$Y$ as a direct sum $Y = Y^0 \oplus Y^- \oplus Y^+$ of $G$-invariant subspaces, as
follows (cf.\ Appendix, \ref{sec:II_A2})
\begin{align*}
	Y^0 &= Y^G = \{ y \in Y \mid gy = y \; \text{for all } g\in G \} \\
	Y^- &= \{ y \in Y \mid cy = -y \; \text{for all } c\in C_\infty \}
\end{align*}
and $Y^+$ is a $G$-invariant supplement to $Y^0 \oplus Y^-$ in $Y$; one proves
\dpage
easily that $Y^+$ is unique, cf.\ Appendix, \emph{loc. cit.}

More explicitly, if $\sigma \in \TT$ is an embedding of $K$ into $\algcl\Q$,
let $[\sigma] \in X(\TT)$ be the corresponding character of $T$; the
$[\sigma]$'s, $\sigma \in \Gamma$, form a basis of $X(\TT)$ and $g\cdot[\sigma]
= [g\circ\sigma]$ if $g \in G$. The space $Y^0$ is generated by the norm
element $\sum_{\sigma \in \Gamma} [\sigma]$, and its $G$-invariant supplement
is
\[
	Y^- \oplus Y^+ = \left\{ \sum_{\sigma \in \Gamma} b_\sigma \, [\sigma]
	\mid b_\sigma \in \Q, \; \sum_{\sigma \in \Gamma} b_\sigma = 0 \right\}.
\]
Hence, any character $\chi \in X(\TT)$ can be written in the form
\begin{gather}
	\chi = a \sum_{\sigma\in \Gamma} [\sigma] + \sum_{\sigma \in \Gamma}
	b_\sigma \, [\sigma] \tag{$*$}\label{eqn:II_31_star} \\
	a, b_\sigma \in \Q, \; \sum_{\sigma} b_\sigma = 0, \; a + b_\sigma \in
	\Z. \notag
\end{gather}
(In particular, we see that $da \in \Z$ where $d = [K : \Q]$.) The subspace
$Y^-$ can now be described as follows
\[
	Y^- = \left\{ \sum_{\sigma} b_\sigma \, [\sigma] \mid b_\sigma \in \Q, \;
	\sum_{\sigma} b_\sigma = 0, \; b_{c\sigma} = -b_\sigma \text{ for all }
	c \in C_\infty \text{ and } \sigma \in \Gamma \right\}.
\]
On the other hand, the projection $\TT \to T_{\mathfrak{m}}$ defines an
injection of $X(T_{\mathfrak{m}})$ into $X(\TT)$; we identify
$X(T_{\mathfrak{m}})$ with its image under this injection.

\begin{prop}
	$X(T_{\mathfrak{m}}) \otimes_\Z \Q = Y^0 \oplus Y^-$.
\end{prop}
This follows from Appendix, \ref{sec:II_A2}.

\begin{corp}
	\dpage
	The character group $X(T_{\mathfrak{m}})$ is a sublattice of finite
	index of $X(\TT) \cap (Y^0 \oplus Y^-)$.
\end{corp}
\begin{corp}
	If $\chi \in X(\TT)$ is written in the form \eqref{eqn:II_31_star},
	then $2a \in \Z$.
\end{corp}
In fact, given $c \in C_\infty$ and $\sigma \in \Gamma$, we have
\[
	2a = 2a + b_\sigma + b_{c\sigma} = (a+b_\sigma) + (a+b_{c\sigma}) \in
	\Z.
\]

\subsection{The morphism \texorpdfstring{$j^* \colon \GG_m \to
T_{\mathfrak{m}}$}{j*: Gm -> Tm}}
We have seen that any character $\chi \in X(T_\mathfrak{m})$ can be written in 
the form
\[
	\chi = a \sum_{\sigma \in \Gamma} [\sigma] + \sum_{\sigma \in \Gamma} 
	b_{\sigma}[\sigma]
\]
with $a, b_\sigma \in \Q$, $\sum b_{\sigma} = 0$, $2a \in \Z$. Hence $\chi 
\mapsto 2a$ defines a homomorphism $j \colon X(T_\mathfrak{m}) \to X(\GG_m) = 
\Z$ and we obtain by duality a morphism of algebraic groups $j^{*} \colon \GG_m 
\to T_\mathfrak{m}$. If $\phi_0 \colon S_\mathfrak{m} \to \GL_{V_0}$ is a 
representation of $S_\mathfrak{m}$, we obtain by composition with $j^{*}$ a 
morphism of algebraic groups $\GG_m \to \GL_{V_0}$. This representation of 
$\GG_m$ defines (and is defined by) a grading $V_0 = \sum_{i \in \Z} V_0^{(i)}$ 
of $V_0$; recall that $\GG_m$ acts on $V_0^{(i)}$ by means of the character $i 
\in \Z = X(\GG_m)$.

We say that $V_0$ is homogeneous of degree $n$ if $V_0 = V_0^{(n)}$.
\dpage
\begin{obs}
For representations coming from the $\ell$-adic homology $H_{*}(\algcl{X})$ of 
a projective smooth variety $X$, the grading defined above should coincide with 
the natural one: $H_{*}(\algcl{X}) = \sum_i H_i(\algcl{X})$.
\end{obs}
\subsubsection*{Exercise}
\begin{enumerate}
\item\label{exr:II_32_1} Let $\numnorm \colon S_\mathfrak{m} \to \GG_m$ be the morphism
	defined in Exercise~\ref{exr:II_25_2} of~\ref{sec:II_25}. Show
	that $\numnorm \circ j \colon \GG_m \to S_\mathfrak{m} \to
	\GG_m$ is $\lambda \mapsto \lambda^2$.  Show that any morphism
	$S_\mathfrak{m} \to \GG_m$ is equal to
	$\varepsilon\numnorm^{n}$, where $\varepsilon$ is a character
	of $C_\mathfrak{m}$ with values in $\{\pm 1\}$ and $n \in \Z$.
\item Let $\phi \colon S_\mathfrak{m} \to \GL_{V_0}$ be a linear
	representation of $S_\mathfrak{m}$.
	
	Assume $\phi$ is homogeneous of degree $d$, and put $h = \dim_{V_0}$.
	\begin{enumerate}[a)]
	\item Show that $dh$ is even (apply Exerc.~\ref{exr:II_32_1} to
		$\det(\phi) \colon S_\mathfrak{m} \to \GG_m$).
	\item Prove that there exists on $V_0$ a positive definite quadratic
		form $\Q$ such that \[ \Q(\rho(x)y) = \numnorm(x)^d\Q(y) \] for
		any $y \in V_0$ and any $x \in S_\mathfrak{m}(\Q)$. [Let $H$ be
		the kernel of $\numnorm \colon S_\mathfrak{m} \to \GG_m$. Using
		the fact that $H(R)$ is compact, prove the existence of a
		positive definite quadratic form $\Q$ on $V_0$ invariant by
		$H$; then note that $S_\mathfrak{m}$ is generated by $H$ and
		$j^{*}(\GG_m)$.] 
	\end{enumerate}
\end{enumerate}

\subsection{Structure of \texorpdfstring{$T_{\mathfrak{m}}$}{Tm}}
We need first some notations:

Let $H_c$ be the closed subgroup of $G = \Gal(\algcl\Q/\Q)$ generated by
$C_\infty$ (cf.\ \ref{sec:II_31}). There is a unique continuous homomorphism
\dpage
$\varepsilon \colon H_c \to \{ \pm 1 \}$ such that $\varepsilon(c) = -1$ for
all $c \in C_\infty$. Indeed the unicity of $\varepsilon$ is clear, and one
proves its existence by taking the restriction to $H_c$ of the homomorphism $G
\to \{ \pm 1 \}$ associated with an imaginary quadratic extension of $\Q$. We
let $H = \Ker(\varepsilon)$. The groups $H$ and $H_c$ are closed invariant
subgroups of $G$, and $(H: H_c) = 2$.

Let now $K$ be, as before, a finite extension of $\Q$; we identify
it with a subfield of $\Q$; let $G_K = \Gal(\algcl\Q/K)$ be the corresponding
subgroup of $G$. The field $K$ is \emph{totally real} if and only if all the
elements $c$ of $C_\infty$ act trivially on $K$, i.e.\ if and only if $G_K$
contains $G_c$. Hence, there exists a \emph{maximal totally real subfield}
$K_0$ of $K$, whose Galois group is $G_{K_0} = G_K \cdot H_c$. We let $K_1$, be
the field corresponding to $G_K \cdot H$. We have
\[
	K_0 \subset K_1 \subset K \qquad \text{and} \qquad
	[K_1 : K_0] = 1 \text{ or } 2.
\]
As shown by Weil (cf.\ \cite[4]{47}) the fields $K_0$ and $K_1$ are closely
connected to the groups $T_{\mathfrak{m}}$ relative to $K$. Indeed, if $\chi =
\sum_{\sigma} b_\sigma[\sigma]$ is an element of the group denoted by $Y^-$ in
\ref{sec:II_31}, we have $b_{c\sigma} = -b_\sigma$ for all $c \in C_\infty$. If
$h = c_1 \cdots c_n$, this gives
\[
	b_{h\sigma} = (-1)^n b_\sigma = \varepsilon(h) b_\sigma
\]
and by continuity the same holds for all $h \in H_c$. One deduces from this:

\begin{prop}
	The norm map defines an isomorphism of the space $Y_{K_1}^0$ relative
	to $K$ onto the space $Y_K^-$ relative to $K$.
\end{prop}

More precisely, if $\chi_1 = \sum b_{\sigma_1}[\sigma_1]$ belongs to
\dpage
$Y_{K_1}^-$, where $\sigma_1 \in \Gamma_{K_1}$, the image of $\chi_1$, by the
norm map is
\[
	\Nm_{K_1/K_0}^*(\chi_1) = \sum_{\sigma} b_{\sigma/K_1}[\sigma], \qquad
	\sigma \in \Gamma_K,
\]
where $\sigma/K_1$ is the restriction of $\sigma$ to $K$. It is clear that this
map is injective. Conversely, if $\chi = \sum_{\sigma} b_\sigma[\sigma]$
belongs to $Y_K^-$, we saw above that $b_{h\sigma} = \varepsilon(h)b_\sigma$
for all $h \in H_c$, hence $b_{h \sigma} = b_\sigma$ for $h \in H$ and of
course also for $h \in H\cdot G_K$. This shows that $b_\sigma$ depends only on
the restriction of $\sigma$ to $K_1$, and hence that $\chi$ belongs to the
image of the norm map.

\begin{corp}
	The tori $T_{\mathfrak{m}}$ attached to $K$ and $K_1$ are isogenous to
	each other.
\end{corp}
There remains to describe the tori $T_{\mathfrak{m}}$ attached to $K_1$.  There
are two cases:
\begin{enumerate}[(1)]
\item $K_1 = K_0$.
	\emph{In this case, we have $Y^- = 0$ and $T_{\mathfrak{m}}$ is
	one-dimensional, and isomorphic to $\GG_m$.}

	Indeed, if $\chi = \sum_{\sigma} b_\sigma[\sigma]$ belongs to $Y^-$,
	and $c \in C_\infty$, we have $b_{c \sigma} = -b_\sigma$ (cf.\
	\ref{sec:II_31}) but also $b_{c \sigma} = b_\sigma$ since $c \in G_K
	\cdot H_c = G_K \cdot H$. This shows that $b_\sigma = 0$ for all
	$\sigma$, hence $Y^- = 0$.

\item $[K_1 : K_0] = 2$.
	The field $K_1$ is then a \emph{totally imaginary quadratic extension}
	of $K_0$ (and it is the only one contained in $K$, as one checks
	readily). In this case $Y^-$ is of dimension $d = [K_0 : \Q]$ and
	$T_{\mathfrak{m}}$ is $(d+1)$-dimensional.

	More precisely, the space $Y$ attached to $K_1$ is $2d$-dimensional
	\dpage
	and the involution $\sigma$ of $K_1$ corresponding to $K_0$ decomposes
	$Y$ in two eigenspaces of dimension $d$ each; the space $Y^-$ is the
	one corresponding to the eigenvalue $-1$ of $\sigma$. This is proved by
	the same argument as above, once one remarks that all $c \in C_\infty$
	induce $\sigma$ on $K_1$.
\end{enumerate}

\begin{obs}
In this last case (which is the most interesting one), the torus
$T_{\mathfrak{m}}$ is isogenous to the product of $\GG_m$ by the
$d$-dimensional torus kernel of the norm map from $K_1$ to $K_0$.
\end{obs}

\subsection{How to compute Frobeniuses} %3.4
\label{sec:II_34}
Let $\phi$ be a linear representation of $S_\mathfrak{m}$ of degree
$n$. By extending the groundfield, the restriction of $ \phi $ to
$ T_\mathfrak{m} $ can be put in diagonal form; let $ \chi_1,\cdots, \chi_n $ 
be the $ n $ characteres of $T_\mathfrak{m}$ so obtained and write (in additive 
notation)

\[
	\chi_i = \sum_{\sigma \in \Gamma} n_\sigma (i)[\sigma], \qquad
	(n_\sigma(i) \in \Z).
\]

We say that $ \chi_i $ is \strong{positive} if all the $n_\sigma(i)$'s are 
$\geq 0$. Let $v \in \Supp(\mathfrak{m})$, and let $F_v \in S_\mathfrak{m}(\Q)
$ be the corresponding Frobenius element, cf.\ \ref{sec:II_23}. Since
$C_\mathfrak{m} = S_\mathfrak{m} / T_\mathfrak{m}$ is finite, there exists an
integer $N \geq 1$ such that $ F_v \in T_\mathfrak{m}(\Q)$. If $\mathfrak{p}_v$
is the prime ideal of $v$, this means that there exist $ \alpha \in K^\times$,
with $\mathfrak{p}_v^N = ( \alpha )$, $ \alpha \equiv 1 \mod \mathfrak{m}$, and
$ \alpha > 0 $ at all real places of $K$.
 
\dpage

\begin{prop}\label{prop:II_34_1}
	The eigenvalues of $ \phi(F_v^N)$ are the numbers 
\[
	\chi_i( \alpha ) = \prod_\sigma \sigma( \alpha )^{n_\sigma(i)}, \qquad
(i = 1,\dots,n).
\]
This is trivial by construction, because $F_v^N$ is the image of $\alpha$ under
$ \TT(\Q) \rightarrow T_\mathfrak{m}(\Q)$.
\end{prop}

\begin{corp}\label{corp:II_34_1} 
The eigenvalues of $\phi(F_v)$ are $\{ p_v \}$-units (i.e.\ they are units at 
all places of $\algcl{\Q}$ not dividing $p_v$).
\end{corp}

\begin{corp}\label{corp:II_34_2}
Let $z_1,\dots,z_n$ be the eigenvalues of $\phi(F_v)$, indexed so that $z_i^N 
= \chi_i(\alpha)$. Let $w$ be a place of $\algcl{\Q}$ dividing $p_v$, 
normalized so that $w(p_v) = v(p_v) = e_v$. Then $w(z_i) = 
\sum_{\substack{\sigma\in \Gamma \\ w \circ \sigma}} n_\sigma(i)$.
\end{corp}

We have $w(z_i^N) = w(\prod_{\sigma \in \Gamma} \left(\sigma
(\alpha)^{n_\sigma(i)}\right) = \sum_{\sigma \in \Gamma}n_\sigma(i)
w \circ \sigma(\alpha)$, and
\begin{align*}
	w \circ \sigma(\alpha) &= 0 &&\text{if} \quad w \circ \sigma \neq v,\\
	w \circ \sigma(\alpha) &= N &&\text{if} \quad w \circ \sigma = v,
\end{align*}
since $(\alpha) = \mathfrak{p}_v^N$.

Hence the result.

\begin{corp}\label{corp:II_34_3}
Let $\ell$ be a prime number and let $\phi_\ell \colon \Gal(\algcl{K}
/K) \rightarrow \Aut(V_\ell)$ be the $\ell$-adic representation of $K$ 
associated to $\phi$. Then $\phi_\ell$ is integral (cf.\ Ch.~\ref{ch:i},~\ref{sec:I_22}) if
and only if all the characteres $\chi_i$ occuring in $\phi$ are positive.
\end{corp}

\begin{proof}[ of Corollary 3]
Assume firts the $\chi_i$'s are positive. Let $v \not\in \Supp(m)$ and let
$z_1,\dots,z_n$ be the corresponding eigenvalues of
\dpage
$F_v$ as in Corollary \ref{corp:II_34_2}. Corrollaries \ref{corp:II_34_1} and 
\ref{corp:II_34_2} show that the $w(z_i)$ are positive for all valuations $w$
of $\algcl\Q$; hence the $z_i$ are integral over $\Z$. Hence the $\phi_\ell$'s
are integral.

Conversely, assume $\phi_\ell$ is integral for some $\ell$. There exists a 
finite subset $S'$ of $M_K^0$, containing $\Supp(\mathfrak{m})$, such that if
$v \not\in S'$, the eigenvalues of $\phi(F_v)$ are integral. Choose a prime
number $p$ which splits completely in $K$ and is such that $p_v = p$ implies
$v \not\in S'$. Let $w$ be a valuation of $\algcl\Q$ dividing $p$. The
valuations $w \circ \sigma$, $\sigma \in \Gamma$, are pairwise inequivalent.
Let $\sigma \in \Gamma$; and let $v$ be the normalized valuation of $K$ 
equivalent to $w \circ \sigma$ so that $\lambda v = w \circ \sigma$ for some
$\lambda > 0$. Let $z_1,\dots,z_n$ be the eigenvalues of $\phi(F_v)$. By
Corollary~\ref{corp:II_34_2}, $w(z_i) = \lambda n_\sigma (i)$. Since the $z_i$ 
are integral, this shows that the $n_\sigma(i)$'s are all positive.
\end{proof}

\begin{prop}\label{prop:II_34_2}
Let $v \not\in \Supp(\mathfrak{m})$ and let $\chi$ be a character of
$S_{\mathfrak{m}}$. Let $\chi_\TT \in X(T_{\mathfrak{m}})$ be the restriction
of $\chi$ to $T_{\mathfrak{m}}$ and let $i = j(\chi_\TT)$ be the integer
defined in~\ref{sec:II_32}.  Then, for any archimedian absolute value $\omega$
of $\algcl{\Q}$ extending the usual absolute value of $\algcl{Q}$, we have
\[
	\omega(\chi (F_v)) = (\numnorm_v)^{i/2}.
\]
\end{prop}

\begin{proof}
Si $\chi = a \sum_{\sigma \in \Gamma} [\sigma] + \sum_{\sigma \in \Gamma}
b_\sigma[\sigma]$ as in~\ref{sec:II_31}, we have
\[
\omega(\chi(F_v))^N = \omega(\chi(F_v^N))
= \prod_{\sigma} \omega \circ \sigma(\alpha)^a \cdot \prod_{\sigma} 
\omega \circ \sigma(\alpha)^{b_\sigma},
\]

\dpage
and $ \prod_{\sigma} \omega \circ \sigma(\alpha)^a = \omega(\Nm(\alpha))^a
= \numnorm v^{i N/2}$, where $i = 2a$. It remains to show that 
$\chi = \prod_{\sigma} \omega \circ \sigma(\alpha)^{b_\sigma}$ is equal to $1$. 
Let $c = c_\omega$ be the ``Frobenius'' attached to $\omega$ (cf.~\ref{sec:3.1}). 
Since $b_\sigma + b_{c \sigma} = 0$, we have $x \cdot y = 1$ with 
$y = \prod_{\sigma} \omega \circ \sigma(\alpha)^{b_{c\sigma}}$. But 
$y = \prod_{\tau} \omega \circ c \circ \tau(\alpha)^{b_\tau}$, and, since 
$\omega \circ c = \omega$, we have $y = x$, hence $x^2 = 1$, and $x = 1$, 
since $x > 0$.
\end{proof}

\subsubsection*{Exercises}
\begin{enumerate}
	\item Check the product formula for the eigenvalue of the $\phi(F_v)$. 
(Use Cor.~\ref{corp:II_34_1} and \ref{corp:II_34_2} to Prop.~\ref{prop:II_34_1}
and Prop.~\ref{prop:II_34_2}).
	\item Show that Prop.~\ref{prop:II_34_2} and Cor.\ref{corp:II_34_1} and
\ref{corp:II_34_2} to Prop.~\ref{prop:II_34_1} determine the eigenvalues of the
$\phi(F_v)$'s up to multiplication by roots of unity. 
	\item (Generalization of Cor.~\ref{corp:II_34_1} to
Prop.~\ref{prop:II_34_1}). Let $(\rho_\ell)$ be a strictly compatible system of
rational $\ell$-adic representations, with exceptional set $S$ (cf.\
Chap.~\ref{ch:i},~\ref{sec:I_23}). Show that, for any $v \in M_K^0 \setminus
S$, the eigenvalues of $F_{v,\rho_\ell}$, $\ell \neq p_v$, are $p_v$-units.
\end{enumerate}

\begin{subappendices}
\section{Killing arithmetic groups in tori}

\subsection{Arithmetic groups in tori}
Let $A$ be a linear algebraic group over $\Q$, and let $\Gamma$ be a subgroup
of the group $A(\Q)$ of rational points of $A$. Then $\Gamma$ is said to be an
\strong{arithmetic subgroup}\index{Arithmetic subgroup} if for any algebraic
embedding
\dpage
$A \subseteq \GL_n$ ($n$ arbitrary) the groups $\Gamma$ and $A(\Q) \cap
\GL_n(\Z)$ are \strong{commensurable}\index{Commensurable (subgroups)} (two
subgroups $\Gamma_1$, $\Gamma_2$ are said to be commensurable if $\Gamma_1 \cap
\Gamma_2$ is of finite index in $\Gamma_1$ and $\Gamma_2$). It is well-known
that it suffices to check that $\Gamma$ and $A(\Q) \cap \GL_n(\Z)$ are
commensurable for one embedding $A \subseteq \GL_n$.

\begin{ex}
Let $K$ be a number field and let $E$ be the group of units of $K$.
Then $E$ is an arithmetic subgroup of $\TT = \WRes_{K/\Q}(\GG_m)$.
\end{ex}
If $\TT$ is a torus over $\Q$, let $\TT^0$ be the intersection of the kernels
of the homomorphisms of $\TT$ into $\GG_m$. The torus $\TT$ is said to be
\strong{anisotropic}\index{Anisotropic (torus)} if $\TT = \TT^0$; in terms of
the character group $X = X(\TT)$ this means that $X$ has no non-zero elements
which are left fixed by $G = \Gal(\algcl\Q/\Q)$.

\begin{thm}
Let $\TT$ be a torus over $\Q$, and let $\Gamma$ be an arithmetic subgroup of
$\TT$. Then $\Gamma \cap \TT^0$ is of finite index in $\Gamma$, and the
quotient $\TT^0(R)/\Gamma \cap \TT^0$ is compact.
\end{thm}
This is due to T. Ono; for a proof of a more general statement (``Godement's
conjecture'') see \citeauthor{18}~\cite{18}.

\begin{cor}
Let $\TT$ be a torus over $\Q$, and let $\Gamma$ be an arithmetic subgroup of
$\TT$. If $\TT$ is anisotropic, then $\TT(R)/\Gamma$ is compact.
\end{cor}

\subsubsection*{Exercise}
\dpage
Let $\TT$ be a torus over $\Q$, with character group $X$.
\begin{enumerate}[a)]
\item Show that
	\[
		\TT(\Q) = \Hom_{\Gal} (X, \algcl{\Q}^\times).
	\]
\item Let $U$ be the subgroup of $\algcl{\Q}^\times$ whose elements are the 
algebraic units of $\algcl{\Q}$. Let
\[
	\Gamma = \Hom_{\Gal}(X, U)
\]
Show that $\Gamma$ is an arithmetic subgroup of $\TT(\Q)$ and that any 
arithmetic subgroup of $\TT(\Q)$ is contained in \TT.
\end{enumerate}

\subsection{Killing arithmetic subgroups}
\label{sec:II_A2}
Let $\TT$ be a torus over $\Q$, and let $X(\TT)$ be its character group; put
$Y(\TT) = X(\TT) \otimes_\Z \Q$. Let $\Lambda$ be the set of classes of
$\Q$-irreducible representations of $G = \Gal(\algcl\Q/\Q)$ through its finite
quotients. For each $\lambda \in \Lambda$, let $Y$ be the corresponding
isotypic sub-$G$-module of $Y$, i.e.\ the sum of all sub-$G$-modules of $Y$
isomorphic to $\lambda$. One has the direct sum decomposition
\[
	Y = \coprod_{\lambda \in \Lambda} Y_\lambda
\]
Let $Y^0 = Y_1$, where 1 is the unit representation of $G$; let $Y^-$ be the
sum of those $Y$ where for all the infinite Frobeniuses $c \in C_\infty$
(cf.\ \ref{sec:II_31}) we have $\lambda(c) = -1$; let $Y^+$ be the sum of the
other $Y_\lambda$.  We have
\dpage
\begin{align*}
	Y^0 &= Y^G = \{ y \in Y \mid gy = y \; \text{for all } g\in G \} \\
	Y^- &= \{ y \in Y \mid cy = -y \; \text{for all } c\in C_\infty \}, \\
	Y   &= Y^0 \oplus Y^- \oplus Y^+.
\end{align*}
Note that $Y = Y^0$ if and only if $\TT$ is anisotropic.  If $c \in C_\infty$,
and $H = \{ 1, c \}$, then, since $\TT(\R) = \Hom_H(X(\TT),\C^\times)$, we see
that \emph{$\TT(\R)$ is compact if and only if $Y = Y^-$.}

\begin{prop}
	Let $\Gamma$ be an arithmetic subgroup of the torus $\TT$,
	and $\overline{\Gamma}$ its Zariski closure (cf.\ \ref{sec:II_12}). Then:
	\begin{equation}
		Y(\TT/\overline{\Gamma}) = Y^0 \oplus Y^-.
		\tag{$*$}
		\label{eqn:II_A2_star}
	\end{equation}
\end{prop}
{[Since the torus $\TT/\overline{\Gamma}$ is a quotient of $\TT$, we identify
$Y(\TT/\overline{\Gamma})$ with a submodule of $Y(\TT)$.]}
\begin{proof}
	Suppose first that $Y$ is \emph{irreducible}, i.e.\ that $\TT$ has no
	proper subtori and is $\ne 0$.

	If $Y = Y^0$, then $\TT$ is isomorphic to $\GG_m$ and hence $\Gamma$ is
	finite.  This shows that $Y(\TT/\overline{\Gamma}) = Y(\TT)$, hence
	\eqref{eqn:II_A2_star}. If $Y = Y^-$, then $\TT(\R)$ is compact. Since
	$\Gamma$ is a discrete subgroup of $\TT(\R)$, it is finite.  Hence
	$Y(\TT/\overline{\Gamma}) = Y(\TT)$ and \eqref{eqn:II_A2_star} follows.

	If $Y = Y^+$, then $\TT(\R)$ is not compact. Consequently, $\Gamma$ is
	infinite since $\TT(\R)/\TT$ is compact by Ono's theorem. Hence
	$\overline{\Gamma}$ is an algebraic subgroup of $\TT$ of dimension $\ge
	1$. Its connected component is a non-trivial subtorus of $\TT$. This
	shows that $\overline{\Gamma} = \TT$, hence $Y(\TT/\overline{\Gamma}) =
	0$. Hence again \eqref{eqn:II_A2_star}.

	The general case follows easily from the irreducible one; for
	\dpage
	instance, choose a torus $\TT'$ to $\TT$ which splits in direct product
	of irreducible tori and note that $\Gamma$ is commensurable with the
	image by $\TT' \to \TT$ of an arithmetic subgroup of $\TT$.
\end{proof}

\subsubsection*{Exercise}
Let $y \in Y$. Define $\operatorname{N} y$ as the mean value of the transforms of
$y$ by $G$.
\begin{enumerate}[label=\textit{\alph*}.]
\item Prove that $\operatorname{N}$ is a $G$-linear projection of $Y$ onto
	$Y^0$ hence $\Ker(\operatorname{N}) = Y^- \oplus Y^+$.
\item Prove that $Y$ is generated by the elements $cy + y$, with $y \in
	\Ker(\operatorname{N})$ and $c \in C_\infty$.
\end{enumerate}
\end{subappendices}
