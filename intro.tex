\chapter*{Introduction}
The \textquote{$\ell$-adic representations} considered in this book
are the algebraic analogue of the locally constant sheaves
(or \textquote{local coefficients}) of Topology. A typical example
is given by the $\ell^n$-th division points of abelian varieties
(cf. chap.I, 1.2); the corresponding $\ell$-adic spaces, first
introduced by Weil [40] are one of our main tools in the
study of these varieties. Even the case of dimension 1
presents non trivial problems; some of them will be
studied in chap.IV.

The general notion of an $\ell$-adic representation was
first defined by Taniyama
