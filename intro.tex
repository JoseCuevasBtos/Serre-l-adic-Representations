\chapter*{Introduction}
The \textquote{$\ell$-adic representations} considered in this book are the
algebraic analogue of the locally constant sheaves (or \textquote{local
coefficients}) of Topology. A typical example is given by the $\ell^n$-th
division points of abelian varieties (cf. chap.~\ref{ch:i}, \ref{sec:I_12});
the corresponding $\ell$-adic spaces, first introduced by
\citeauthor{40}~\cite{40} are one of our main tools in the study of these
varieties. Even the case of dimension 1 presents non trivial problems; some of
them will be studied in chap.~\ref{ch:iv}.

The general notion of an $\ell$-adic representation was first defined by
Taniyama \cite{35} (see also the review of this paper given by Weil in
\emph{Math.\ Rev.}, 20, 1959, rev.1667).  He showed how one can relate
$\ell$-adic representations relative to different prime numbers $\ell$
\emph{via} the properties of the Frobenius elements (see below). In the same
paper, Taniyama also studied some \emph{abelian} representations which are
closely related to complex multiplication (cf.\ \citeauthor{41}~\cite{41},
\cite{42} and \citeauthor{34}~\cite{34}). These abelian representations,
together with some applications to elliptic curves, are the subject matter of
this book.

There are four Chapters, whose contents are as follows:

Chapter~\ref{ch:i} begins by giving the definition and some examples of
$\ell$-adic representations (\S\ref{sec:I_1}). In \S\ref{sec:I_2}, the ground
field is assumed to be a \emph{number field}. Hence, Frobenius elements are
defined, and one has the notion of a \emph{rational} $\ell$-adic
representation: one for which their characteristic polynomials have rational
coefficients (instead of merely $\ell$-adic ones). Two representations
corresponding to different primes are \emph{compatible} if the characteristic
polynomials of their Frobenius elements are the same (at least almost
everywhere); not much is known about this notion in the non abelian case (cf.\
the list of open questions at the end of \ref{sec:I_23}). A last section shows
how one attaches $L$-functions to rational $\ell$-adic representations; the
well known connection between equidistribution and analytic properties of
$L$-functions is discussed in the Appendix.

Chapter~\ref{ch:ii} gives the construction of some abelian $\ell$-adic
representations of a number field $K$. As indicated above, this construction is
essentially due to Shimura, Taniyama and Weil. However, I have found it
convenient to present their results in a slightly different way, by defining
first some algebraic groups over $\Q$ (the groups $\mathbb{S}$) whose
representations -- in the usual algebraic sense -- correspond to the sought for
$\ell$-adic representations of $K$.  The same groups had been considered before
by Grothendieck in his still conjectural theory of \textquote{motives} (indeed,
motives are supposed to be \textquote{$\ell$-adic cohomology without $\ell$} so
the connection is not surprising). The construction of these groups
$\mathbb{S}$ and of the $\ell$-adic representations attached to them, is given
in \S\ref{sec:II_2} (\S\ref{sec:II_1} contains some preliminary constructions
on algebraic groups, of a rather elementary kind). I have also briefly
indicated what relations these groups have with complex multiplication (cf.\
\ref{sec:II_28}). The last \S{} contains some more properties of the
$S_{\mathfrak{m}}$'s.

Chapter~\ref{ch:iii} is concerned with the following question: let $p$ be an
abelian $\ell$-adic representation of the number field $K$; can $p$ be obtained
by the method of chap.~\ref{ch:ii}?  The answer is: this is so if and only if
$p$ is \emph{\textquote{locally algebraic}} in the sense defined in
\S\ref{sec:III_1}.  In most applications, local algebraicity can be checked
using a result of Tate saying that it is equivalent to the existence of a
\textquote{Hodge-Tate} decomposition, at least when the representation is
semi-simple. The proof of this result of Tate is rather long, and relies
heavily on his theorems on $p$-divisible groups \cite{39}; it is given in the
Appendix.  One may also ask whether any abelian rational semi-simple
$\ell$-adic representation of $K$ is \emph{ipso facto} locally algebraic; this
may well be so, but I can prove it only when $K$ is a composite of quadratic
fields; the proof relies on a transcendency result of Siegel and Lang (cf.\
\S\ref{sec:III_3}).

Chapter~\ref{ch:iv} is concerned with the $\ell$-adic representation
$\rho_\ell$, defined by an \emph{elliptic curve} $E$. Its aim is to determine,
as precisely as possible, the image of the Galois group by $\rho_\ell$, or at
least its \emph{Lie algebra}. Here again the ground field is assumed to be a
number field (the case of a function field has been settled by
\citeauthor{10}~\cite{10}).  Most of the results have been stated in \cite{25},
\cite{31} but with at best some sketches of proofs. I have given here complete
proofs, granted some basic facts on elliptic curves, which are collected in
\S\ref{sec:IV_1}. The method followed is more \textquote{global} than the one
indicated in \cite{25}. One starts from the fact, noticed by Cassels and
others, that the number of isomorphism classes of elliptic curves isogenous to
$E$ is \emph{finite}; this is an easy consequence of \v Safarevi\v c's theorem
(cf.\ \ref{sec:IV_14}) on the finiteness of the number of elliptic curves
having good reduction outside a given finite set of places. From this, one gets
an irreducibility theorem (cf.\ \ref{sec:IV_21}). The determination of the Lie
algebra of $\Img(\rho_\ell)$ then follows, using the properties of abelian
representations given in chap.~\ref{ch:ii}, \ref{ch:iii}; one has to know that
$\rho_\ell$, if abelian, is locally algebraic, but this is a consequence of the
result of Tate given in chap.~\ref{ch:iii}. The variation of $\Img(\rho_\ell)$
with $\ell$ is dealt with in \S\ref{sec:IV_3}. Similar results for the local
case are given in the Appendix.
