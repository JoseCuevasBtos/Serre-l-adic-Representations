\chapter{\texorpdfstring{$\ell$}{l}-adic representations}%
\label{ch:i}

\section{The notion of an \texorpdfstring{$\ell$}{l}-adic representation}
\subsection{Definition}\label{sec:I_11}
Let $K$ be a field,
\dpage
and let $\sepcl K$ be a separable algebraic closure of $K$.  Let $G =
\Gal(\sepcl K / K)$ be the Galois group of the extension $\sepcl K/K$.
The group $G$, with the Krull topology, is compact and totally disconnected.
Let $\ell$ be a prime number, and let $V$ be a finite-dimensional vector space
over the field $\Q_\ell$ of $\ell$-adic numbers.  The full linear group
$\Aut(V)$ is an $\ell$-adic Lie group, its topology being induced by the
natural topology of $\End(V)$; if $n = \dim(V)$, we have $\Aut(V) \cong \GL(n,
\Q_\ell)$.

\begin{mydef}
An $\ell$-adic representation of $\G$ (or, by abuse of language, of $K$) is a
continuous homomorphism $\rho \colon G \to \Aut(V)$.
\end{mydef}

\begin{obs}
\begin{enumerate}
	\item A \emph{lattice} of $V$ is a sub-$\Z_\ell$-module $T$ which is
		free of finite rank, and generate $V$ over $\Q_\ell$, so that
		$V$ can be identified with $T \otimes_{\Z_\ell} \Q_\ell$.
		Notice that there exists a lattice of $V$ which is stable under
		$\G$. This follows from the fact that $\G$ is compact.

		Indeed,
		\dpage
		let $L$ be any lattice of $V$, and let $H$ be the set of
		elements $g \in \G$ such that $\rho(g)L = L$. This is an open
		subgroup of $G$, and $G/H$ is finite. The lattice $T$
		generated by the lattices $\rho(g)L$, $g \in G/H$, is stable
		under $G$.

		Notice that $L$ may be identified with the projective limit of
		the free $(\Z/\ell\Z)$-modules $T/\ell^m T$, on which $\G$
		acts; the vector space $V$ may be reconstructed from $T$ by $V
		= T \otimes_{\Z_\ell} \Q_\ell$.

	\item If $\rho$ is an $\ell$-adic representation of $\G$, the group $\G
		= \Img(\rho)$ is a closed subgroup of $\Aut(V)$, and hence, by
		the $\ell$-adic analogue of Cartan's theorem (cf.\ 
		\cite[5-42]{28}) $\G$ is itself an $\ell$-adic Lie group.
		Its Lie algebra $\mathfrak{g} = \Lie(\G)$ is a subalgebra of
		$\End(V) = \Lie(\Aut(V))$. The Lie algebra $\mathfrak{g}$ is
		easily seen to be invariant under extensions of finite type of
		the ground field $K$ (cf.\ \cite{24}, 1.2).
\end{enumerate}
\end{obs}

\subsubsection*{Exercises}
\begin{enumerate}
	\item\label{ex:basis_triangular_grp}
		Let $V$ be a vector space of dimension 2 over a field $k$
		and let $H$ be a subgroup of $\Aut(V)$. Assume that $\det(1-h)
		= 0$ for all $h \in H$. Show the existence of a basis of $V$
		with respect to which $H$ is contained either in the subgroup $
		\begin{psmallmatrix}
			1 & * \\
			0 & *
		\end{psmallmatrix}
		$ or in the subgroup $
		\begin{psmallmatrix}
			1 & 0 \\
			* & *
		\end{psmallmatrix}
		$ of $\Aut(V)$.
	\item\label{ex:I11_ex2}
		Let $\rho \colon G \to \Aut(V_\ell)$ be an $\ell$-adic
		representation of $\G$, where $V_\ell$ is a $\Q_\ell$-vector
		space of dimension 2. Assume $\det(1-\rho(s))= 0 \mod\ell$ for
		all $s \in G$. Let $T$ be a lattice of $V_\ell$ stable by $G$.
		Show the existence of a lattice $T'$ of $V_\ell$ with the
		following two properties:
	\begin{enumerate}
		\item $T'$ is stable by $G$.
		\item Either $T'$ is a sublattice of index $\ell$ of $T$ and
			\dpage
			$G$ acts trivially on $T/T'$ or $T$ is a sublattice of
			index $\ell$ of $T'$ and $G$ acts trivially on $T/T'$.
			\label{errata:t't}

			(Apply exercise~\ref{ex:basis_triangular_grp} above to
			$k = F_\ell$ and $V = T/\ell T$.)

	\end{enumerate}
		\item Let $\rho$ be a semi-simple $\ell$-adic representation of
			$G$ and let $U$ be an invariant subgroup of $G$.
			Assume that, for all $x \in U$, $\rho(x)$ is unipotent
			(all its eigenvalues are equal to 1). Show that
			$\rho(x) = 1$ for all $x \in U$. (Show that the
			restriction of $\rho$ to $U$ is semi-simple and use
			Kolchin's theorem to bring it to triangular form.)
		\item Let $\rho \colon G \to \Aut(V_\ell)$ be an $\ell$-adic
			representation of $G$, and $T$ a lattice of $V_\ell$
			stable under $G$. Show the equivalence of the following
			properties:
		\begin{enumerate}
			\item The representation of $G$ in the $F_\ell$-vector
				space $T/\ell T$ is irreducible.
			\item The only lattices of $V_\ell$ stable under $G$
				are the $\ell^n T$, with $n \in \Z$.
		\end{enumerate}
\end{enumerate}

\subsection{Examples}\label{sec:I_12}
\subsubsection{Roots of unity}
Let $\ell \ne \char(K)$. The group $G = \Gal(\sepcl K / K)$ acts on the group
$\mu_m$ of $\ell^m$-th roots of unity, and hence also on $T_\ell(\mu) =
\invlim_{m\in\N} \mu_m$. The $\Q_\ell$-vector space $V_\ell(\mu) = T_\ell(\mu)
\otimes_{\Z_\ell} \Q_\ell$ is of dimension 1, and the homomorphism $\chi_\ell
\colon G \to \Aut(V_\ell) = \Q_\ell^\times$ defined by the action of $G$ on
$V_\ell$ is a 1-dimensional $\ell$-adic representation of $G$. The character
$\chi_\ell$ takes its values in the group of units $U$ of $\Z_\ell$; by
definition
$$ g(z) = z^{\chi_\ell(g)} \quad \text{if } g \in G, \; z^{\ell^m} = 1. $$

\subsubsection{Elliptic curves}
Let $\ell \ne \char(K)$. Let $E$ be an elliptic
curve defined over $K$ with a given rational point $o$. One knows that
\dpage
there is a unique structure of group variety on $E$ with $o$ as neutral
element. Let $E_m$ be the kernel of multiplication by $\ell^m$ in $E(\sepcl
K)$, and let
\[
	T_\ell(E) = \invlim_m E_m, \qquad V_\ell(E) = T_\ell(E)
	\otimes_{\Z_\ell} \Q_\ell.
\]
The Tate module $T_\ell(E)$ is a free $\Z_\ell$-module on which $G =
\Gal(\sepcl K / K)$ acts (cf.\ \cite{12}, chap.\ VII). The corresponding
homomorphism $\pi_\ell \colon G \to \Aut(V_\ell(E))$ is an $\ell$-adic
representation of $G$. The group $G_\ell = \Img(\pi_\ell)$ is a closed
subgroup of $\Aut(T_\ell(E))$, a 4-dimensional Lie group isomorphic to $\GL(2,
\Z_\ell)$. (In chapter IV, we will determine the Lie algebra of $G_\ell$, under
the assumption that $K$ is a number field.)

Since we can identify $E$ with its dual (in the sense of the duality of abelian
varieties) the symbol $(x,y)$ (cf.\ \cite{12}, \textit{loc.\ cit.}) defines
canonical isomorphisms
\[
	\textstyle\bigwedge^2 T_\ell(E) = T_\ell(\mu), \qquad
	\bigwedge^2 V_\ell(E) = V_\ell(\mu).
\]
Hence $\det(\pi_\ell)$ is the character $\chi_\ell$ defined in example 1.

\subsubsection{Abelian varieties}
Let $A$ be an abelian variety over $K$ of dimension $d$. If $\ell \ne
\char(K)$, we define $T_\ell(A)$, $V_\ell(A)$ in the same way as in example 2.
The group $T_\ell(A)$ is a free $\Z_\ell$-module of rank $2d$ (cf.\ \cite{12},
\textit{loc.\ cit.}) on which $G = \Gal(\sepcl K/K)$ acts.

\subsubsection{Cohomology representations}
Let $X$ be an algebraic variety defined over the field $K$, and let $\sepcl X =
X \times_K \sepcl K$ be the corresponding variety over $\sepcl K$. Let $\ell
\ne \char(K)$, and let $i$ be an integer. Using the étale cohomology of
\citeauthor{3}~\cite{3} we let
\[
	H^i(\sepcl X, \Z_\ell) = \invlim_n H^i\big( (\sepcl X)_{\text{ét}},
	\Z/\ell^n\Z \big),
	\qquad H^i_\ell(\sepcl X) = H^i(\sepcl X, \Z_\ell) \otimes_{\Z_\ell}
	\Q_\ell.
\]
\dpage
The group $H^i_\ell(\sepcl X)$ is a vector space over $\Q_\ell$ on which $G =
\Gal(\sepcl K/K)$ acts (via the action of $G$ on $\sepcl X$). It is finite
dimensional, at least if $\char(K) = 0$ or if $X$ is proper. We thus get an
$\ell$-adic representation of $G$ associated to $H^i_\ell(\sepcl X)$; by taking
duals we also get homology $\ell$-adic representations.
Examples 1, 2, 3 are particular cases of homology $\ell$-adic representations
where $i = 1$ and $X$ is respectively the multiplicative group $\GG_m$, the
elliptic curve $E$, and the abelian variety $A$.

\subsubsection*{Exercise}
\begin{enumerate}[(a)]
	\item Show that there is an elliptic curve $E$, defined over $K_0 =
		\Q(T)$, with $j$-invariant equal to $T$.
	\item\label{exr:elliptic_I12_b}
		Show that for such a curve, over $K = \C(T)$, one has $G_\ell =
		\SL(T_\ell(E))$ (cf.\ \citeauthor{10}~\cite{10} for an
		algebraic proof).
	\item Using \ref{exr:elliptic_I12_b}, show that, over $K_0$, we have
		$G_\ell = \GL(T_\ell(E))$.
	\item Show that for any closed subgroup $H$ of $\GL(2, \Z_\ell)$ there
		is an elliptic curve (defined over some field) for which
		$G_\ell = H$.
\end{enumerate}

\section{\texorpdfstring{$\ell$}{l}-adic representations of number fields}
\subsection{Preliminaries}%
\label{sec:I_21}
(For the basic notions concerning number fields, see for instance
\citeauthor{6}~\cite{6}, \citeauthor{13}~\cite{13} or
\citeauthor{44}~\cite{44}.)
Let $K$ be a number field (i.e.\ a finite extension of $\Q$). Denote by $M_K^0$
the set of all finite places of $K$, i.e., the set of all normalized discrete
valuations of $K$ (or, alternatively, the set of prime ideals in the ring
$\mathcal{O}_K$ of integers of $K$).
The \strong{residue field} $k_v$ of a place $v \in M_K^0$
is a finite field with $\numnorm(v) = p_v^{\deg(v)}$ elements, where
\dpage
$p_v = \char(k_v)$ and $\deg(v)$ is the degree of $k_v$ over $F_{p_v}$. The
ramification index $e_v$ of $v$ is $v(p_v)$.

Let $L/K$ be a finite Galois extension with Galois group $G$,
and let $w \in M_L^0$.
The subgroup $D_w$ of $G$ consisting of those $g \in G$ for which $gw = w$ is
the \strong{decomposition group} of $w$. The restriction
of $w$ to $K$ is an integral multiple of an element $v \in M_K^0$; by abuse
of language, we also say that $v$ is the restriction of $w$ to $K$, and we
write $w \mid v$ (\textquote{$w$ divides $v$}). Let $L$ (resp.\ $K$) be the
completion of $L$ (resp.\ $K$) with respect to $w$ (resp.\ $v$). We have
$D_w = \Gal(L_w/K_v)$. The group $D_w$ is mapped homomorphically onto
the Galois group $\Gal(\lambda_w/k_v)$ of the corresponding residue extension
$\lambda_w/k_v$. The kernel of $G \to \Gal(\lambda_w/k_v)$ is the inertia group
$I_w$ of $w$. The quotient group $D_w/I_w$ is a finite cyclic group generated
by the \strong{Frobenius element} $F_w$; we have $F(\lambda) =
\lambda^{\numnorm(v)}$ for all $\lambda \in \lambda_w$.
The valuation $w$ (resp.\ $v$) is called \strong{unramified} if $I_w = \{ 1
\}$. Almost all places of $K$ are unramified.

If $L$ is an arbitrary algebraic extension of $\Q$, one defines $M_K^0$ to be
the projective limit of the sets $M_{L_\alpha}^0$, where $L_\alpha$ ranges
over the finite sub-extensions of $L/\Q$. Then, if $L/K$ is an 
arbitrary Galois extension of the number field $K$, and $w \in M_L^0$, one
defines $D_w$, $I_w$, $F_w$ as before. If $v$ is an unramified place of $K$,
and $w$ is a place of $L$ extending $v$, we denote by $F_v$ the conjugacy
class of $F_w$ in $G = \Gal(L/K)$.

\begin{mydef}
Let $\rho \colon \Gal(\algcl K/K) \to \Aut(V)$ be an $\ell$-adic representation
of $K$, and let $v \in M_K^0$. We say that $\rho$ is unramified at $v$ if
$\rho(I_w) = \{ 1 \}$ for any valuation $w$ of $\algcl K$ extending $v$.
\end{mydef}

If the representation $\rho$ is unramified at $v$, then the
\dpage
restriction of $\rho$ to $D_w$ factors through $D_w/I_w$ for any $w\mid v$;
hence $\rho(F_w) \in \Aut(V)$ is defined; we call $\rho(F_w)$ the
\strong{Frobenius} of $w$ in the representation $\rho$, and we denote it by
$F_{w, \rho}$. The conjugacy class of $F_{w, \rho}$ in $\Aut(V)$ depends only
on $v$; it is denoted by $F_{v, \rho}$. If $L/K$ is the extension of $K$
corresponding to $H = \Ker(\rho)$, then $\rho$ is unramified at $v$ if and only
if $v$ is unramified in $L/K$.

\subsection{\v Cebotarev's density theorem}
\label{sec:I_22}
Let $P$ be a subset of $M_K^0$. For each integer $n$, let $a_n(P)$
be the number of $v \in P$ such that $\numnorm v \le n$. If $a$ is a real number,
one says that $P$ \strong{has density} $a$ if
\[
	\lim \frac{a_n(P)}{a_n(M_K^0)} = a \qquad \text{when}\quad n \to \infty.
\]
Note that $a_n(M_K^0) \sim n/\log(n)$, by the prime number theorem (cf.\
Appendix, or \cite{13}, chap.~VIII), so that the above relation may be
rewritten:
\[
	a_n(P) = a \cdot \frac{n}{\log(n)} + o \left(\frac{n}{\log(n)}\right).
\]
\begin{ex}
A finite set has density $0$. The set of $v \in M_K^0$ of degree $1$ (i.e.\ such
that $\numnorm v$ is prime) has density $1$. The set of ordinary prime numbers
whose first digit (in the decimal system, say) is $1$ has no density.
\end{ex}

\begin{thm}
Let $L$ be a finite Galois extension of the number field $K$, with Galois group
$G$. Let $X$ be a subset of $G$, stable by
\dpage
conjugation. Let $P_X$ has density equal to $\Card(X)/\Card(G)$. 
\end{thm}
For the proof, see \cite{7}, \cite{1}, or the Appendix.

\begin{cor}
For every $g \in G$, there exist infinitely many unramified places $w \in
M_K^0$ such that $F_w = g$.
\end{cor}

For infinite extensions, we have:
\begin{cor}
Let $L$ be a Galois extension of $K$, which is unramified outside a finite set
$S$.
\begin{enumerate}[a)]
\item The Frobenius elements of the unramified places of $L$ are dense in
	$\Gal(L/K)$.
\item Let $K$ be a subset of $\Gal(L/K)$, stable by conjugation. Assume that
	the boundary of $X$ has measure zero with respect to the Haar measure
	$\mu$ of $X$, and normalize $\mu$ such that its total mass is $1$. Then
	the set of places $v \not\in S$ such that $F_v \subset X$ has a density
	equal to $\mu(X)$.
\end{enumerate}
\end{cor}

Assertion (b) follows from the theorem, by writing $L$ as an increasing union
of finite Galois extensions and passing to the limit (one may also use Prop. 1
of the Appendix). Assertion (a) follows from (b) applied to a suitable
neghborhood of a given class of $\Gal(L/K)$.

\subsubsection*{Exercise}
Let $G$ be an $\ell$-adic Lie group and let $X$ be an analytic subset of $G$
(i.e.\ a set defined by the vanishing of a family of analytic functions on $G$).
Show that the boundary of $X$ has measure zero
\dpage
with respect to the Haar measure of $G$.

\subsection{Rational \texorpdfstring{$\ell$}{l}-adic representations}
\label{sec:I_23}
Let $\rho$ be an $\ell$-adic representation of the number field $K$. If $v \in
M_K^0$, and if $v$ is unramified with respect to $\rho$, we let $P_{v,\rho}(T)$
denote the polynomial $\det(1 - F_{v,\rho} T)$.

\begin{definition}
The $\ell$-adic representation $\rho$ is said to be rational (resp.\ integral)
if there exists a finite subset $S$ of $M_K^0$ such that
\begin{enumerate}[(a)]
	\item Any element of $M_K^0 \setminus S$.
\todo[color=pink!40]{{\color{pink}$\heartsuit$} Ver si la notación de eliminar conjunto está bien}
	\item 
\end{enumerate}

\end{definition}

\subsection{Representations with values in a linear algebraic group}
\label{sec:I_24}
Let $H$ be a linear algebraic group defined over a field $K$. If
$k'$ is a commutative $k$-algebra, let $H(k')$ denote the group of points
of $H$ with values in $k'$. Let $A$ denote the coordinate ring (or
``affine ring'') of $H$. An element $f \in A$ is said to be \strong{central} if
$f(xy) = f(yx)$ for any $x, y \in H(k')$ and any commutative $k$-algebra
$k'$. If $x \in H(k')$ we say that the conjugacy class of $x$ in $H$ is
\strong{rational over $k$} if $f(x) \in k$ for any central element $f$ of $A$.

\begin{mydef}
Let $H$ be a linear algebraic group over $\Q$, and let
$K$ be a field. A continuous homomorphism $\rho \colon \Gal(\sepcl K/K) \to H(\Q_\ell)$
is called an $\ell$-adic representation of $K$ with values in $H$.
\end{mydef}
(Note that $H(\Q_\ell)$ is, in a natural way, a topological group and even
an $\ell$-adic Lie group.)

If $K$ is a number field, one defines in an obvious way what it
\dpage
means for $\rho$ to be unramified at a place $v \in M_K^0$; if $w\mid v$, one
defines the Frobenius element $F_{w, \rho} \in H(\Q_\ell)$ and its conjugacy
class $F_{v, \rho}$. We say, as before, that $\rho$ is \strong{rational} if
\begin{enumerate}[(a)]
\item\label{ax:rational_ladic_a}
	there is a finite set $S$ of $M_K^0$ such that $\rho$ is unramified
	outside $S$,
\item\label{ax:rational_ladic_b}
	if $v \notin S$, the conjugacy class $F_{v, \rho}$ is rational over $\Q$.
\end{enumerate}
Two rational representations $\rho$, $\rho'$ (for primes $\ell$, $\ell'$) are said to
be \strong{compatible} if there exists a finite subset $S$ of $M_K^0$ such that $\rho$
and $\rho'$ are unramified outside $S$ and such that for any central 
element $f \in A$ and any $v \in M_K^0 \setminus S$ we have $f(F_{v, \rho}) =
f(F_{v, \rho})$. One defines in the same way the notions of \strong{compatible}
and \strong{strictly compatible systems} of rational representations.

\begin{obs}
\begin{enumerate}
\item If the algebraic group $H$ is abelian, then condition
	\ref{ax:rational_ladic_b} above means that $F_{v, \rho}$ (which is now
	an element of $H(\Q_\ell)$) is rational over $\Q$, i.e.\ belongs to
	$H(\Q)$.
\item Let $V_0$ be a finite-dimensional vector space over $\Q$, and
	let $\GL_{V_0}$ be the linear algebraic group over $\Q$ whose group of
	points in any commutative $\Q$-algebra $k$ is $\Aut(V_0 \otimes_\Q k)$; in 
	particular, if $V_\ell = V_0 \otimes_\Q \Q_\ell$, then
	$\GL_{V_0}(\Q_\ell) = \Aut(V_\ell)$. If $\varphi \colon H \to
	\GL_{V_0}$ is a homomorphism of linear algebraic groups over $\Q$, call
	$\varphi_\ell$ the induced homomorphism of $H(\Q_\ell)$ into
	$\GL_{V_0}(\Q_\ell) = \Aut(V_\ell)$. If $\rho$ is an $\ell$-adic
	representation of $\Gal(\algcl K/K)$ into $H(\Q_\ell)$, one gets by
	composition a linear $\ell$-adic representation $\varphi_\ell \circ
	\rho \colon \Gal(\sepcl K/K) \to \Aut(V_\ell)$. Using the fact that the
	coefficients of the characteristic polynomial are central functions,
	one sees that
	\dpage
	$\varphi_\ell \circ \rho$ is \emph{rational} if $\rho$ is rational ($K$
	a number field).  Of course, compatible representations in $H$ give
	compatible linear representations. We will use this method of
	constructing compatible representations in the case where $H$ is
	abelian (see ch.~\ref{ch:ii}, \ref{sec:II_25}).
\end{enumerate}
\end{obs}

\begin{subappendices}
\section{Equipartition and \texorpdfstring{$L$}{L}-functions}
\subsection{Equipartition}
Let $X$ be a compact topological space and $C(X)$ the Banach
space of continuous, complex-valued, functions on $X$, with its usual norm
$\|f\| = \sup_{x \in X} |f(x)|$. For each $x \in X$ let $\delta_x$ be the Dirac
measure associated to $x$; if $f \in C(X)$, we have $\delta_x(f) = f(x)$.

Let $(x_n)_{n\ge 1}$ be a sequence of points of $X$. For $n \ge 1$, let
\[
	\mu_n = \frac{ \delta_{x_1} + \cdots + \delta_{x_n} }{n}
\]
and
\dapge
let $\mu$ be a Radon measure on $X$ (i.e.\ a continuous linear form on $C(X)$,
cf.\ Bourbaki, Int., chap.~III, \S 1). The sequence $(x_n)$ is said to be
\strong{$\mu$-equidistributed},\index{Equidistribution}
or \emph{$\mu$-uniformly distributed}, if $\mu_n \to \mu$ \emph{weakly} as $n
\to \infty$, i.e.\ if $\mu_n(f) \to \mu(f)$ as $n \to \infty$ for any $f \in
C(X)$. Note that this implies that $\mu$ is positive and of total mass 1.  Note
also that $\mu_n(f) \to \mu(f)$ means that
\[
	\mu(f) = \lim_{n\to\infty} \frac{1}{n} \sum_{i=1}^{n} f(x_i).
\]

\begin{lem}\label{lem:IA_1}
Let $(\varphi_\alpha)$ be a family of continuous functions on $X$ with the
property that their linear combinations are dense in $C(X)$. Suppose that, for
all $\alpha$, the sequence $( \mu_n(\varphi_\alpha) )_{n>1}$ has a limit.  Then
the sequence $(x_n)$ is equidistributed with respect to some measure $\mu$ it
is the unique measure such that $\mu(\varphi_\alpha) = \lim_{n\to\infty}
\mu_n(\varphi_\alpha)$ for all $\alpha$.
\end{lem}

If $f \in C(X)$, an argument using equicontinuity shows that the sequence
$(\mu_n(f))$ has a limit $\mu(f)$, which is continuous and linear in $f$; hence
the lemma.

\begin{prop}
Suppose that $(x_n)$ is $\mu$-equidistributed. Let $U$ be a subset of $X$ whose
boundary has $\mu$-measure zero, and, for all $n$, let $n_U$ be the number of
$m \le n$ such that $x_m \in U$. Then $\lim_{n\to\infty} (n_U/n) = \mu(U)$.
\end{prop}
Let $\Int U$ be the interior of $U$. We have $\mu(\Int U) = \mu(U)$.  Let
$\varepsilon > 0$.  By the definition of $\mu(\Int U)$ there is a continuous
function $\varphi \in C(X)$, $0 \le \varphi \le 1$, with $\varphi = 0$ on $X
\setminus \Int U$ and $\mu(\varphi) \ge \mu(U) - \varepsilon$.  Since
$\mu_n(\varphi) \le n_U/n$ we have
\[
	\liminf_{n\to\infty} \frac{n_U}{n} \ge \lim_{n\to\infty} \mu_n(\varphi)
	= \mu(\varphi) \ge \mu(U) - \varepsilon,
\]
\dpage
from which we obtain $\liminf n_U/n \ge \mu(U)$. The same argument
applied to $X \setminus U$ shows that
$$ \liminf_{n\to\infty} \frac{n - n_U}{n} \ge \mu(X \setminus U). $$
Hence $\limsup_n n_U/n \le \mu(U) \le \liminf n_U/n$, which implies the
proposition.

\begin{ex}
\begin{enumerate}[label=\arabic*., ref=\arabic*]
\item Let $X = [0,1]$, and let $\mu$ be the Lebesgue measure. A sequence
	$(x_n)$ of points of $X$ is $\mu$-equidistributed if and only if for
	each interval $[a, b]$, of length $d > 0$ in $[0,1]$ the number of $m
	\le n$ such that $x_m \in [a, b]$ is equivalent to $dn$ as $n \to
	\infty$.
\item\label{ex:IA1_2}
	Let $G$ be a compact group and let $X$ be the space of conjugacy classes
	of $G$ (i.e.\ the quotient space of $G$ by the equivalence relation
	induced by inner automorphisms of $G$). Let $\mu$ be a measure on $G$;
	its image of $G \to X$ is a measure on $X$, which we also denote by
	$\mu$. We then have:
\end{enumerate}
\end{ex}
\begin{prop}\label{prop:IA_2}
The sequence $(x_n)$ of elements of $X$ is $\mu$-equidistributed if and only if
for any irreducible character $\chi$ of $G$ we have
$$ \lim_{n\to\infty} \frac{1}{n} \sum_{i=1}^{n} \chi(x_i) = \mu(\chi). $$
\end{prop}

The map $C(X) \to C(G)$ is an isomorphism of $C(X)$ onto the
space of central functions on $G$; by the Peter-Weyl theorem, the
\dpage
irreducible characters $\chi$ of $G$ generate a dense subspace of $C(X)$.
Hence the proposition follows from lemma~\ref{lem:IA_1}.

\begin{corp}
Let $\mu$ be the Haar measure of $G$ with $\mu(G) = 1$.
Then a sequence $(x_n)$ of elements of $X$ is $\mu$-equidistributed if and
only if for any irreducible character $\chi$ of $G$, $\chi \ne 1$ we have
$$ \lim_{n\to\infty} \frac{1}{n} \sum_{i=1}^{n} \chi(x_i) = 0. $$
\end{corp}
This follows from Prop.~\ref{prop:IA_2} and the following facts:
\begin{align*}
	\mu(\chi) &= 0 \qquad \text{if $\chi$ is irreducible $\ne 1$} \\
	\mu(1) &= 1.
\end{align*}

\begin{corp}[\citeauthor{46}~\cite{46}]
Let $G = \R/\Z$, and let $\mu$ be the normalized Haar measure on $G$. Then
$(x_n)$ is $\mu$-equidistributed if and only if for any integer $m \ne 0$ we
have
$$ \sum_{n\le N} e^{2\pi mi x_n} = o(N) \qquad (N \to \infty). $$
\end{corp}
For the proof, it suffices to remark that the irreducible characters of $\R/\Z$
are the mappings $x \mapsto e^{2\pi mi x}$ ($m \in \Z$).

\subsection{The connection with \texorpdfstring{$L$}{L}-functions}
Let $G$ and $X$ be as in Example~\ref{ex:IA1_2} above: $G$ a compact group and
$X$ the space of its conjugacy classes. Let $x_v$, $v \in M$, be a family of
elements of $X$, indexed by a denumerable set $M$, and let $v \mapsto \numnorm
v$ be a function on $M$ with values in the set of integers $\ge 2$.
\dpage
We make the following \emph{hypotheses:}
\begin{enumerate}[(1), series=Lfunc_hyp]
\item
	The infinite product $\prod_{v\in M} \frac{1}{1 - (\numnorm v)^{-s}}$
	converges for every $s \in \C$ with $\Re(s) > 1$, and extends to a
	meromorphic function on $\Re(s) > 1$ having neither zero nor pole
	except for a simple pole at $s = 1$.
\item Let $\rho$ be an irreducible representation of $G$, with character
	$\chi$, and put
	\[
		L(s, \rho) = \prod_{v\in M} \frac{1}{\det(1 -
		\rho(x_v)(\numnorm v)^{-s})}.
	\]
	Then this product converges for $\Re(s) > 1$, and extends to a
	meromorphic function on $\Re(s) > 1$ having neither zero nor pole
	except possibly for $s = 1$.
\end{enumerate}
The \emph{order} of $L(s, \rho)$ at $s = 1$ will be denoted by $-c_\chi$. Hence,
if $L(s,\rho)$ has a pole (resp.\ a zero) of order $m$ at $s = 1$, one has
$c_\chi = m$ (resp.\ $c_\chi = -m$).

Under these assumptions, we have:
\begin{thm}
\begin{enumerate}[(a)]
\item The number of $v \in M$ with $\numnorm v \le n$ is equivalent to $n/\log
	n$ (as $n \to \infty$).
\item For any irreducible character $\chi$ of $G$, we have
	$$ \sum_{\numnorm v\le n} \chi(x_v) = c_\chi \, \frac{n}{\log n} +
	o(n/\log n), \qquad (n \to \infty). $$
\end{enumerate}
\end{thm}
The theorem results, by a standard argument, from the theorem of
Wiener-Ikehara, cf.\ \ref{sec:IA_3} below.
Suppose now that the function $v \mapsto \numnorm v$ has the following
property:

\begin{enumerate}[resume*=Lfunc_hyp]
\item\dpage
	There exists a constant $C$ such that, for every $n \in \Z$, the number
	of $v \in M$ with $\numnorm v = n$ is $\le C$.
\end{enumerate}
One may then arrange the elements of $M$ as a sequence
$(v_i)_{i\ge 1}$. so that $i \le j$ implies $\numnorm v_i \le \numnorm v_j$ (in
general, this is possible in many ways). It then makes sense to speak about the
equidistribution of the sequence of $x_v$'s; using (3), one shows easily that
this does not depend on the chosen ordering of $M$. Applying theorem 1 and
proposition 2, we obtain:
\begin{thm}
The elements $x_v$ ($v \in M$) are equidistributed in $X$
with respect to a measure $\mu$ such that for any irreducible character
$\chi$ of $G$ we have
$$ \mu(\chi) = c_\chi. $$
\end{thm}
\begin{cor}
	The elements $x_v$ ($v \in M$) are equidistributed in $X$
	normalized Haar measure of $G$ if and only if $c_\chi = 0$ for every
	irreducible character $\chi \ne 1$ of $G$, i.e., if and only if the
	$L$-functions relative to the non trivial irreducible characters of $G$
	are holomorphic and non zero at $s = 1$.
\end{cor}

\begin{ex}
\begin{enumerate}[series=ex_IA3]
\item Let $G$ be the Galois group of a \emph{finite} Galois extension
	$L/K$ of the number field $K$, let $M$ be the set of unramified places
	of $K$, let $x_v$ be the Frobenius conjugacy class defined by $v \in M$,
	and let $\numnorm v$ be the norm of $v$, cf.\ \S\ref{sec:I_21}.

	Properties (1), (2), (3) are satisfied with $c_\chi = 0$ for all
	irreducible $\chi \ne 1$. This is trivial for (3). For (1), one remarks
	that $L(s,l)$ is the zeta function of $K$ (up to a finite number of
	terms), hence has a simple pole at $s = 1$ and is holomorphic on the
	\dpage
	rest of the line $\Re(s) = 1$, cf.\ for instance
	\citeauthor{13}~\cite{13}, chap.\ VII; for a proof of (2), cf.\
	\citeauthor{1}~\cite[121]{1}.  Hence theorem 2 gives the
	equidistribution of the Frobenius elements, i.e.\ the \v Cebotarev
	density theorem, cf.~\ref{sec:I_22}.

\item Let $C$ be the idèle class group of a number field $K$, and let $\rho$ be
	a continuous homomorphism of $C$ into a compact abelian Lie group $G$.
	An easy argument (cf.\ ch.\ III, 2.2) shows that $\rho$ is almost
	everywhere unramified (i.e., if $U_v$ denotes the group of units at
	$v$, then $\rho(U_v) = 1$ for almost all $v$). Choose $\pi_v \in K$
	with $v(\pi_v) = 1$. If $\rho$ is unramified at $v$, then $\rho(\pi_v)$
	depends only on $v$, and we set $x_v = \rho(\pi_v)$. We make the
	following \emph{assumption:}
	\begin{displayquote}
		\slshape
		\textbf{(*)}
		The homomorphism $\rho$ maps the group $C$ of idèles of
		volume 1 onto $G$.
	\end{displayquote}
	(Recall that the \strong{volume} of an idèle $\vec a = (a_v)$ is
	defined as the product of the normalized absolute values of its
	components $a_v$, cf.\ \citeauthor{13}~\cite{13} or
	\citeauthor{44}~\cite{44}.)

	Then, the elements $x_v$ are \emph{uniformly distributed} in $G$ with
	respect to the normalized Haar measure. This follows from theorem~1 and
	the fact that the $L$-functions relative to the irreducible characters
	$\chi$ of $G$ are Hecke $L$-functions with Grössencharakters; these
	$L$-functions are holomorphic and non-zero for $\Re(s) \ge 1$ if $\chi
	\ne 1$, see \cite{13}, chap.\ VII.
\end{enumerate}
\end{ex}

\begin{obs}
This example (essentially due to Hecke) is given in Lang
(\emph{loc.\ cit.}, ch.\ VIII, \S 5) except that Lang has replaced the condition
(*) by the condition ``$\rho$ is surjective'', which is insufficient. This
led him to affirm that, for example, the sequence $(\log p)_p$ (and also
the sequence $(\log n)_n$) is uniformly distributed modulo 1; however,
\dpage
one knows that this sequence is not uniformly distributed for any
measure on $\R/\Z$ (cf.\ \citeauthor{22}~\cite[179-180]{22}).
\end{obs}

\begin{enumerate}[resume*=ex_IA3]
\item (Conjectural example).
	Let $E$ be an elliptic curve defined over a number field $K$ and let
	$M$ be the set of finite places $v$ of $K$ such that $E$ has good
	reduction at $v$, cf.\ 1.2 and chap.~\ref{ch:iv}.  Let $v \in M$, let
	$\ell \ne p_v$ and let $F_v$ be the Frobenius conjugacy class of $v$ in
	$\Aut(T_\ell(E))$. The eigenvalues of $F_v$ are algebraic numbers; when
	embedded into $\C$ they give conjugate complex numbers $\pi_v$,
	$\bar{\pi}_v$ with $|\pi_v| = (\numnorm v)^{1/2}$.
	We may write then
	\[
		\pi_v = (\numnorm v)^{1/2} e^{i \phi_v}; \quad
		\bar{\pi}_v = (\numnorm v)^{1/2} e^{-i \phi_v} \qquad
		\text{with } 0 \le \phi_v \le \pi.
	\]
	On the other hand, let $G = \SU(2)$ be the Lie group of $2 \times 2$
	unitary matrices with determinant 1. Any element of the space $X$ of
	conjugacy classes of $G$ contains a unique matrix of the form
	\[
		\begin{pmatrix}
			e^{i \phi} & 0 \\
			0 & e^{-i \phi}
		\end{pmatrix}, \qquad 0 \le \phi \le \pi.
	\]
	The image in $X$ of the Haar measure of $G$ is known to be
	$\frac{2}{\pi}\sin^2 \phi \, \ud\phi$.  The irreducible
	representations of $G$ are the $m$-th symmetric powers $\rho_m$ of the
	natural representation $\rho_1$ of degree 2.

	Take now for $x_v$ the element of $X$ corresponding to the angle
	$\phi = \phi_v$ defined above. The corresponding $L$ function,
	relative to $\rho_m$, is:
	\[
		L_{\rho_m}(s) = \prod_{v} \prod_{a=0}^{a=m} \frac{1}{ 1 -
		e^{i(m - 2a)\phi_v} (\numnorm v)^{-s} }.
	\]
	If we put:
	\[
		L_m^1(s) = \prod_{v} \prod_{a=0}^{a=m} \frac{1}{ 1 -
		\pi_v^{m-a} \bar{\pi}_v^a (\numnorm v)^{-s} }
	\]
	\dpage
	we have
	\[
		L_{\rho_m}(s) = L_m^1(s - m/2).
	\]
	The function $L$ has been considered by \citeauthor{36}~\cite{36}. He
	conjectures that $L_m^1$, for $m \ge 1$, is holomorphic and non zero
	for $\Re(s) \ge 1 + m/2$, provided that $E$ has no complex
	multiplication. Granting this conjecture, the corollary to theorem 2
	would yield the uniform distribution of the $x_v$'s, or, equivalently,
	that the angles $\phi_v$ of the Frobenius elements are uniformly
	distributed in $[0, \pi]$ with respect to the measure
	$\frac{2}{\pi}\sin^2 \phi \, \ud\phi$ (``conjecture of
	Sato-Tate'').

	One can expect analogous results to be true for other $\ell$-adic
	representations.
\end{enumerate}

\subsection{Proof of theorem~1}
The logarithmic derivative of $L$ is
\[
	\frac{L'(s)}{L(s)} = -\sum_{\substack{v\ge 1 \\ m\ge 1}}
	\frac{\chi(x_v^m) \log(\numnorm v)}{(\numnorm v)^{ms}},
\]
where $x_v^m$ is the conjugacy class consisting of the $m$-th powers of
elements in the class $x_v$. One sees this by writing $L$ as the product
\[
	\prod_{j, v} \frac{1}{1 - \lambda_v^{(j)}( \numnorm v )^{-s}}
\]
where
\dpage
the $\lambda_v^{(j)}$ are the eigenvalues of $x_v$ in the given representation.
Now the series
\[
	\sum_{\substack{v\ge 1 \\ m\ge 1}} \frac{\log(\numnorm v)}{ |(\numnorm
	v)^{ms}| },
\]
converges for $\Re(s) > 1/2$.  Indeed it suffices to show that
\[
	\sum_v \frac{\log(\numnorm v)}{ (\numnorm v)^\sigma } < \infty
\]
if $\sigma > 1$; but this series is majorized by
\[
	\text{(Constant)} \times \sum_{v} \frac{1}{(\numnorm v)^{\sigma +
	\varepsilon}}, \qquad (\varepsilon > 0).
\]
On the other hand, the convergence for $\sigma > 1$ of the product
\[
	\prod_{v} \frac{1}{1 - (\numnorm v)^{-\sigma}}
\]
shows that
\[
	\sum_{v} \frac{1}{(\numnorm v)^\sigma} < \infty
\]
for $\sigma > 1$; hence our assertion. One can therefore write
\[
	\frac{L'(s)}{L(s)} = -\sum_{v} \frac{\chi(x_v) \log(\numnorm
	v)}{(\numnorm v)^s} + \phi(s)
\] 
where $\phi(s)$ is holomorphic for $\Re(s) > \frac{1}{2}$. Moreover, by
hypothesis,
\dpage
$L'/L$ can be extended to a meromorphic function on $\Re(s) \ge 1$
which is holomorphic except possibly for a simple pole at $s = 1$ with residue
$-c_\chi$. One may then apply the Wiener-Ikehara theorem (cf.\ \cite[123]{13}):
\begin{thm}
	Let $F(s) = \sum_{n=1}^\infty a_n/n^s$ be a Dirichlet series with
	complex coefficients. Suppose there exists a Dirichlet series $F(s) =
	\sum_n a_n^+/n^s$ with positive real coefficients such that
\begin{enumerate}[(a)]
	\item $|a_n| \le a_n^+$ for all $n$;
	\item The series $F^+$ converges for $\Re(s) > 1$;
	\item The function $F$ (resp.\ $F^+$) can be extended to a meromorphic
		function on $\Re(s)\ge 1$ having no poles except (resp.\ except
		possibly) for a simple pole at $s=1$ with residue $c_+ > 0$
		(resp.\ $c$).
\end{enumerate}
	Then
	\[
		\sum_{m\le n} a_n = cn + o(n) \qquad (n \to \infty),
	\]
	(where $c = 0$ if $F$ is holomorphic at $s = 1$).
\end{thm}
One applies this theorem to
\[
	F(s) = -\sum_{v} \frac{\chi(x_v) \log(\numnorm v)}{(\numnorm v)^s},
\]
and we take for $F^+$ the series
\[
	d \sum_{v} \frac{\log(\numnorm v)}{(\numnorm v)^s},
\]
where $d$ is the degree of the given representation $\rho$; this is possible
\dpage
since $\chi(x_v)$ is a sum of $d$ complex numbers of absolute value 1,
hence $|\chi(x_v)| \le d$; moreover, the series
\[
	\sum_{v} \frac{\log(\numnorm v)}{(\numnorm v)^s}
\]
differs from the logarithmic derivative of
\[
	\prod_{v} \frac{1}{1 - (\numnorm v)^{-s}}
\]
by a function which is holomorphic for $\Re(s) > 1/2$ as we saw above.
Hence by the Wiener-Ikehara theorem we have
\[
	\sum_{\numnorm v \le n} \chi(x_v) \log(\numnorm v) = c_\chi n + o(n)
	\qquad (n \to \infty).
\]
Consequently, by the Abel summation trick (cf.\ \cite[124]{13}, Prop.~1),
\[
	\sum_{\numnorm v \le n} \chi(x_v) = c_\chi \frac{n}{\log n} + o(n/\log
	n) \qquad (n \to \infty).
\]
and in particular,
\[
	\sum_{\numnorm v \le n} 1 = \frac{n}{\log n} + o(n/\log n) \qquad (n
	\to \infty).
\]
Hence,
\[
	\frac{ \sum_{\numnorm v \le n} \chi(x_v) }{ \sum_{\numnorm v \le n} 1 }
	\longrightarrow c_\chi \qquad \text{as } n\to \infty,
\]
and we may apply proposition~2 to conclude the proof.
\hfill
q.e.d.

\end{subappendices}
